\label{chap:preface}
\begin{table}[htbp]
	\newcommand{\tabincell}[2]{\begin{tabular}{@{}#1@{}}#2\end{tabular}} %换行指令
	\centering
	\caption{名词列表}
	\renewcommand\arraystretch{1.0}	%设置表格内行间距
	\setlength{\tabcolsep}{8mm}{
	\begin{tabular}{llll}
		\toprule 
		 名词(缩略词)   && 定义 \\
		\midrule
		anterior intraparietal cortex(AIP)   &&前顶叶内皮层  \\
		\midrule
		boold oxygen-level dependent [singal](BOLD)     &&血氧水平依赖   \\
		\midrule
		rostral cingulate motor area(CAMr)     &&头侧扣带运动区   \\
		\midrule
		cinggulate motor areas(CAMs)      &&扣带运动区   \\
		\midrule
		diffusion tensor imaging(DTI)       &&扩散张量成像   \\
		\midrule
		electroencephalography(EEG)       &&脑电图   \\
		\midrule
		frontal eye field(FEF)       &&额叶眼动区   \\
		\midrule
		funtional magnetic resonace imaging(fMRI)       &&功能性核磁共振成像   \\
		\midrule
		antero-dorsal granular area(GrAD)      &&前背颗粒区   \\
        \midrule
        antero-lateral granular area(GrAL)       &&前外侧颗粒区   \\
        \midrule
        dorsal granlar area(GrD)      &&背侧颗粒区   \\
        \midrule
        medial granular area       &&内侧颗粒区  \\
        \midrule
        posterior granular area       &&后颗粒区   \\
        \midrule
        poster-lateral granular area(GrPL)       &&后侧颗粒区   \\
        \midrule
        poster-medial granular area(GrPM)      &&后内侧颗粒区   \\
        \midrule
        ventral granular area(CrV)       &&腹侧颗粒区   \\
        \midrule
        thousand years ago(Ka)      &&千年前   \\
        \midrule
        lateral intraparietal cortex(LIP)       &&顶内沟外侧壁   \\
        \midrule
        million years ago(Ma)       &&百万年前   \\
        \midrule
        mediodorsal nucleus of the thalamus(MD)      &&丘脑(丘脑背侧核)   \\
        \midrule
        medial intraparietal cortex(MIP)      &&内侧顶叶内皮层   \\
        \midrule
        middle superior temporal area(MST)     &&中上颞区   \\
        \midrule
        middle temporal area(MT)       &&中颞区   \\
        \midrule
        orbital frontal cortex(OFC)       &&眶额皮层   \\
        \midrule
        positron emission tomography(PEF)       &&正电子发射断层扫描   \\
        \midrule
        prefrontal cortex(PF)       &&前额叶皮层   \\
        \midrule
        presupplementary motor area(preSMA)       &&前辅助运动区   \\
        \midrule
        receiving operating characteristic(ROC)       &&受试者工作特征   \\
        \midrule
        repetitive transcranial magnetic stimulation(rTMS)      &&重复经颅磁刺激   \\
        \midrule
        primary somatosensory cortex(S1)      &&初级躯体感觉皮层   \\
        \midrule
        second somatosensory cortex(S2)      &&第二躯体感觉皮层   \\
        \midrule
        supplementary eye field(SEF)      &&辅助眼区   \\
        \midrule
        standard error of the mean(SEM)      &&均值的标准误差   \\
        \midrule
        supplementary motor area(SMA)      &&辅助运动区   \\
        \midrule
        superior temporal polysensory area(STP)      &&上颞多感官区   \\
        \midrule
        part of the inferior temporal cortex(TE)     &&颞下皮层的一部分   \\
        \midrule
        caudal part of the inferior temporal cortex(TEO)      &&颞下皮层的尾部   \\
        \midrule
        transcranial magnetic stimulation(TMS)     &&经颅磁刺激   \\
        \midrule
        polysensory superior temporal area(TPO)      &&多感官颞上区   \\
        \midrule
        Wisconsin general testing apparatus(WGTA)      &&威斯康星州通用测试仪器   \\

		\bottomrule  

	\end{tabular}}
    \end{table}%


\chapter{前额叶皮层作为一个整体:从当前环境和事件中产生目标}

\section{概述}
颗粒型PF皮层位于三个信息处理的顶端 层次:一个是当前环境,另一个是目标,还有一个是行为结果。作为背景层次的顶点,PF皮层整合了空间和非空间视觉、触觉和内脏感觉、听觉、味觉和嗅觉的皮层输入,以及来自海马体的输入。作为目标层次的顶点,PF皮层代表了行动的目标,包括具体和抽象目标的序列、集合和类别,它可以通过与前运动皮层的联系影响它们的实现。作为结果层次的顶点,它代表特定食物和液体的所有感官维度,并通过与杏仁核的联系,根据当前的生物需求更新其动机评价。通过整合这三个层次,类人猿的PF皮层可以产生适合当前环境和当前需要的目标。作为一个具体的 作为对拟人灵长类动物在进化过程中所面临的觅食问题的具体适应,它们可以学习生成适合当前环境和当前需求的目标。作为对类人灵长类动物在进化过程中所面临的觅食问题的一种特殊适应,它们可以学会在单一事件的基础上产生目标,从而减少危险事件的数量。事件,从而减少危险或无益的觅食选择的数量。选择。类人灵长类动物比它们的祖先更迅速地解决新的觅食问题 比它们的祖先更快解决新的觅食问题,因为它们新进化的PF区实施了一个快速、通用的学习系统。一个快速的、通用的学习系统,它增强了祖先的 强化学习机制,该机制在动物历史的早期就已演化出来。动物历史上早期进化的强化学习机制。

\section{介绍}
前面的五章把PF皮层拆开了,现在是时候把它重新组装起来了。第3章和第4章分别讨论了内侧PF皮层和眶PF皮层。例如,我们认为,关于当前生物需求的信息通过与杏仁核的连接到达眶侧和内皮层,海马体向内皮层提供关于导航和其他涉及动作的事件的信息。第5章从搜索和注意力的角度解释了尾侧PF皮层的功能,通过与背侧和腹侧视觉流的连接来实现。第6章提出,关于空间、时间和秩序的信息从后顶叶皮层到达前额叶背侧皮层,并有助于基于这些上下文的选择。第七章说,关于视觉和听觉信号的信息从颞叶皮层到达前额叶腹侧皮层,并在其他情境的基础上做出选择。
这种讨论PF皮层的方式可能会造成这样一种印象,即它是作为五个独立的区域运作的。但PF皮层是一个整体,它能这样做是因为它的内在联系允许它整合通过不同途径到达的信息。因此,灵长类动物的PF皮层可以根据对结果和整体环境的总体预测来生成目标。PF皮层的贡献超出了这种深远的整合,但以这种方式将信息聚集在一起的能力是其基本功能的基础之一。
由于PF皮层作为一个整体发挥作用,我们需要一个全面的理论,而第一章提出了这样一个理论的两个要求:展示PF皮层能做大脑其他部分不能做的事情,并解释为什么它的连接使它能够以这种方式发挥作用。像以前一样,我们从联系开始。

\section{联系}
第3-7章每个章节都有一个部分强调了PF皮层的一个部分的连接。我们在这里总结总体模式。
皮质和杏仁核
1、颗粒状PF皮层接收来自视觉、听觉、体感、嗅觉、味觉、嗅觉和内脏皮层的信息。因此,灵长类动物的PF皮层有相对直接的来自距离感受器的输入,如视觉和听觉感受器,以及传递特定行为结果的输入,如食物和液体的味觉、嗅觉、视觉和感觉。因此,灵长类动物的PF皮层对特定的行为结果有一个强大的、高维的表征,特别是包括它们的视觉属性。它还具有复杂的视觉和听觉表征,可以指导觅食或社会选择。视觉标志反映了灵长类动物的一些进化进步,如中央凹和三色视觉。其他哺乳动物的PF皮层和灵长类动物新皮层的其他部分至少缺乏这些特性中的一些。
2、


\section{定义和术语}


\section{指纹}

\subsection{损伤和激活}

\subsection{损伤和活动}

\subsection{活动和激活}




\subsection{结论}



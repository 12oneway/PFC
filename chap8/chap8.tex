\chapter{前额叶皮层作为一个整体:从当前环境和事件中产生目标}
概述\\


颗粒型PF皮层位于三个信息处理的顶端 层次:一个是当前环境,另一个是目标,还有一个是行为结果。作为背景层次的顶点,PF皮层整合了空间和非空间视觉、触觉和内脏感觉、听觉、味觉和嗅觉的皮层输入,以及来自海马体的输入。作为目标层次的顶点,PF皮层代表了行动的目标,包括具体和抽象目标的序列、集合和类别,它可以通过与前运动皮层的联系影响它们的实现。作为结果层次的顶点,它代表特定食物和液体的所有感官维度,并通过与杏仁核的联系,根据当前的生物需求更新其动机评价。通过整合这三个层次,类人猿的PF皮层可以产生适合当前环境和当前需要的目标。作为一个具体的 作为对拟人灵长类动物在进化过程中所面临的觅食问题的具体适应,它们可以学习生成适合当前环境和当前需求的目标。作为对类人灵长类动物在进化过程中所面临的觅食问题的一种特殊适应,它们可以学会在单一事件的基础上产生目标,从而减少危险事件的数量。事件,从而减少危险或无益的觅食选择的数量。选择。类人灵长类动物比它们的祖先更迅速地解决新的觅食问题 比它们的祖先更快解决新的觅食问题,因为它们新进化的PF区实施了一个快速、通用的学习系统。一个快速的、通用的学习系统,它增强了祖先的 强化学习机制,该机制在动物历史的早期就已演化出来。动物历史上早期进化的强化学习机制。

\section{介绍}


\section{目的}

\section{定义和术语}


\section{指纹}

\subsection{损伤和激活}

\subsection{损伤和活动}

\subsection{活动和激活}




\subsection{结论}



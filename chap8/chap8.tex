\chapter{前额叶皮层作为一个整体:从当前环境和事件中产生目标} \label{chap:chap8}

\section{概述}
颗粒型PF皮层位于三个信息处理的顶端 层次:一个是当前环境,另一个是目标,还有一个是行为结果。作为背景层次的顶点,PF皮层整合了空间和非空间视觉、触觉和内脏感觉、听觉、味觉和嗅觉的皮层输入,以及来自海马体的输入。作为目标层次的顶点,PF皮层代表了行动的目标,包括具体和抽象目标的序列、集合和类别,它可以通过与前运动皮层的联系影响它们的实现。作为结果层次的顶点,它代表特定食物和液体的所有感官维度,并通过与杏仁核的联系,根据当前的生物需求更新其动机评价。通过整合这三个层次,类人猿的PF皮层可以产生适合当前环境和当前需要的目标。作为一个具体的 作为对拟人灵长类动物在进化过程中所面临的觅食问题的具体适应,它们可以学习生成适合当前环境和当前需求的目标。作为对类人灵长类动物在进化过程中所面临的觅食问题的一种特殊适应,它们可以学会在单一事件的基础上产生目标,从而减少危险事件的数量。事件,从而减少危险或无益的觅食选择的数量。选择。类人灵长类动物比它们的祖先更迅速地解决新的觅食问题 比它们的祖先更快解决新的觅食问题,因为它们新进化的PF区实施了一个快速、通用的学习系统。一个快速的、通用的学习系统,它增强了祖先的 强化学习机制,该机制在动物历史的早期就已演化出来。动物历史上早期进化的强化学习机制。

\section{介绍}
前面的五章把PF皮层拆开了,现在是时候把它重新组装起来了。第3章和第4章分别讨论了内侧PF皮层和眶PF皮层。例如,我们认为,关于当前生物需求的信息通过与杏仁核的连接到达眶侧和内皮层,海马体向内皮层提供关于导航和其他涉及动作的事件的信息。第5章从搜索和注意力的角度解释了尾侧PF皮层的功能,通过与背侧和腹侧视觉流的连接来实现。第6章提出,关于空间、时间和秩序的信息从后顶叶皮层到达前额叶背侧皮层,并有助于基于这些上下文的选择。第七章说,关于视觉和听觉信号的信息从颞叶皮层到达前额叶腹侧皮层,并在其他情境的基础上做出选择。
这种讨论PF皮层的方式可能会造成这样一种印象,即它是作为五个独立的区域运作的。但PF皮层是一个整体,它能这样做是因为它的内在联系允许它整合通过不同途径到达的信息。因此,灵长类动物的PF皮层可以根据对结果和整体环境的总体预测来生成目标。PF皮层的贡献超出了这种深远的整合,但以这种方式将信息聚集在一起的能力是其基本功能的基础之一。
由于PF皮层作为一个整体发挥作用,我们需要一个全面的理论,而第一章提出了这样一个理论的两个要求:展示PF皮层能做大脑其他部分不能做的事情,并解释为什么它的连接使它能够以这种方式发挥作用。像以前一样,我们从联系开始。

\section{联系}
第3-7章每个章节都有一个部分强调了PF皮层的一个部分的连接。我们在这里总结总体模式。

皮质和杏仁核

1、颗粒状PF皮层接收来自视觉、听觉、体感、嗅觉、味觉、嗅觉和内脏皮层的信息。因此,灵长类动物的PF皮层有相对直接的来自距离感受器的输入,如视觉和听觉感受器,以及传递特定行为结果的输入,如食物和液体的味觉、嗅觉、视觉和感觉。因此,灵长类动物的PF皮层对特定的行为结果有一个强大的、高维的表征,特别是包括它们的视觉属性。它还具有复杂的视觉和听觉表征,可以指导觅食或社会选择。视觉标志反映了灵长类动物的一些进化进步,如中央凹和三色视觉。其他哺乳动物的PF皮层和灵长类动物新皮层的其他部分至少缺乏这些特性中的一些。

2、PF皮层还接收来自海马体和海马复合体其他部分的直接和间接输入。海马体、下托和内嗅皮层都与内侧PF皮层有直接联系。海马体在导航和事件记忆中发挥作用,尤其是对于嵌入其空间和时间背景中的事件和对象。与PF皮层一样,海马复合体接收来自所有感觉模式的输入,并与杏仁核紧密相连,传达行为的某些方面结果。但是,尽管PF皮层通过运动前区域有直接输出,但海马体对这些区域的直接访问较少,缺乏PF皮层所具有的那种内在联系,也没有PF皮层所拥有的那种直接、特定的结果信息((详情见第4章)。

3、杏仁核与PF皮层的许多部分有着紧密的联系(见图3.3)。后顶叶皮层等区域几乎没有这种联系。一些运动前区与杏仁核有联系,但它们很稀疏(Avendaño等人,1983年)。PF皮层和杏仁核的连接在根据动物的当前状态更新行为结果的动机评估中发挥作用。因此,PF皮层能够以后顶叶和运动前皮层所没有的方式代表更新的估值。

4、PF皮层直接或间接投射到内侧和外侧前运动区域,从而可以为这些区域提供运动目标。前运动皮层的嘴侧部分与PF皮层有着广泛的联系,但尾侧部分也从PF皮层获得信息,尽管不那么直接。这些连接在很大程度上排除了腿部和脚部的运动表示。与后肢表现相反,前肢的特化表现在背侧前运动皮层的喙部(Tachibana等人,2004年)、腹侧前运动皮质(He等1993年)、前SMA(Luppino等人,1991年)和头侧扣带运动区(He等人,1995年)。正如第2章所解释的,这种前肢偏向反映了灵长类动物以后肢为主的运动形式,这使手可以自由发挥其他功能。通过与运动前区域的连接,PF皮层在伸手、抓握和操纵方面发挥着优先作用,而不是运动。这些联系促进了目标的实现,例如抓住物体或到达某个地方。

5、PF皮层也有控制注意力和搜索功能的连接,包括对应于显性注意力的眼球运动(详情第5章)。例如,尾部PF皮层既有到脑干动眼神经核的直接投射,也有通过上丘和基底神经节的间接投射。第2章指出,灵长类动物的大多数颗粒PF皮层具有强烈的皮质顶盖投射(Leichnetz等人,1981年)。PF皮层还向顶叶和颞叶发送投射,介导对背侧和腹侧视觉流以及其他感觉模式的自上而下的注意力。

6、灵长类动物PF皮层的各个部分之间有着广泛的联系。这些预测已在其他地方详细记录(Barbas 1988;Carmichael和Price 1995b;Barbas等人1999;Petrides和Pandya 19992002a;Price 1999)。来自PF皮层外部的任何输入都可以在两个突触步骤内到达PF皮层的任何部分(Averbeck$\&$Seo,2008)。PF皮层不仅接收大量输入,而且无论信息最初到达哪里,它都可以快速组合这些输入。

除了与皮层的其他部分和杏仁核的连接外,灵长类动物的PF皮层还与幽闭、基底神经节、丘脑、中脑中的多巴胺能神经元和小脑有连接。接下来的五节依次介绍这些内容。

屏状核

幽闭与PF皮层有相互联系(Tanné-Gariepy等人,2002年),也通过丘脑的背中(MD)核投射到它(Erickson等人2004 ). 它与皮层的其余部分也有类似的联系。来自几个皮层区域的投影汇聚在一个特定的幽闭区,每个幽闭区都连接到额叶的几个部分,包括PF皮层(Tanné-Gariepy等人2002 ). 这种连接模式表明幽闭可能达到一定程度的整合。然而,它似乎缺乏PF皮层特有的广泛的内在联系。

基底节

与大多数大脑皮层一样,PF皮层向基底神经节发送一个沉重的投射,靶向其输入结构纹状体。尽管大部分(如果不是全部的话)大脑皮层都有大量的输入,但基底神经节的输出似乎集中在额叶,尽管一些输出也流向了后顶叶(Clower等人,2005年)和颞叶皮层(Middleton和Strick,1996年)。这种组织表明,PF皮层和基底神经节之间的联系也可能有助于其在整合信息方面的作用。然而,它对这种整合的贡献方式仍然存在争议。基底神经节缺乏能够整合其各个部分信息的远距离内在联系。主流观点强调皮层-基底神经节环的平行组装,重叠最小(Alexander$\&$Crutcher 1990;Nakano 2000)。图8.1描述了其中一些环,包括一个运动前区(SMA)和几个PF区。Middleton和Strick(19942000) 得出结论,总的来说,涉及PF皮层的环在解剖学上与涉及运动前区域的环不同。如果这是真的,基底神经节组织的这一特征表明大多数整合发生在皮层水平,尽管纹状体和黑质纹状体投射的某些方面可以提供一些整合能力。他们这样做可能是因为一种叫做向上螺旋的组织特性(Haber等人,2000年)。黑质纹状体突起不仅回到为黑质的特定部分提供纹状体的环,而且还回到相邻的环。

丘脑

PF皮层和所有其他皮层区域一样,与丘脑有相互联系。它的主要丘脑连接是与MD核。例如,MD的多形部分接收来自上丘的输入(Russchen等人,1987;Erickson等人,2004),并投射到尾部PF皮层,而后者又投射到上丘(Fries 1984)。这些联系有助于引起明显和隐蔽的注意(第5章)。同样,内侧大细胞MD核投射到眶PF皮层(Ray$\&$Price 1993),并接收来自杏仁核的输入(Russchen等人,1987)。

图片及其说明

所以研究人员一点过也不惊讶于MD大细胞分裂的病变具有与眶PF皮层或杏仁核病变相似的影响(Mitchell等人,2007年)(见第4章)。Izquierdo和Murray(2010)已经证明了在强化物贬值任务中,大细胞MD核和眶PF皮层的功能相互作用。

多巴胺能中脑

PF皮层的一些输入来自中脑的多巴胺能神经元。正如第3章所提到的,这些细胞提供了奖励预测误差信号(Schultz 1998)。关于这种信号的大多数理论工作都集中在它作为纹状体教学信号的作用上,但其中一些细胞也直接投射到PF皮层和其他区域(Gaspar等人,1992年),它们可以在皮层和纹状体水平上促进学习(Miller和Buschman,2008年)。

小脑

就像皮层-基底神经节环一样,皮层和小脑在解剖学环中相互连接。这些环涉及脑桥基底核和丘脑(Houk\&Wise1995)。在前额叶区域中,尾侧PF皮层和背侧PF皮层向桥基底核发送最大的投射,只有少量的桥皮质投射来自腹侧或眶侧PF皮层(Schmahmann\&Pandya,1997;Glickstein和Doron,2008年)。Middleton和Strick(19982001)已经表明,皮层和小脑之间的环路包括中外侧PF皮层(区域46)和背内侧PF皮层(区9)。

摘要

PF皮层连接的简要草图从第3-7章中提取了一些关键点,这些关键点涉及每个区域的连接。我们强调PF皮层中信息的汇聚和整合,就像我们之前的许多其他结构一样。但第3章至第7章也具体说明了灵长类动物PF皮层如何做大脑其他部分无法做的事情。这些其他结构中的一些缺乏足够的内在连接,另一些缺乏关于特定结果更新估值的输入,而其他人则无法直接进入运动前皮层。
PF皮层接收各种各样的输入,其内在联系可以快速整合,并对动作产生直接影响。第3章至第7章提供了证据,证明其细胞编码多种信息的连词。下一节将这些财产放在信息处理层次结构的上下文中。


\section{定义和术语}


\section{指纹}

\subsection{损伤和激活}

\subsection{损伤和活动}

\subsection{活动和激活}




\subsection{结论}



\chapter{结论}
这本书提出了关于灵长类动物前额叶皮层基本功能的方案.

\section{摘要}
本章将我们的建议与文献中的其他建议进行比较,并通过五个测试对每个建议进行评估:(1)该建议是否考虑到了PF皮层的进化史?(2)它是否解释了连接解剖学如何允许PF皮层执行所提议的功能?(3)它是否明确了PF皮层的功能与其他区域的功能有何不同?(4)是否与PF皮层的广泛发现相一致?(5)它的陈述是否精确到足以被测试?最后,我们提出一些测试我们建议的方法。
\section{介绍}
第一章阐述了五个目标,我们现在可以更全面地阐述这些目标:
\par
1.说颗粒状PF皮层让灵长类动物做了他们的祖先和其他哺乳动物做得效率较低的事情,如果有的话,以及说类人猿灵长类动物颗粒状PF皮层的扩张让他们比其他灵长类动物做得更好。
\par
2.说颗粒状PF皮层的轴突连接如何允许的,而不是其他区域,提供这些优势。
\par
3.作为一个整体解释粒状PF皮层的基本功能,并解释它的功能与大脑其他部分的功能有何不同。
\par 
4.在复杂的认知任务中,这个功能是如何解释在人类PF皮层中观察到的激活的。
\par 
5.解释我们的建议与文献中其他建议的不同之处,并告诉读者什么样的观察可以反驳它。
\par 
现在是时候评估一下我们在多大程度上实现了这些目标。第二章讨论第一个问题。它研究了PF皮层的进化,包括一些导致灵长类动物进化过程中特定进展的选择压力。我们认为,颗粒状PF皮层的进化是分阶段发生的,这些阶段伴随着其他进步,如视觉和手功能的发展。早期的前辈们的眼睛是向前看的,他们采用了新的移动、抓取和用手喂嘴的方式。在这些动物中出现了第一个颗粒状的PF区域,即PF尾部皮层和颗粒状的OFC。背侧、腹侧和极侧的PF皮层在类人猿灵长类动物中出现得更晚,因为它们变得更大,不得不与它们喜欢的食物严重短缺作斗争,更不用说竞争和捕食的威胁了。
\par 
第三-七章讨论第二个目标。这些章节解释了灵长类PF皮层各主要区域的连接如何允许它做它所做的事情,以及为什么只有它才能执行它的功能。我们逐个区域检查PF皮层,因为它的连接因区域而异:
\par 
1.第三章指出了内侧PF皮层与海马体、杏仁核和内侧前运动区之间的联系,这使得它能够更好地使用“内部”信号来指导在行动和行动规则之间的觅食选择,包括评估它们与所涉及的努力成本相关的当前价值。
\par 
2.第四章强调,眶侧PF皮层的连接使其能够更好地利用外部信号来指导觅食选择,在物体之间进行选择。眶PF皮层与许多感觉区域有联系,包括视觉、躯体感觉、味觉、嗅觉和内脏皮层。这些连接允许眶PF皮层发展有关特定行为结果的高维信息连接,并强调其视觉特征。我们认为,基于单个事件,细粒度OFC将特定结果分配给似乎导致它的特定选择。与杏仁核的相互联系根据当前需求提供了更新的结果评估。
\par 
3.第五章重点介绍了注意力和搜索功能。尾侧PF皮层从视觉区域接收到的连接传递了来自低阶和高阶视觉的信号,包括背侧和腹侧视觉流。基于与这些区域的皮质皮质连接以及控制脑干动眼肌核的皮质投射,尾侧PF皮层可以将显性和隐性注意力引导到潜在的行动目标上。我们认为,尾侧PF皮层(第8区),包括额叶眼场(FEF),参与了对学习产生的目标的搜索——目标导向的注意力——而不是反射性或刺激驱动的注意力。我们不认为FEF是眼球运动区或运动前区,而是将其视为前额叶皮层的一部分:一个将注意力导向有学习价值的物体和地方的部分,包括隐蔽注意力和显性注意力(以眼球运动的形式)。这样,它加强了中央凹和中央凹外信息的处理。
\par 
4.第六章强调了由背侧PF皮层产生目标,部分基于与后顶叶皮层的连接。这些投影提供了有关视觉事件的顺序、时间和位置的信息,这些信息构成了当前行为环境的重要部分。目标的产生也依赖于眶PF皮层关于与这些目标及其当前值相关的结果的信息。前额叶皮层的背侧提供了一种机制,可以消除以前事件记忆引起的干扰,它似乎是通过前瞻性地编码当前目标来做到这一点的,至少在一定程度上是这样。与前发动机区域的连接为实现这些目标提供了一条途径。
\par 
5.第七章提出,腹侧前额叶皮层根据视觉或听觉环境产生目标。它之所以能发挥这一功能,是因为它与下颞叶和上颞叶皮层、眶前皮层和杏仁核有联系。这些联系为它提供了视觉和听觉信号、预测的结果,以及根据当前生物需求对该结果的估值。符号由基本特征和整体对象之间的特征结合的中间层次组成。腹侧前额叶皮层使用同样的连接来应用抽象的规则和策略,从而将以前的经验转移到新的行为问题上。
\par 
在以这种方式将PF皮层拆开后,第八章将其重新组装起来。它将PF皮层作为一个整体来研究,并最终实现了我们的第三个目标:一个关于灵长类PF皮层基本功能的具体建议,重点是在灵长类中进化的区域。我们提出,颗粒状PF皮层生成的目标适合于当前环境和当前需求,并且它可以在单个事件的基础上做到这一点。因此,类人猿可以根据一个或几个经验解决广泛的问题,并且可以避免祖先强化学习机制中固有的许多错误。我们提出,在灵长类动物历史上的特定时间和地点(第2章和第8章),为了应对特定的适应压力,粒状PF区域进化为实现一种新的通用学习机制,这一机制增强了祖先的通用学习系统。祖先系统通过加强反馈来调整关联的强度,从而控制自动行为;灵长类动物解决问题的方式包括对行为的注意控制,以及更少的错误。
\par 
第9章讨论我们的第四个目标:解释在人类认知过程中观察到的基本PF功能是如何解释大脑激活的。它强调了重新再现的力量。因此,人类的PF皮层可以重新表示知觉状态、他人的意图和精神状态以及关系之间的关系。第9章还表明,人类的PF皮层阐述了PF皮层在其他类人猿中的功能;就像猴子的颗粒状PF皮层允许它们通过快速学习和抽象策略来避免错误一样,人类的PF皮层允许我们通过指令、模仿和心理试错行为来避免错误。
\par 
本章的剩余部分将讨论我们的第五个也是最后一个目标:将我们对灵长类动物PF皮层的描述与其他文献进行比较,并提出一些测试方法。因此,接下来的两个部分解释了其他的建议要么缺乏我们提供的进化视角,只适用于灵长类PF皮层的一部分,没有解释为什么它独特的连接组合解释了它的功能,没有说明灵长类PF皮层执行了哪些其他大脑区域不能执行的功能,未能解释PF皮层贡献的广泛行为,或者未能产生可验证的假设。
\par 
我们将可供选择的建议分为两组,分别在不同的部分进行讨论:一组主要依赖猴子的结果,另一组主要依赖人类的证据。当然,前一组的支持者也试图将其推广到人类,后一组的支持者也提到了他们从猴子身上得到的证据来支持他们的观点。所以我们这样划分只是为了方便,有时,当一个理论同时涉及猴子和人类的研究时,我们会在两个部分讨论。
\par 
我们遵循Wood和Grafman(2003)在表格中列出各种理论,并试图根据一套标准来评估它们。我们使用了五个标准,这表明一个成功的PF皮层理论应该:
\par 
1.结合PF皮层的进化史,特别是在灵长类动物中颗粒状PF皮层的出现和在高级类人猿中新颗粒状区域的出现:历史测试。
\par 
2.解释为什么PF皮层的连接解剖使其有可能执行所提议的功能:解剖测试。
\par 
3.识别PF皮层的特定功能,与大脑的其他部分形成对比:特异性测试。
\par 
4.考虑可用数据的广泛范围:通用性检验。
\par 
5.精确到可以被可行的观察所检验:可证伪性检验。
\section{基于猴子证据的理论}
我们自己的建议主要依赖于猴子的证据,因此我们首先考虑这类观点。在表10.1中,"X "表示我们认为列在该行的理论行的理论没有通过特定的测试。问号('?')表示该理论通过了测试、但我们不确定它是否成功地做到了这一点。当用于可证伪性检验时,"?"表示尽管该理论的某些版本没有通过,但其他版本可能通过、即使还没有人把该理论推向这个方向。破折号('-')意味着该理论理论根本不涉及该测试。
\par 
不可避免的是,填写表格需要主观的判断。考虑一下历史测试。如果一个理论对进化论只字不提,就可以认为它没有通过这个测试。然而,我们有时可以看到某个理论可以解决这个问题的方法。 然而,有些理论与进化论的观点根本不一致。一个理论没有通过历史当一个理论把非灵长类物种所共有的一些行为能力假设为灵长类PF皮层的基本功能时,它就不能通过历史的检验。然而,有些理论与进化论的观点根本不一致。一个理论没有通过历史 当一个理论把非灵长类物种所共有的一些行为能力假设为灵长类PF皮层的基本功能时,它就不能通过历史的检验。
\par 
我们也意识到,这些测试并不完全独立。例如,解剖学测试是通过评估一个理论是否具体说明了PF皮层的连接是如何让它做什么的 允许它做它所做的事情。特异性测试评估的是其他区域是否也能做同样的事情。同样的事情,在这样做的时候,我们有时会考虑到连接的问题。
\par 
我们首先讨论了工作记忆理论与所有五个标准的关系 因为它的影响非常大。读者会发现,许多相同的观点 也适用于其他理论。因此,为了避免在讨论这些其他观点时出现重复、我们集中讨论特定的优势或特定的弱点,而不一定要处理所有的测试。
\subsection{工作记忆}
第5章和第6章回顾了导致关注工作记忆是PF皮层的主要功能(如果不是唯一功能的话)的结果(Goldman-Rakic 1987)。多年来,这一观点一直主导着文献,支持者关注的是回溯性的感官记忆。像许多其他关于PF皮层的理论一样,它主要来源于少数几个任务,在这种情况下主要是延迟反应和延迟交替任务。我们理解这种强调,考虑到PF皮层损伤对这些任务造成的严重损害(第6章)。但我们认为工作记忆理论未能通过一个成功理论的许多测试,因此我们现在依次讨论这些缺陷。
\par 
在考虑工作记忆理论如何面对历史考验时,我们要承认普鲁斯和戈德曼-拉基奇(1991a, b)的开创性贡献。尽管戈德曼-拉基奇是工作记忆理论的强大而坚定的支持者,但她与普鲁斯发表的比较研究提供了一个关键的见解,彻底颠覆了该理论。猴子的典型工作记忆任务所依赖的区域,是颗粒状PF皮层的一部分,在非哺乳动物或丛林婴儿中都不存在。然而,这些动物拥有强大的工作记忆能力。
\par 
因此,从比较的角度来看,工作记忆理论是没有意义的。第一章解释了老鼠可以完成需要工作记忆的任务,如径向臂迷宫,在本章的后面我们会更详细地讨论这些争论。然而,像其他非灵长类哺乳动物一样,老鼠缺乏颗粒状的PF皮层。因此,人们必须假设,一旦在类人猿类灵长类动物中出现粒状PF皮层,它就“接管”了其他区域以前执行的功能。这种想法既不朴素,也不可信。老鼠有着复杂的大脑,能够解决工作记忆任务带来的简单问题,这并不奇怪。需要解释的发现是,有中外侧前额叶皮层病变的猴子无法解决这些简单的问题。我们稍后再讨论那个话题。
\par 
第二个测试是解剖学测试,要求工作记忆理论解释灵长类动物PF皮层的连接如何导致其独特的功能。在她对工作记忆理论的主要介绍中,戈德曼-拉基奇(1987)对当时人们所知的PF皮层的连接进行了广泛的回顾。在Wilson etal .(1993)中,她在比较PF皮层背侧和腹侧的功能时,也提到了顶叶和颞叶的连接。
\par 
根据最近的证据,我们建议重新定义这种区别,即后顶叶皮层不仅提供空间,而且还提供时间和其他输入到背侧PF皮层(第6章)。但我们同意与颞叶和后顶叶皮层的连接是PF皮层功能的一个重要方面。最后,Goldman-Rakic还讨论了pf皮质的细胞结构如何支持工作记忆(Goldman-Rakic1995;Constantinidis et al.2001),尽管这些想法并没有直接针对解剖测试。
\par 
第三个测试是特异性测试,正如所阐明的那样,工作记忆理论未能说明PF皮层与后顶叶皮层有何不同。这一理论的支持者非常强调在PF皮层中存在具有延迟期活动的细胞。然而,如Chafee和Goldman-Rakic(1998)等人所示,后顶叶皮层也存在延迟期活动的细胞。工作记忆理论的支持者可以提出后顶叶皮层的记忆编码依赖于PF皮层,他们可以通过指出冷却后顶叶皮层导致后顶叶皮层延迟相关活动减少的事实来支持这一观点(Chafee和Goldman-Rakic 2000)。但是,同样的实验表明,冷却后顶叶皮层对PF皮层的活跃度也有类似的影响,因此,这种论点是没有说服力的。如果修正该理论,提出工作记忆依赖于前额叶皮层和后顶叶皮层之间的信息循环,那么它就无法通过特异性测试。
\par 
第四个检验是一般性检验,它评估理论是否能解释所有可靠的数据。表8.1列出了任务和损伤效应的选择性列表,可以看出工作记忆理论无法解释许多已发表的结果。例如,腹侧PF皮层病变的猴子在去不去辨别任务(Iversen和Mishkin 1970)、同时匹配到前样例任务(Rushworth et al. 1997a)和条件视觉运动任务(Bussey et al. 2001)中表现出损伤。这些任务在提示的呈现和随后的动作之间都没有延迟期。然而,很明显,戈德曼-拉基奇和她的同事(Wilson et al. 1993)希望工作记忆理论至少适用于PF皮层的整个外侧表面,如果不是整个PF皮层的话。
\par 
该理论也未能解释在PF皮层中发现的许多细胞类型(表8.2)。正如其支持者所阐述的那样,工作记忆理论在很大程度上依赖于所谓记忆时期的细胞活动。在简单的任务中,如动眼肌延迟反应任务,许多细胞似乎对空间位置的记忆进行编码。当然,这一发现与理论是弱一致的(Funahashi et al. 1989)。但是,一项更仔细的对照研究表明,只有少数PF细胞编码一个被记住的位置,而不是一个被关注的位置(Lebedev et al. 2004)。因此,工作记忆理论也未能通过一般性检验。
\par 
此外,如果该理论适用于整个PF皮层,它在其他方面也未能通过一般性检验。Tsujimoto等人(2010)研究了猴子执行动眼力延迟反应任务时极缘PF皮层的细胞活动。在延迟期间,他们没有发现编码记忆地点或任何其他东西的活动。在Genovesio等人(2011)对中外侧和尾侧PF皮层的研究中,猴子判断相对距离,很少有细胞编码记忆位置,仅比预期的细胞多2 - 5%
\par 
最后一个考验是,一个有效的理论必须是可证伪的。在这里,我们输入一个关于工作记忆理论的查询,因为它可以用许多不同的方式表述。如果这个词仅仅是指在一段时间内记住先前线索的能力,那么它是可证伪的,是错误的。但是,正如第一章所解释的那样,巴德利和希契(1974)首次提出的工作记忆理论包括一个“中央执行人员”,而戈德曼•拉基奇(1998)在她的工作记忆理论的制定中也包括了中央执行人员。在这种情况下,没有任何观察能真正检验这一理论,因为几乎所有有趣的任务都涉及某种中央执行功能。
\par 
由于工作记忆理论未能通过通用性、特异性和历史检验,我们得出结论,工作记忆不是PF皮层的基本功能。
\subsection{跨时间偶然性}
Fuster(2008)提出,PF皮层在延迟期间整合信息方面发挥作用。他称这种关系为跨时间偶然性,因为它涉及跨越时间间隔整合信息。例如,对于延迟响应任务,正确的选择取决于在延迟期间之前发生的事件。所以跨时间整合和层次的概念抓住了PF皮层功能的这一方面。
\par 
在他的书中,Fuster(2008)回顾了他所看到的PF皮层的进化史。然而,他只处理了我们在第二章中提出的几个问题,我们不同意他的论点的关键方面。例如,福斯特认为啮齿动物、食肉动物和灵长类动物的PF皮层是同源的,这是我们所反对的观点。最重要的是,Fuster并没有具体说明PF皮层的各个部分赋予了在它们第一次出现时进化出这些结构的物种的优势。所以福斯特的理论没有通过历史的检验。
\par 
Fuster(2008)也回顾了PF皮层的连接。他认为PF皮层位于感知-行动循环的顶端,这一观点与我们关于层次结构的观点主要在特异性程度和术语上有所不同。因此,在这方面,他的理论解释了为什么PF皮层可以执行它所做的功能(解剖测试),而为什么其他区域不能(特异性测试)。
\par 
Fuster提供了病变和细胞活性数据的广泛回顾,因此他试图解释广泛的数据范围。然而,考虑到层次和“跨时间整合”的概念并不适用于所有的数据,他采用了各种解释,而不是像我们在第8章中所做的那样,就PF皮层的基本功能作为一个整体提出一个单一的建议。因此,福斯特得出结论,PF皮层在工作记忆、集合和行为抑制等方面发挥作用。我们在表中输入一个“X”,这意味着尽管他试图解释广泛的数据范围,但他没有提供超越层次结构调用的综合建议。
\subsection{计划和顺序}
到目前为止所讨论的理论强调了在延迟期间维护和集成信息。这些理论通常强调回溯性记忆和记忆中感官信号的维持。但是第6章和第7章回顾了PF皮层中延迟期活动对目标进行前瞻性编码的证据。前瞻编码是一种工作记忆,但不是PF皮层工作记忆理论支持者所设想的那种。
\par 
前瞻性编码使得Tanji和Hoshi(2008)强调了PF皮层在规划中的作用,尤其是在一系列目标的规划中。正如第6章所提到的,Mushiake等人(2006)发现了在解决视觉迷宫任务的一系列运动中为未来步骤编码的细胞。
\par 
在他们的综述中,Tanji和Hoshi引用了他们自己实验室的研究,表明PF皮层细胞编码目标序列的抽象结构(Shima et al. 2007)。这些作者和其他人一样,认为PF皮层在其顶端有一个层次结构。例如,他们表明,preSMA中的细胞编码特定序列的时间组织(Shima和Tanji 2000)。通过回顾这种能力背后的联系,Tanji和Hoshi的规划理论在通过解剖测试方面取得了一些进展。
\par 
Hoshi(2008)遵循同一路线,专门比较了PF皮层背侧和腹侧的细胞活性与运动前皮层背侧和腹侧的细胞活性以及初级运动皮层的细胞活性。他进一步讨论了这些区域之间的联系,并试图满足解剖学和特异性测试。
\par 
计划包括准备一系列的目标。因此,我们一致认为,目标、目标序列和目标序列的抽象的前瞻性捕获了PF皮层功能的一个重要方面。但作为PF皮层的理论,序列规划理论无法通过一般性检验。同样的结论也适用于任何过于依赖序列的理论。回想一下,在猴子中,颗粒状PF皮层的损伤会导致缺乏目标序列的简单任务受损,例如条件视觉运动任务(Bussey et al. 2001),匹配样本任务(Rushworth et al. 1997a),或没有延迟的去不去任务(Iversen和Mishkin 1970)。PF皮层在计划序列方面的理论并不能解释这些结果。
\par 
我们承认Tanji和Hoshi(2008)在他们的提案中包含了PF皮层有助于“执行控制”的许多方面的想法,包括对行动的选择性注意,有意行动的选择,以及行为规则的实施。然而,他们的方法面临着一个困难,即当人们将PF皮层的计划理论推广到包括执行控制的更多方面时,就很难知道什么是证伪。
\subsection{时间扩展事件}
Gaffan和他的同事认为,PF皮层处理并代表“暂时扩展”或“暂时复杂”事件(Browning和Gaffan 2008)。第8章解释了最初产生这一理论的具体实验。简单地说,它包括教猴子选择一张图片会导致另一张图片呈现2秒,然后才会提供奖励(见图8.8A)。在控制条件下,延迟不被屏幕上的任何刺激所填充。Browning等人发现,颞下皮层和PF皮层之间的连接断开的猴子在实验(填充延迟)条件下有损伤,但在对照组(未填充延迟)条件下没有损伤(见图8.8B)。只有前一种情况在奖励之前有一系列事件。
\par 
Wilson等人(2010)认为,这一理论提供了对PF皮层整体功能的解释,并将重点从模糊和不可测试的概念(如执行功能)转变为一种重要的表征性知识。我们同意后一点。我们也同意,学习在不同时间框架内展开的事件序列,抓住了PF皮层功能的一个重要方面,特别是因为它有助于上下文层次(第8章)。
\par 
然而,这一理论显然无法通过一般性检验,因为,正如前面提到的,根据任何常规分析,患有PF损伤的猴子在执行缺乏事件序列的任务时可能会受损。另一方面,如果一个人认为所有的任务都有一个事件序列,那么这个理论就变得不可证伪。
\subsection{条件性学习}
Goldman-Rakic(1987)和Passingham(1993)都强调,受PF皮层病变影响的任务往往具有条件性结构(第6-8章)。在条件性任务中,正确的选择取决于一个指令线索。以这种方式解释,这类任务不仅包括明显的任务,如有条件的视觉运动学习和成对的联想学习,还包括延迟反应任务。条件性学习理论抓住了这样一个事实:PF皮层在根据当前环境灵活地生成目标方面起着关键作用,不管是在延迟反应任务上有延迟,还是在条件性视觉运动学习任务上没有延迟。该理论是由 Passingham(1993)提出的理论,涉及到了解剖学测试。
\par 
然而,其他领域也关键性地参与到条件行为中。例如,运动前皮层的病变会导致条件性视觉运动任务的严重损害、前运动皮层的病变会导致条件性视觉运动任务的严重损害。(Halsband和Passingham 1985; Petrides 1987),而投射到运动前区的丘脑核的病变也是如此。投射到运动前区的丘脑核也是如此(Canavan等人,1989)。对灵长类PF的全面描述皮层的全面说明必须把它与其他区域的贡献区分开来。因此,条件性学习理论未能通过特异性测试。
\par 
该理论也未能通过通用性测试。它不能解释为什么PF皮层的病变会对上一节中描述的使用时间上延伸的事件的任务造成损害。那是一个简单的带有奖励延迟的辨别任务、 而不是一个条件性任务(Browning和Gaffan 2008)。

\section{目的}

\section{定义和术语}


\section{指纹}

\subsection{损伤和激活}

\subsection{损伤和活动}

\subsection{活动和激活}




\subsection{结论}



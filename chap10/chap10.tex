\chapter{结论} \label{chap:chap10}
这本书提出了关于灵长类动物前额叶皮层基本功能的方案.

\section{摘要}
本章将我们的建议与文献中的其他建议进行比较,并通过五个测试对每个建议进行评估:(1)该建议是否考虑到了PF皮层的进化史?(2)它是否解释了连接解剖学如何允许PF皮层执行所提议的功能?(3)它是否明确了PF皮层的功能与其他区域的功能有何不同?(4)是否与PF皮层的广泛发现相一致?(5)它的陈述是否精确到足以被测试?最后,我们提出一些测试我们建议的方法。
\section{介绍}
第一章阐述了五个目标,我们现在可以更全面地阐述这些目标:
\par
1.说颗粒状PF皮层让灵长类动物做了他们的祖先和其他哺乳动物做得效率较低的事情,如果有的话,以及说类人猿灵长类动物颗粒状PF皮层的扩张让他们比其他灵长类动物做得更好。
\par
2.说颗粒状PF皮层的轴突连接如何允许的,而不是其他区域,提供这些优势。
\par
3.作为一个整体解释粒状PF皮层的基本功能,并解释它的功能与大脑其他部分的功能有何不同。
\par 
4.在复杂的认知任务中,这个功能是如何解释在人类PF皮层中观察到的激活的。
\par 
5.解释我们的建议与文献中其他建议的不同之处,并告诉读者什么样的观察可以反驳它。
\par 
现在是时候评估一下我们在多大程度上实现了这些目标。第二章讨论第一个问题。它研究了PF皮层的进化,包括一些导致灵长类动物进化过程中特定进展的选择压力。我们认为,颗粒状PF皮层的进化是分阶段发生的,这些阶段伴随着其他进步,如视觉和手功能的发展。早期的前辈们的眼睛是向前看的,他们采用了新的移动、抓取和用手喂嘴的方式。在这些动物中出现了第一个颗粒状的PF区域,即PF尾部皮层和颗粒状的OFC。背侧、腹侧和极侧的PF皮层在类人猿灵长类动物中出现得更晚,因为它们变得更大,不得不与它们喜欢的食物严重短缺作斗争,更不用说竞争和捕食的威胁了。
\par 
第三-七章讨论第二个目标。这些章节解释了灵长类PF皮层各主要区域的连接如何允许它做它所做的事情,以及为什么只有它才能执行它的功能。我们逐个区域检查PF皮层,因为它的连接因区域而异:
\par 
1.第三章指出了内侧PF皮层与海马体、杏仁核和内侧前运动区之间的联系,这使得它能够更好地使用“内部”信号来指导在行动和行动规则之间的觅食选择,包括评估它们与所涉及的努力成本相关的当前价值。
\par 
2.第四章强调,眶侧PF皮层的连接使其能够更好地利用外部信号来指导觅食选择,在物体之间进行选择。眶PF皮层与许多感觉区域有联系,包括视觉、躯体感觉、味觉、嗅觉和内脏皮层。这些连接允许眶PF皮层发展有关特定行为结果的高维信息连接,并强调其视觉特征。我们认为,基于单个事件,细粒度OFC将特定结果分配给似乎导致它的特定选择。与杏仁核的相互联系根据当前需求提供了更新的结果评估。
\par 
3.第五章重点介绍了注意力和搜索功能。尾侧PF皮层从视觉区域接收到的连接传递了来自低阶和高阶视觉的信号,包括背侧和腹侧视觉流。基于与这些区域的皮质皮质连接以及控制脑干动眼肌核的皮质投射,尾侧PF皮层可以将显性和隐性注意力引导到潜在的行动目标上。我们认为,尾侧PF皮层(第8区),包括额叶眼场(FEF),参与了对学习产生的目标的搜索——目标导向的注意力——而不是反射性或刺激驱动的注意力。我们不认为FEF是眼球运动区或运动前区,而是将其视为前额叶皮层的一部分:一个将注意力导向有学习价值的物体和地方的部分,包括隐蔽注意力和显性注意力(以眼球运动的形式)。这样,它加强了中央凹和中央凹外信息的处理。
\par 
4.第六章强调了由背侧PF皮层产生目标,部分基于与后顶叶皮层的连接。这些投影提供了有关视觉事件的顺序、时间和位置的信息,这些信息构成了当前行为环境的重要部分。目标的产生也依赖于眶PF皮层关于与这些目标及其当前值相关的结果的信息。前额叶皮层的背侧提供了一种机制,可以消除以前事件记忆引起的干扰,它似乎是通过前瞻性地编码当前目标来做到这一点的,至少在一定程度上是这样。与前发动机区域的连接为实现这些目标提供了一条途径。
\par 
5.第七章提出,腹侧前额叶皮层根据视觉或听觉环境产生目标。它之所以能发挥这一功能,是因为它与下颞叶和上颞叶皮层、眶前皮层和杏仁核有联系。这些联系为它提供了视觉和听觉信号、预测的结果,以及根据当前生物需求对该结果的估值。符号由基本特征和整体对象之间的特征结合的中间层次组成。腹侧前额叶皮层使用同样的连接来应用抽象的规则和策略,从而将以前的经验转移到新的行为问题上。
\par 
在以这种方式将PF皮层拆开后,第八章将其重新组装起来。它将PF皮层作为一个整体来研究,并最终实现了我们的第三个目标:一个关于灵长类PF皮层基本功能的具体建议,重点是在灵长类中进化的区域。我们提出,颗粒状PF皮层生成的目标适合于当前环境和当前需求,并且它可以在单个事件的基础上做到这一点。因此,类人猿可以根据一个或几个经验解决广泛的问题,并且可以避免祖先强化学习机制中固有的许多错误。我们提出,在灵长类动物历史上的特定时间和地点(第2章和第8章),为了应对特定的适应压力,粒状PF区域进化为实现一种新的通用学习机制,这一机制增强了祖先的通用学习系统。祖先系统通过加强反馈来调整关联的强度,从而控制自动行为;灵长类动物解决问题的方式包括对行为的注意控制,以及更少的错误。
\par 
第9章讨论我们的第四个目标:解释在人类认知过程中观察到的基本PF功能是如何解释大脑激活的。它强调了重新再现的力量。因此,人类的PF皮层可以重新表示知觉状态、他人的意图和精神状态以及关系之间的关系。第9章还表明,人类的PF皮层阐述了PF皮层在其他类人猿中的功能;就像猴子的颗粒状PF皮层允许它们通过快速学习和抽象策略来避免错误一样,人类的PF皮层允许我们通过指令、模仿和心理试错行为来避免错误。
\par 
本章的剩余部分将讨论我们的第五个也是最后一个目标:将我们对灵长类动物PF皮层的描述与其他文献进行比较,并提出一些测试方法。因此,接下来的两个部分解释了其他的建议要么缺乏我们提供的进化视角,只适用于灵长类PF皮层的一部分,没有解释为什么它独特的连接组合解释了它的功能,没有说明灵长类PF皮层执行了哪些其他大脑区域不能执行的功能,未能解释PF皮层贡献的广泛行为,或者未能产生可验证的假设。
\par 
我们将可供选择的建议分为两组,分别在不同的部分进行讨论:一组主要依赖猴子的结果,另一组主要依赖人类的证据。当然,前一组的支持者也试图将其推广到人类,后一组的支持者也提到了他们从猴子身上得到的证据来支持他们的观点。所以我们这样划分只是为了方便,有时,当一个理论同时涉及猴子和人类的研究时,我们会在两个部分讨论。
\par 
我们遵循Wood和Grafman(2003)在表格中列出各种理论,并试图根据一套标准来评估它们。我们使用了五个标准,这表明一个成功的PF皮层理论应该:
\par 
1.结合PF皮层的进化史,特别是在灵长类动物中颗粒状PF皮层的出现和在高级类人猿中新颗粒状区域的出现:历史测试。
\par 
2.解释为什么PF皮层的连接解剖使其有可能执行所提议的功能:解剖测试。
\par 
3.识别PF皮层的特定功能,与大脑的其他部分形成对比:特异性测试。
\par 
4.考虑可用数据的广泛范围:通用性检验。
\par 
5.精确到可以被可行的观察所检验:可证伪性检验。
\section{基于猴子证据的理论}
我们自己的建议主要依赖于猴子的证据,因此我们首先考虑这类观点。在表10.1中,"x"表示我们认为列在该行的理论行的理论没有通过特定的测试。问号('?')表示该理论通过了测试、但我们不确定它是否成功地做到了这一点。当用于可证伪性检验时,"?"表示尽管该理论的某些版本没有通过,但其他版本可能通过、即使还没有人把该理论推向这个方向。破折号('-')意味着该理论理论根本不涉及该测试。
\par 
不可避免的是,填写表格需要主观的判断。考虑一下历史测试。如果一个理论对进化论只字不提,就可以认为它没有通过这个测试。然而,我们有时可以看到某个理论可以解决这个问题的方法。 然而,有些理论与进化论的观点根本不一致。一个理论没有通过历史当一个理论把非灵长类物种所共有的一些行为能力假设为灵长类PF皮层的基本功能时,它就不能通过历史的检验。然而,有些理论与进化论的观点根本不一致。一个理论没有通过历史 当一个理论把非灵长类物种所共有的一些行为能力假设为灵长类PF皮层的基本功能时,它就不能通过历史的检验。
\par 
我们也意识到,这些测试并不完全独立。例如,解剖学测试是通过评估一个理论是否具体说明了PF皮层的连接是如何让它做什么的 允许它做它所做的事情。特异性测试评估的是其他区域是否也能做同样的事情。同样的事情,在这样做的时候,我们有时会考虑到连接的问题。
\par 
我们首先讨论了工作记忆理论与所有五个标准的关系 因为它的影响非常大。读者会发现,许多相同的观点 也适用于其他理论。因此,为了避免在讨论这些其他观点时出现重复、我们集中讨论特定的优势或特定的弱点,而不一定要处理所有的测试。
\subsection{工作记忆}
第5章和第6章回顾了导致关注工作记忆是PF皮层的主要功能(如果不是唯一功能的话)的结果(Goldman-Rakic 1987)。多年来,这一观点一直主导着文献,支持者关注的是回溯性的感官记忆。像许多其他关于PF皮层的理论一样,它主要来源于少数几个任务,在这种情况下主要是延迟反应和延迟交替任务。我们理解这种强调,考虑到PF皮层损伤对这些任务造成的严重损害(第6章)。但我们认为工作记忆理论未能通过一个成功理论的许多测试,因此我们现在依次讨论这些缺陷。
\par 
在考虑工作记忆理论如何面对历史考验时,我们要承认普鲁斯和戈德曼-拉基奇(1991a, b)的开创性贡献。尽管戈德曼-拉基奇是工作记忆理论的强大而坚定的支持者,但她与普鲁斯发表的比较研究提供了一个关键的见解,彻底颠覆了该理论。猴子的典型工作记忆任务所依赖的区域,是颗粒状PF皮层的一部分,在非哺乳动物或丛林婴儿中都不存在。然而,这些动物拥有强大的工作记忆能力。
\par 
因此,从比较的角度来看,工作记忆理论是没有意义的。第一章解释了老鼠可以完成需要工作记忆的任务,如径向臂迷宫,在本章的后面我们会更详细地讨论这些争论。然而,像其他非灵长类哺乳动物一样,老鼠缺乏颗粒状的PF皮层。因此,人们必须假设,一旦在类人猿类灵长类动物中出现粒状PF皮层,它就“接管”了其他区域以前执行的功能。这种想法既不朴素,也不可信。老鼠有着复杂的大脑,能够解决工作记忆任务带来的简单问题,这并不奇怪。需要解释的发现是,有中外侧前额叶皮层病变的猴子无法解决这些简单的问题。我们稍后再讨论那个话题。
\par 
第二个测试是解剖学测试,要求工作记忆理论解释灵长类动物PF皮层的连接如何导致其独特的功能。在她对工作记忆理论的主要介绍中,戈德曼-拉基奇(1987)对当时人们所知的PF皮层的连接进行了广泛的回顾。在Wilson etal .(1993)中,她在比较PF皮层背侧和腹侧的功能时,也提到了顶叶和颞叶的连接。
\par 
根据最近的证据,我们建议重新定义这种区别,即后顶叶皮层不仅提供空间,而且还提供时间和其他输入到背侧PF皮层(第6章)。但我们同意与颞叶和后顶叶皮层的连接是PF皮层功能的一个重要方面。最后,Goldman-Rakic还讨论了pf皮质的细胞结构如何支持工作记忆(Goldman-Rakic1995;Constantinidis et al.2001),尽管这些想法并没有直接针对解剖测试。
\par 
第三个测试是特异性测试,正如所阐明的那样,工作记忆理论未能说明PF皮层与后顶叶皮层有何不同。这一理论的支持者非常强调在PF皮层中存在具有延迟期活动的细胞。然而,如Chafee和Goldman-Rakic(1998)等人所示,后顶叶皮层也存在延迟期活动的细胞。工作记忆理论的支持者可以提出后顶叶皮层的记忆编码依赖于PF皮层,他们可以通过指出冷却后顶叶皮层导致后顶叶皮层延迟相关活动减少的事实来支持这一观点(Chafee和Goldman-Rakic 2000)。但是,同样的实验表明,冷却后顶叶皮层对PF皮层的活跃度也有类似的影响,因此,这种论点是没有说服力的。如果修正该理论,提出工作记忆依赖于前额叶皮层和后顶叶皮层之间的信息循环,那么它就无法通过特异性测试。
\par 
第四个检验是一般性检验,它评估理论是否能解释所有可靠的数据。表8.1列出了任务和损伤效应的选择性列表,可以看出工作记忆理论无法解释许多已发表的结果。例如,腹侧PF皮层病变的猴子在去不去辨别任务(Iversen和Mishkin 1970)、同时匹配到前样例任务(Rushworth et al. 1997a)和条件视觉运动任务(Bussey et al. 2001)中表现出损伤。这些任务在提示的呈现和随后的动作之间都没有延迟期。然而,很明显,戈德曼-拉基奇和她的同事(Wilson et al. 1993)希望工作记忆理论至少适用于PF皮层的整个外侧表面,如果不是整个PF皮层的话。
\par 
该理论也未能解释在PF皮层中发现的许多细胞类型(表8.2)。正如其支持者所阐述的那样,工作记忆理论在很大程度上依赖于所谓记忆时期的细胞活动。在简单的任务中,如动眼肌延迟反应任务,许多细胞似乎对空间位置的记忆进行编码。当然,这一发现与理论是弱一致的(Funahashi et al. 1989)。但是,一项更仔细的对照研究表明,只有少数PF细胞编码一个被记住的位置,而不是一个被关注的位置(Lebedev et al. 2004)。因此,工作记忆理论也未能通过一般性检验。
\par 
此外,如果该理论适用于整个PF皮层,它在其他方面也未能通过一般性检验。Tsujimoto等人(2010)研究了猴子执行动眼力延迟反应任务时极缘PF皮层的细胞活动。在延迟期间,他们没有发现编码记忆地点或任何其他东西的活动。在Genovesio等人(2011)对中外侧和尾侧PF皮层的研究中,猴子判断相对距离,很少有细胞编码记忆位置,仅比预期的细胞多2 - 5%
\par 
最后一个考验是,一个有效的理论必须是可证伪的。在这里,我们输入一个关于工作记忆理论的查询,因为它可以用许多不同的方式表述。如果这个词仅仅是指在一段时间内记住先前线索的能力,那么它是可证伪的,是错误的。但是,正如第一章所解释的那样,巴德利和希契(1974)首次提出的工作记忆理论包括一个“中央执行人员”,而戈德曼•拉基奇(1998)在她的工作记忆理论的制定中也包括了中央执行人员。在这种情况下,没有任何观察能真正检验这一理论,因为几乎所有有趣的任务都涉及某种中央执行功能。
\par 
由于工作记忆理论未能通过通用性、特异性和历史检验,我们得出结论,工作记忆不是PF皮层的基本功能。
\subsection{跨时间偶然性}
Fuster(2008)提出,PF皮层在延迟期间整合信息方面发挥作用。他称这种关系为跨时间偶然性,因为它涉及跨越时间间隔整合信息。例如,对于延迟响应任务,正确的选择取决于在延迟期间之前发生的事件。所以跨时间整合和层次的概念抓住了PF皮层功能的这一方面。
\par 
在他的书中,Fuster(2008)回顾了他所看到的PF皮层的进化史。然而,他只处理了我们在第二章中提出的几个问题,我们不同意他的论点的关键方面。例如,福斯特认为啮齿动物、食肉动物和灵长类动物的PF皮层是同源的,这是我们所反对的观点。最重要的是,Fuster并没有具体说明PF皮层的各个部分赋予了在它们第一次出现时进化出这些结构的物种的优势。所以福斯特的理论没有通过历史的检验。
\par 
Fuster(2008)也回顾了PF皮层的连接。他认为PF皮层位于感知-行动循环的顶端,这一观点与我们关于层次结构的观点主要在特异性程度和术语上有所不同。因此,在这方面,他的理论解释了为什么PF皮层可以执行它所做的功能(解剖测试),而为什么其他区域不能(特异性测试)。
\par 
Fuster提供了病变和细胞活性数据的广泛回顾,因此他试图解释广泛的数据范围。然而,考虑到层次和“跨时间整合”的概念并不适用于所有的数据,他采用了各种解释,而不是像我们在第8章中所做的那样,就PF皮层的基本功能作为一个整体提出一个单一的建议。因此,福斯特得出结论,PF皮层在工作记忆、集合和行为抑制等方面发挥作用。我们在表中输入一个“X”,这意味着尽管他试图解释广泛的数据范围,但他没有提供超越层次结构调用的综合建议。
\subsection{计划和顺序}
到目前为止所讨论的理论强调了在延迟期间维护和集成信息。这些理论通常强调回溯性记忆和记忆中感官信号的维持。但是第6章和第7章回顾了PF皮层中延迟期活动对目标进行前瞻性编码的证据。前瞻编码是一种工作记忆,但不是PF皮层工作记忆理论支持者所设想的那种。
\par 
前瞻性编码使得Tanji和Hoshi(2008)强调了PF皮层在规划中的作用,尤其是在一系列目标的规划中。正如第6章所提到的,Mushiake等人(2006)发现了在解决视觉迷宫任务的一系列运动中为未来步骤编码的细胞。
\par 
在他们的综述中,Tanji和Hoshi引用了他们自己实验室的研究,表明PF皮层细胞编码目标序列的抽象结构(Shima et al. 2007)。这些作者和其他人一样,认为PF皮层在其顶端有一个层次结构。例如,他们表明,preSMA中的细胞编码特定序列的时间组织(Shima和Tanji 2000)。通过回顾这种能力背后的联系,Tanji和Hoshi的规划理论在通过解剖测试方面取得了一些进展。
\par 
Hoshi(2008)遵循同一路线,专门比较了PF皮层背侧和腹侧的细胞活性与运动前皮层背侧和腹侧的细胞活性以及初级运动皮层的细胞活性。他进一步讨论了这些区域之间的联系,并试图满足解剖学和特异性测试。
\par 
计划包括准备一系列的目标。因此,我们一致认为,目标、目标序列和目标序列的抽象的前瞻性捕获了PF皮层功能的一个重要方面。但作为PF皮层的理论,序列规划理论无法通过一般性检验。同样的结论也适用于任何过于依赖序列的理论。回想一下,在猴子中,颗粒状PF皮层的损伤会导致缺乏目标序列的简单任务受损,例如条件视觉运动任务(Bussey et al. 2001),匹配样本任务(Rushworth et al. 1997a),或没有延迟的去不去任务(Iversen和Mishkin 1970)。PF皮层在计划序列方面的理论并不能解释这些结果。
\par 
我们承认Tanji和Hoshi(2008)在他们的提案中包含了PF皮层有助于“执行控制”的许多方面的想法,包括对行动的选择性注意,有意行动的选择,以及行为规则的实施。然而,他们的方法面临着一个困难,即当人们将PF皮层的计划理论推广到包括执行控制的更多方面时,就很难知道什么是证伪。
\subsection{时间扩展事件}
Gaffan和他的同事认为,PF皮层处理并代表“暂时扩展”或“暂时复杂”事件(Browning和Gaffan 2008)。第8章解释了最初产生这一理论的具体实验。简单地说,它包括教猴子选择一张图片会导致另一张图片呈现2秒,然后才会提供奖励(见图8.8A)。在控制条件下,延迟不被屏幕上的任何刺激所填充。Browning等人发现,颞下皮层和PF皮层之间的连接断开的猴子在实验(填充延迟)条件下有损伤,但在对照组(未填充延迟)条件下没有损伤(见图8.8B)。只有前一种情况在奖励之前有一系列事件。
\par 
Wilson等人(2010)认为,这一理论提供了对PF皮层整体功能的解释,并将重点从模糊和不可测试的概念(如执行功能)转变为一种重要的表征性知识。我们同意后一点。我们也同意,学习在不同时间框架内展开的事件序列,抓住了PF皮层功能的一个重要方面,特别是因为它有助于上下文层次(第8章)。
\par 
然而,这一理论显然无法通过一般性检验,因为,正如前面提到的,根据任何常规分析,患有PF损伤的猴子在执行缺乏事件序列的任务时可能会受损。另一方面,如果一个人认为所有的任务都有一个事件序列,那么这个理论就变得不可证伪。
\subsection{条件性学习}
Goldman-Rakic(1987)和Passingham(1993)都强调,受PF皮层病变影响的任务往往具有条件性结构(第6-8章)。在条件性任务中,正确的选择取决于一个指令线索。以这种方式解释,这类任务不仅包括明显的任务,如有条件的视觉运动学习和成对的联想学习,还包括延迟反应任务。条件性学习理论抓住了这样一个事实:PF皮层在根据当前环境灵活地生成目标方面起着关键作用,不管是在延迟反应任务上有延迟,还是在条件性视觉运动学习任务上没有延迟。该理论是由 Passingham(1993)提出的理论,涉及到了解剖学测试。
\par 
然而,其他领域也关键性地参与到条件行为中。例如,运动前皮层的病变会导致条件性视觉运动任务的严重损害、前运动皮层的病变会导致条件性视觉运动任务的严重损害。(Halsband和Passingham 1985; Petrides 1987),而投射到运动前区的丘脑核的病变也是如此。投射到运动前区的丘脑核也是如此(Canavan等人,1989)。对灵长类PF的全面描述皮层的全面说明必须把它与其他区域的贡献区分开来。因此,条件性学习理论未能通过特异性测试。
\par 
该理论也未能通过通用性测试。它不能解释为什么PF皮层的病变会对上一节中描述的使用时间上延伸的事件的任务造成损害。那是一个简单的带有奖励延迟的辨别任务、 而不是一个条件性任务(Browning和Gaffan 2008)。
\subsection{行为抑制,反应抑制,抑制控制}
关于PF皮层(或其部分)的一些理论强调行为抑制、抑制控制或反应抑制(例如Roberts 和 Wallis 2000;Eagle et al. 2008)。例如,PF皮层病变的患者在需要进行抗眼跳时抑制proaccade的功能受损(Ploner et al. 2005),而腹侧PF皮层病变导致停止信号反应时间任务的停止功能受损(Aron et al. 2003)。这些发现与目标的产生和消除有关,但并不表明PF皮层的基本功能是抑制控制。
\par 
行为抑制理论通过了历史检验,因为它认为 PF 皮层进化为抑制自动或优势行为,包括本能行为。 它还通过了特异性测试,因为它说明了 PF 皮层所做的其他区域不做的事情。
\par 
因此,行为抑制的概念在我们的建议中占有一席之地。 当默认行为、习惯性行为或优势行为停止产生预期结果时,对这些行为的抑制一定是 PF 皮层所做工作的一部分(第 3 章和第 8 章)。 我们的提议强调了 PF 皮层的肯定功能,但它通过认识到专注的目标生成意味着抑制自动生成的目标(第 8 章)间接包含了这些负面功能。 默认或优势行为,如注视吸引注意力的刺激,需要被抑制以追求其他目标,如在反扫视任务中。 PF 皮层必须取消目标以及生成目标,并且它必须选择要避免的对象和位置以及要采取行动的对象和位置。 从这个意义上说,我们的建议包括行为抑制等负面功能,但它是隐含的而不是明确的。
\par 
然而,抑制理论未能通过普遍性检验,因为它忽略了 PF 皮层执行的肯定功能。 以反应和奖励的形式表达,该理论预测 PF 皮层的损伤会导致异常的持久性反应,不再产生奖励,称为坚持。 然而,证据表明,在做出无回报的选择后坚持过多并不比在做出有回报的选择后坚持过少更为突出。
\par 
例如,米尔纳 (Milner, 1963) 强调了她的观察结果,即额叶病变患者在威斯康星卡片分类任务中会犯持续性错误。 当 Mishkin (1964) 提出 PF 皮层损伤导致“中枢组持续存在”时,他接受了这个想法。但此类患者也会犯随机错误(Barcelo 和 Knight 2002)。 除了未能改变产生负面反馈的规则(坚持不懈)之外,他们还未能坚持产生积极反馈的规则(坚持不力)。
\par 
逆转任务也可以检验行为抑制理论,该理论预测 PF皮层损伤应该主要是在无奖励试验(负反馈)后增加错误。动作逆转任务的结果与这一预测相矛盾(见图3.8)。正常的类人猿会保持奖励行为,而内侧PF皮层受损的类人猿则较少这样做(Kennerley 等人 2006 年;Rudebeck 等人 2008 年)。 同样,在物体反转任务中(见图 4.8),有眼眶 PF 皮层损伤的类人猿使用正反馈的效率低下,但几乎可以正常使用负反馈(Rudebeck 和 Murray 2008)。 卡米尔等人。 (2011) 研究了具有可比病变的患者,并证实了两项任务的这些发现(见图 4.9)。
\par 
条件视觉运动任务也可以检验行为抑制理论。 第 7 章解释了一些类人猿在执行此任务时自发地采用了“改变-转移”和“重复-停留”策略 (Wise 和 Murray 1999)。 如果 PF 皮层的功能主要是抑制先前的反应,那么 PF 皮层损伤不应该对“停留”策略造成损害。 坚持先前的奖励反应应该受益于导致坚持不懈的病变。 事实上,图 8.6 显示,腹侧和眼眶 PF 皮层联合损伤的类人猿在“停留”和“转移”策略中表现出相同(和严重)的损伤。 抑制理论还预测 PF 皮层损伤应该避免持续的、过度学习的条件性视觉运动关联,也称为刺激-反应 (S-R) 关联或习惯(第 3 章)。 然而,腹侧和眼眶 PF 皮质的联合损伤会导致严重损伤(Bussey 等人,2001 年)。
\par 
同样根据抑制理论,三臂强盗任务中的高波动性条件(见图 4.5 和 4.6)应导致眼眶 PF 皮质损伤后的大损伤。 有缺陷的类人猿应该坚持以前的选择,而不是灵活地改变他们的选择。 然而,受损的类人猿通常会“追踪”或“匹配”不稳定的奖励概率(Walton 等人,2010 年)。
\par 
反向奖励权变任务提供了对行为抑制理论的直接检验。 在这项任务中,选择较少量的食物会产生较大的量,反之亦然。 根据该理论,PF 皮层的损伤应该会导致抑制优势反应的损害:获取更多食物。 事实上,眼眶 PF 皮层的损伤对学习这项任务没有影响 (Chudasama et al. 2007)。
\par 
该理论的支持者可能会争辩说,尽管有所有这些相互矛盾的证据,但其他结果似乎也支持它。去不去歧视任务提供了一个例子。 在这项任务中,一个物体指示类人猿做出“反应”,而另一个物体指示类人猿不做任何反应。 腹侧 PF 皮层受损的类人猿往往会在“不通过”试验中错误地“进行”(Iversen和 Mishkin 1970),称为委托错误。乍一看,这个结果似乎反映了抑制不适当反应的缺陷。然而,在Iversen和Mishkin使用的任务中,只有正确的“go”试验才会产生奖励,这意味着“going”成为默认响应。 如果没有特定的控制程序,类人猿会对“不进行”试验产生强烈的“进行”偏见,以便更快地进行下一次“进行”试验,从而获得下一次获得奖励的机会。
\par 
McEnenay 和 Butter (1969) 通过对类人猿进行一系列逆转测试来克服“去不去”任务中的这个缺陷,这建立了“不去”成为对刺激的默认反应的条件。 在实验者确定了默认行为后,OFC 受损的类人猿在下一次逆转后“走”的速度很慢。 总的来说,去不去歧视任务的结果表明,受伤的类人猿异常缓慢地改变他们的行为,但他们不支持反应抑制方面的解释。 相反,这些结果强调了 PF 皮层的肯定功能:快速学习(第 8 章)。
\par 
Jones 和 Mishkin (1972) 对眼眶和腹侧 PF 皮层进行了联合损伤,他们将坚持定义为一系列较长的试验,在逆转后表现低于机会水平。 他们发现了他们所期望的,行为抑制理论的支持者经常引用这一发现。 然而,Bussey 等人。 ( 2001 ) 进行了相同的损伤,发现快速学习和抽象策略应用(例如“lose-shift”)均受到严重损害。 和 Izquierdo 等人。 (2004) 观察到眼眶 PF 皮层损伤的类人猿未能形成逆转学习集。 第 8 章解释了快速学习、抽象策略和反转集都需要使用单个事件来生成目标。 从这个角度来看,琼斯和米什金获得的结果没有为行为抑制理论提供支持,他们也没有为第 8 章提出的肯定建议提供支持。
\par 
似乎支持行为抑制理论的另一个结果涉及狨猴。 腹侧 PF 皮层的损伤会导致超维转移任务受损(Dias 等人,1996 年)。 第 7 章解释说,在这项任务中,类人猿必须转变为关于哪个刺激维度与两种新刺激之间的选择相关的新规则(见图 7.10)。 迪亚斯等。 根据他们称之为注意力集的先前规则,根据未能抑制选择来解释他们的结果。 然而,PF 皮层损伤的影响不会持续存在:动物在学习第一次异次元转变后的第二次异次元转变中表现正常(Dias 等人,1997 年)。 因此,这些结果似乎反映了快速学习的肯定功能,尤其是在新情况下,而不是规则抑制的损害。
\par 
最后,灭绝任务的结果也被引用来支持行为抑制理论 (Butter 1969)。 眼眶 PF 皮层受损的类人猿对刺激的“反应”持续时间比正常类人猿更长,一旦这些动作不再产生奖励(见图 4.4)。 快速学习的障碍也是导致这一结果的原因。
\par 
因为它没有通过普遍性检验,我们可以拒绝行为抑制理论而不否认 PF 皮层在抑制功能中的重要作用。 除此之外,经常被引用以支持该理论的研究结果与类人猿通过快速学习和抽象策略(第 8 章)使用事件来减少错误的提议非常一致:一个肯定的功能而不是一个消极的功能。
\subsection{类别、规则和策略的抽象}
Miller 和他的同事提出,PF 皮层代表抽象类别 (Miller et al. 2003 ; Roy et al. 2010) 和抽象规则 (Miller 和 Buschman 2008)。 原始证据来自 Freedman 等人的研究。 ( 2002 ) 关于视觉分类,Nieder 等人。 ( 2002 ) on number, and of White and Wise ( 1999 ) and Wallis et al. ( 2001 ) 关于抽象规则。 与此同时,Genovesio 等人。 (2005) 强调了 PF 皮层在表示抽象策略中的作用。
\par 
我们已将这些数据纳入我们的提案,部分是通过在解释分层处理时诉诸抽象概念。 PF 皮层在上下文、目标和结果层次结构中的位置解释了它可以做什么而其他区域不能做什么。
\par 
但是 PF 皮层的纯抽象理论没有通过普遍性测试。 例如,如前所述,患有腹侧 PF 病变的类人猿在没有延迟的简单任务中表现出障碍 (Iversen 和 Mishkin 1970),该任务处理具体的刺激和反应。 它们还在简单的条件视觉运动映射上显示出巨大的缺陷,同样没有延迟,甚至在它们通过大量训练变得自动化之后也是如此(Bussey 等人,2001 年)。 这些有条件的视觉运动任务取决于涉及特定刺激和特定动作的具体刺激-反应关联。类人猿们在最近和遥远的过去都反复经历过这些映射。 它们不依赖于抽象规则、策略或类别。
\par 
此外,来自成像的证据表明,PF 皮层在生成具体和抽象目标方面发挥作用。 罗等人。 (2008) 特别比较了与选择具体手指运动和选择抽象规则相关的激活。 对于这两项任务,中外侧 PF 皮层的激活峰值没有差异。
\subsection{快速学习}
米勒和他的同事测试了猕猴快速学习的能力。 他们教授规则 (Miller 和 Buschman 2008)、提示-响应关联 (Cromer et al. 2011a) 和新类别 (Antzoulatos 和 Miller 2011)。 他们还比较了 PF 皮层或纹状体中的细胞编码选择的相对时间 (Miller和 Buschman 2008; Antzoulatos和 Miller 2011)。
\par 
很明显,米勒和他的同事们特别感兴趣的是快速学习,我们也有同样的兴趣。 然而,我们以不同的方式处理了这个问题,表明颗粒状 PF 皮层的损伤阻碍了快速学习(第 8 章)。我们的建议的不同之处在于强调通过单一事件学习的重要性,并将 PF 皮层功能与其他减少错误的方法联系起来,例如抽象规则和策略的应用。
\par 
就其本身而言,快速学习理论未能通过普遍性测试。 第 7 章和第 8 章解释了颗粒 PF 皮层的一个关键功能涉及抽象行为指导策略的应用。 类人猿必须学习这些策略,但一旦学会,它们只需要将它们应用到手头的问题上。 快速学习理论不解释策略任务的损伤,也不解释 PF 皮层损伤后的其他损伤,例如延迟反应任务。
\par 
米勒和他的同事们并没有专门针对历史测试,尽管他们可以这样做。 因此,我们就这一点提出疑问。
\subsection{分级处理}
我们的提议说,颗粒状 PF 皮层位于背景、结果和目标层次结构的顶端。 当然,其他人强调了 PF 皮层层次结构的重要性。 其中一些理论源自对人类的研究(Koechlin 等人 2003 年;Badre 2008 年),但该想法起源于Jones和Powell(1970 年),他们基于类人猿的皮质皮质连接。 这个想法影响了关于 PF 皮层的几种理论,这些理论通常指向抽象的逐渐增加,因为一个人在 PF 皮层内移动得更多(Badre 2008)。 然而,等级理论的侧重点有所不同。 一些指向域通用性(Wilson 等人 2010),另一些指向关系整合(Wendelken 等人 2008),还有一些指向产生目标选择的因素的复杂性和涉及的时间范围(Summerfield 和 Koechlin 2009) ). 任何 PF 皮层理论都可以包含层次结构的概念。
\par 
正如第 8 章提到的,Fuster (2008) 强调了 PF 皮层位于感知-动作层次结构的顶端这一观点。 他的理论假设了一系列行动途径,其中 PF 皮层形成了最间接的途径。 然而,Fuster 没有具体说明 PF 皮质功能与后顶叶皮质功能之间的差异,后顶叶皮质也参与感知-动作映射。 结果,他的理论没有通过特定城市的检验。 简单地提到层次结构并不能解释灵长类动物 PF 皮层做什么而大脑的其他部分不做,除了这些其他区域在较低的层次结构级别运行。
\par 
我们的提议明确指出,它位于处理层次结构的顶端,允许颗粒状 PF 皮层独特地整合从当前背景和事件中生成目标所需的所有信息,这在很大程度上基于对特定结果及其当前的知识 价值。 然后 PF 皮层向前运动皮层提供空间目标,前运动皮层计算运动计划以实现目标。 正如第 7 章和第 8 章所解释的,下颞叶皮层没有产生目标所需的特定结果信息,它也不直接投射到运动前皮层。 因此,与下颞叶皮层相比,颗粒状 PF 皮层具有所有需要的连接:访问前运动皮层、来自颞叶视觉区域的对象和提示信息、来自编码事件的海马体系统的输入,以及有关事件的结果信息 食物和液体的视觉特性,以及它们最新的生物学价值(第 4 章)。
\par 
同样的论点适用于后顶叶皮层、运动前皮层、颞叶皮层的其余部分,以及海马体的各种组合。 Fuster 没有说,在每种情况下,这些区域的功能有何不同,因为它们在感知-行动层次结构中的地位较低。 我们的提议具体说明了区别:颗粒状 PF 皮层具有所有这些必要的连接,而其他区域则没有。
\subsection{一体化} 
Miller 和 Cohen (2001) 已经强调,正如我们的提议一样,PF 皮层整合了来自所有感觉域的信息,以便选择未来的目标。 他们的理论比文献中的其他理论更像我们的提议。 在他们的评论中,Miller 和 Cohen 借鉴了类人猿研究和人类成像研究。 所以我们在人类研究部分再次讨论他们的想法。 在这里,我们关注主要来自类人猿研究的方面。
\par 
PF 皮层在类人猿中发挥整合功能的证据来自实验,例如 Rao 等人的实验。 ( 1997 ), 谁表明 PF 皮层中的细胞整合了来自背侧和腹侧视觉流的视觉信息。 当每种表示作为行动目标时,这些细胞编码空间和非空间视觉信息。
\par 
虽然我们承认他们的想法与我们的相似,但米勒和科恩 (Miller and Cohen) (2001) 在论文中提出的建议并未解决历史或特异性测试。 一个成功的理论应该解释为什么灵长类动物的 PF 皮层已经进化到可以做它所做的事情,或者为什么它的连接决定了如何单独做到这一点。 然而,我们在他们的提案中看不到任何内容,这会阻止对其进行详细说明以解决这些关键问题。
\par 
我们输入可证伪性测试的查询,因为集成功能的调用,如执行功能,可能非常模糊以至于无法进行测试。
\subsection{自适应编码,一般问题解决}
Duncan (2001) 提出,像我们一样,寻找 PF 皮层的单一、基本功能可能是错误的。 相反,他认为 PF 皮层可能有助于各种各样的认知功能,尤其是当任务变得困难时。
\par 
邓肯的大部分提议(他称之为多需求理论)都依赖于人类数据,因此我们将对他的想法进行更彻底的考虑推迟到下一节。 但是为了支持他的理论,他引用了类人猿细胞活动研究的证据,这些研究表明,很大一部分细胞在各种各样的任务中表现出与任务相关的活动。 我们知道,特定类别的经验会改变 PF 皮层中类别的表示(Roy 等人,2010 年),这里仅举一个 Duncan 称为自适应编码的例子。 Gaffan (2002) 也提出 PF 皮层可以被视为通用的问题解决者。
\par 
然而,仍有待证明 PF 皮层中的细胞比大脑其他部分(例如后顶叶、颞叶或海马回皮层)中的细胞具有更强的适应性。 它们可能会,但在某种意义上,大脑皮层的所有部分都在“学习”,因此没有一个区域可以垄断自适应编码。 它们在了解的内容上有所不同,在其他一些方面也有所不同,但 PF 皮层之外的自适应编码的缺失或缺乏仍有待证明。
\par 
尽管如此,邓肯和加凡都提出了几个有价值的想法。 自适应编码、一般问题解决和跨域处理的概念捕捉了一些关于颗粒 PF 皮层及其基本功能的重要信息。 我们认为我们的提案与这些想法一致,但相信它更具体和完整,部分原因是它解决了 PF 皮层的进化问题,部分原因是它解释了为什么 PF 皮层的连接解剖决定了它的作用。 换句话说,我们的理论通过了历史和解剖学测试,但邓肯和加凡的著作都没有涉及。
\par 
如果朝着那个目标发展,也许一般的问题解决和多需求理论可以通过这些测试。 在我们的提案(第 8 章)中,我们强调颗粒状 PF 皮质作为减少错误的适应性,并且它的许多区域在我们谱系历史的特定时间和地点进化。 例如,我们提出腹侧PF 皮层、背侧 PF 皮层,可能还有极地 PF 皮层随着类人灵长类动物体型的增加而进化,并开始依赖特别不稳定和竞争激烈的觅食环境中的食物资源。正如我们在本章前面和第 8 章所说,我们的提议表明,颗粒 PF 皮层的进化为类人猿提供了系统发育上新的通用问题解决者,以配合依赖于缓慢调整的较旧的通用学习系统 基于强化反馈的关联。
\subsection{动物学习论}
除了到目前为止提出的理论之外,还有另一个关于动物学习的更普遍的想法,可以用来解释灵长类动物 PF 皮层的基本功能。 动物学习理论认为,所有学习都源于建立和修改刺激、反应和结果之间的关联。 我们称之为祖先强化学习系统,我们认为灵长类动物 PF 皮层的进化主要是为了增强它。 旧的通用学习系统在动物的历史早期就已经发展起来,并为它们提供了很好的服务。 事实上,祖先的通用学习系统的力量解释了为什么灵长类动物以外的动物在没有颗粒状 PF 皮层的情况下也能相处得很好。
\par 
如果所有学习都取决于刺激、反应和结果之间的关联,那么所有 PF 介导的学习也必须与刺激、反应和结果之间的关联有关。 根据这种观点,除了感觉处理、概括和不同程度的联想复杂性之外,动物在行为能力上几乎没有差异。 该理论通过说灵长类动物 PF 皮层的贡献来自第 8 章讨论的那种综合功能来解决特异性测试。 一般来说,动物学习理论认为,PF 皮层允许灵长类动物做与其他动物相同的事情,而且做得更好。 例如,与其他哺乳动物相比,卡他林灵长类动物的三色视觉进化允许在特定光谱范围内进行更精细的颜色辨别。 否则,根据动物学习理论,鸽子、猪和人都以同样的方式学习。
\par 
当应用于 PF 皮层功能时,动物学习理论未通过可证伪性测试。 换句话说,因为动物学习理论是不可证伪的,所以它在 PF 皮层的应用同样是不可证伪的。 尽管任何行为都可以用刺激、反应和结果来描述,但这并不意味着所有动物都通过相同的学习机制来学习它们之间的关联。 事实上,将这些概念应用于行为的能力并不能说明潜在的学习机制。
\par 
此外,动物学习理论未能通过历史检验,因为它引用了一种功能,该功能早在哺乳动物的无颗粒 PF 皮层或灵长类动物的颗粒 PF 皮层出现之前就已进化。
\par 
然而,我们采用了动物学习理论的两个方面。 第 3 章解释说,在大鼠中,内侧 PF 皮层的某些损伤会导致结果导向行为受损,但习惯性行为完好无损。 内侧 PF 皮层其他部分的病变具有相反的效果。 因此,正如我们在第 3 章中讨论的那样,第一个在哺乳动物中进化的 PF 区域——无颗粒 PF 区域——调节通过祖先强化学习机制获得的行为。 因此,人们可以将无颗粒 PF 皮层视为哺乳动物祖先完全依赖强化学习系统与灵长类动物颗粒 PF 皮层后来增强之间的中间阶段。
\par 
其次,像 Balleine 和他的同事一样,我们也区分需要细心处理的行为和不需要细心处理的行为。 第 8 章论证了 PF 皮层在目标的专注生成过程中变得活跃。 然而,我们与动物学习理论家的不同之处在于拒绝将结果导向行为等同于对行为的细心控制。 事实上,Balleine 和 O’Doherty (2010) 甚至将所有行为分为习惯或结果导向行为,他们称之为目标导向行为。 他们将这个想法应用到人类身上。
\par 
然而,许多以结果为导向的行为是在没有注意到它们的情况下发生的。 序列行为已得到广泛研究,但也许最具启发性的例子涉及动物学习理论家青睐的学习类型:刺激-反应-结果 (S–R–O) 关联。
\par 
Johnsrude 和她的同事 (2000) 研究了健康人和眼眶 PF 皮层或杏仁核受损的患者。 受试者将注意力集中在一项任务上,该任务要求他们计算在给定的一系列试验中出现在不同选定位置的红点数量。 有时会出现黑点。 随着一个黑点或红点,一个视觉图案出现了,当这个点是红色的时候,受试者就可以获得奖励:糖果或葡萄干。 受试者不知道的是,各种模式在 10$\%$、50$\%$ 或 90$\%$ 的时间内与奖励相关。 后来,受试者表明他们在工具上习惯于选择高价值模式而不是低价值模式,尽管他们不知道自己为什么做出选择。 也就是说,他们不经意地学习了作为结果导向行为基础的 S-R-O 关联。 有趣的是,受试者为他们的选择编造了完全不相关的理由,比如“这个图案看起来很有趣”之类的话。 这一发现表明,结果导向行为不止一种。 从这个意义上说,动物学习理论未能通过普遍性检验,因为它无法解释这些不同种类的结果导向行为。
\subsection{概括}
刚刚调查的理论都没有像第 2 章那样解决 PF 皮层的进化问题。 尽管一些理论的支持者回顾了 PF 皮层的连接,但他们很少这样做是为了说明为什么只有 PF 可以做它所做的事情,就像我们在第 3 章到第 8 章中所做的那样。 许多理论并没有像我们所做的那样解释 PF 皮层与大脑皮层其他部分的不同之处。 其他理论无法解释重要发现。 并且有些理论是如此普遍以至于它们无法被证伪。
\section{基于人类受试者证据的理论}
表 10.2 列出了在很大程度上依赖于人类研究的理论,并根据表 10.1 中使用的相同标准进行了评估。 除了以一种敷衍的方式比较类人猿和人类而不参考他们的共同祖先之外,没有一个理论专门针对历史测试。 然而,我们冒昧地指出了提议的功能在哪些地方无法通过历史测试,甚至可以想象。
\subsection{监控}
Petrides (1994) 和我们一样认为,工作记忆理论未能捕捉到 PF 皮层的基本功能。 相反,他和他的同事针对 PF 皮层的作用提出了一个两阶段模型 (Owen et al. 1996a)。 该理论的一部分认为中外侧 PF 皮层的功能是监测记忆中的项目,另一部分提出腹侧 PF 皮层控制从长期记忆中非自动检索项目。
\par 
监视的概念来自有序的对象任务,也称为自排序任务或主题排序任务。 在给人们的任务的图片版本中,受试者必须以他们选择的任何顺序指向图片,唯一的规则是他们不能在一次试验中两次指向同一张图片。 这条规则意味着受试者必须建立对他们到目前为止所选择的图片的记忆,以便识别那些仍然可用于当前选择的图片。 Petrides 建议此过程涉及监控内存中的项目列表。
\par 
术语监控也可以应用于 n-back 任务。与有序对象任务一样,受试者看到或听到一系列项目,例如字母,他们必须监视它们在一系列中的位置。 因此,内存中项目的监控不同于内存中项目的维护,因为项目在内存中以某种方式被标记或区分。 在我们看来,关键是所有这些任务都涉及秩序。
\par 
该理论的任何表述都没有以任何严肃的方式解决历史检验问题,但 Champod 和 Petrides(2007 年)试图解决特异性问题。 他们声称中外侧 PF 皮层的功能是监控,而后顶叶皮层的功能是处理记忆中的项目。
\par 
这种解释有两个问题。 首先,成像数据显示在两种情况下两个区域都有激活,尽管程度不同。 其次,Postle 等人。 (2006) 将 rTMS 应用于 PF 皮层,这种临时损伤破坏了记忆中的项目操作。 因此,监控理论未能通过普遍性检验。
\par 
正如所阐述的那样,该理论解决了解剖学测试,因为它根据连通性解释了 PF 皮层功能的两个阶段——监测和检索。 但它并没有像我们的建议那样解释为什么 PF 皮层的连接决定了它独特的功能。 中外侧 PF 皮层的连接无法区分监控或操纵记忆中的项目。 因此,我们为此测试输入一个查询。
\subsection{积极维护记忆中的项目}
Ungerleider (1995) 提出中外侧 PF 皮层在信息的主动维护中发挥作用。 酒井等。 ( 2002a ) 采用这个术语来解释中外侧 PF 皮层中延迟周期激活在保护空间项目免于分心方面的影响。 对 PF 皮层功能的讨论通常会引用这个概念 (D’Esposito 2007)。
\par 
要具有任何价值,术语主动维护必须与被动维护形成对比,被动维护是一种不需要注意力的记忆维护。 双任务范式可以操纵注意力来测试这种区别。例如,如果受试者必须参与另一项涉及发音的任务,则受试者在记住一系列字母或数字时会出错,这会扰乱“语音循环”中的复述 (Baddeley 1986)。 以同样的方式,人们可以通过要求受试者对不相关的目标进行扫视来干扰空间项目的排练,这会破坏工作记忆的“视觉空间暂存器”中保存的项目(Guerard 等人,2009 年)。 在任何一种情况下,记忆都会因为竞争任务的需求受到干扰而受到影响。 因此保养可以说是用心或主动。
\par 
我们承认,人的 PF 皮层可能有助于记忆中项目的主动排练,第 6 章介绍了这种排练。 然而,作为 PF 皮层的理论,主动维护理论无法解释大部分数据。 它狭隘地关注与工作记忆理论相同的观察结果,因此未能通过普遍性检验。 事实上,主动维护理论与工作记忆理论差别不大,因此无法通过相同的测试。 早些时候,我们解释说,类人猿的 PF 皮层损伤会导致许多任务受损,而这些任务几乎不需要主动维护或工作记忆。 此外,许多众所周知的 PF 损伤后的损伤与主动维持无关,例如单词回忆或情景记忆检索的损伤。
\par 
对于特异性测试,我们输入一个查询。 该理论提出,PF 皮层,而不是后顶叶皮层,对于主动维护至关重要。 但它并没有对这些差异提供有说服力的说明。
\subsection{执行控制和目标维护}
在他们的 PF 皮层理论中,Miller 和 Cohen (2001) 像我们一样拒绝工作记忆理论。 我们在这里再次讨论他们的理论,除了前面关于类人猿的部分中的讨论之外,因为它在很大程度上依赖于应用于人类受试者的 Stroop 任务。 Miller 和 Cohen 提出 PF 皮层的功能是“积极维护代表目标和实现目标的手段的活动模式”。 该理论表明,PF 皮层通过施加自上而下的偏差对其他区域施加执行控制。 Miller 和 Cohen 并不像我们的提案那样强调目标的产生,但人们可以将这一想法读入他们对文献的处理中。
\par 
在 Stroop 任务中,受试者看到用蓝色墨水拼写的“红色”等词。 在一种情况下,他们只需要大声朗读单词,而在另一种情况下,他们需要报告墨水的颜色。 在第二种情况下,自动或优势反应是说“红色”,从而读出这个词,这是不正确的。 正如米勒和科恩所说,在斯特鲁普条件下,受试者必须牢记规则,即报告墨水的颜色,这比仅仅阅读单词需要更多的注意力。 麦克唐纳等人。 (2000) 发现当受试者准备报告墨水颜色时中外侧 PF 皮层会激活,但当他们准备阅读墨水拼写的单词时则不会(自动响应)。
\par 
在 Stroop 任务中,规则的表示保存在内存中,用于阅读单词或报告墨水颜色。 Miller 和 Cohen 提出,PF 皮层根据所需任务对低阶机制施加自上而下的偏见,我们在第 5 章中回顾了这种偏见的证据。
\par 
很明显,米勒和科恩 (Miller and Cohen) (2001) 提出的理论在很多方面与我们的建议(第 8 章)相似。 事实上,我们依赖于许多相同的证据,因此人们会期望有相当大的相似性。
\par 
然而,我们注意到一些重要的差异。 首先,我们的理论坚定地将粒状 PF 皮层置于比较的角度; 米勒和科恩提出的理论则不然。 其次,我们的提议强调使用单一事件来指导行为。 我们认为这种能力提供了 PF 皮层赋予灵长类动物的关键适应性优势之一,这一点 Miller 和 Cohen 没有解决,至少没有直接解决。 我们将目标的表示(前额叶功能)与实现目标的方式分开,我们将其归因于前运动皮层。
\subsection{情节控制}
Koechlin 和 Summerfield (2007) 对比了他们所谓的情境控制和情景控制。 上下文控制是指特定上下文对适当动作的规范。 情节控制是指由于适用于特定情节或事件的规则而对该控制进行的修改。 例如,假设在别人家里的情况下,当有人敲门时,一个人不愿意去应门。 然而,如果房主之前曾要求该人应门,则此(询问)事件将建立一个临时规则,并且该人的行为将不同于上下文本身所规定的。 科奇林等人。 (2003) 报道了在情景控制期间 PF 皮层的激活,峰值激活靠近背侧 PF 皮层和腹侧 PF 皮层之间的边界。
\par 
在这种情况下,术语情节是指导致规则存储在内存中的事件,这种用法不同于第 8 章中术语事件的使用,后者将事件视为主体所做的事情以及发生的事情 结果。 因此,正如 Miller 和 Cohen 所建议的(2001 年),情景控制的概念类似于在记忆中积极维护目标或规则。
\par 
然而,Summerfield 和 Koechlin (2009) 进一步发展了他们的提议。 他们认为,沿着 PF 皮层的外侧和内侧部分存在从尾端到头端的层次结构,这与指定上下文或评估值的时间差异有关。 Grafman 和他的同事 (Krueger et al. 2007) 在事件频率方面提出了类似的建议。 在第 9 章中,我们对 PF 皮层内的尾端到喙端层次结构采取了不同的观点。 我们根据重新表示来解释这种层次结构,而不调用时间范围内的差异。 当然,许多平行的层次结构可以共存于 PF 皮层中,并且有可能通过重新表示出现长时域。
\par 
情景控制理论不涉及历史测试。 但是,可以详细说明这样做,建议在我们这样做时随着新区域的发展添加额外的层。 它也没有解决解剖学测试问题,尽管未来的一些公式可能会这样做。
\subsection{操纵}
Postle等人 (1999) 引入了术语“操纵”来解释当受试者重新排列记忆中的项目时,他们观察到中外侧 PF 皮层的延迟期激活。 向受试者提供五个字母,在操作条件下,他们必须在延迟期间按字母顺序重新排序。 在此基础上,他们提出了 PF 皮层的操纵理论。
\par 
然而,在我们看来,他们的结果可能反映了对处理订单信息的依赖,而不是对内存中项目的操作。 在第 6 章中,我们建议当项目的顺序(无论是空间的还是非空间的)对执行任务至关重要时,激活发生在中外侧 PF 皮层。 n-back 任务就是一个例子。 我们还审查了当人们在刺激中产生新顺序时该区域激活的证据。
\par 
解决推理问题也可以说涉及对内存中项目的操作,而不是简单的维护。 我们承认,当人们进行类比推理时,中外侧 PF 皮层会发生激活(Prabhakaran 等人,1997 年)。 在此类任务中,例如 Raven 的渐进矩阵任务,受试者在尝试解决推理问题时会看到所有测试项目,因此该任务对感官短期记忆的负荷最小。 但是,与 Postle 和 D'Esposito 使用的任务一样,这类问题涉及序列,因此涉及顺序。 在这种情况下,空间秩序尤为重要。
\par 
然而,操纵理论未能通过普遍性检验。 可以在不一定涉及操作的任务上找到激活。 例如,我们从 Pochon 等人那里知道。 ( 2001 ) 当受试者准备回忆空间项目时,延迟期激活发生在中外侧 PF 皮层,即使他们不需要操纵记忆中的项目,因为他们稍后会按照呈现的顺序回忆它们。
\par 
PF皮层的操纵理论也未能通过解剖学测试,因为它没有提出操纵如何发生的解剖学解释,并且与源自人类研究的其他理论一样,它没有解决历史测试。
\subsection{注意选择}
Passingham 和 Rowe (2002) 提出了一项关于执行功能的具体建议,该建议与监督有关。 他们认为中外侧 PF 皮层在注意力选择中发挥作用。 在他们的第一个实验中 (Rowe et al. 2000),人类受试者在三个位置看到了提示,他们必须记住这些提示。 延迟一段时间后,出现一条线,受试者必须将光标移动到该线穿过的记忆提示位置。 为了完成这项任务,受试者必须在记忆中的三个位置中进行选择。 Passingham 和 Rowe 提出,他们通过使用注意力来增强记忆中相关位置的表示来做到这一点。
\par 
注意选择的概念与短期记忆中的监控项目非常相似。 两者都涉及标记或强调某些表示等。 注意选择一词的优点是,除了感官信息外,它还适用于目标的产生和实现目标所需的行动。 例如,当受试者需要生成一系列手指运动时,他们可以通过注意其中一个键、其中一个手指或键和手指的位置来实现。
\par 
尽管有其优势,注意选择理论未能通过普遍性检验。 正如 Goldman-Rakic 和 Leung (2002) 指出的那样,即使受试者不需要在记忆中的项目中做出选择,激活也会在回忆时发生在成像研究中。 在他们的影像学研究中,Leung 等人。 ( 2005 ) 通过简单地询问是否有一个探测项目在集合中来测试一组空间项目的记忆。 此任务不需要注意选择记忆中的项目,只需要完整地回忆一组位置。 我们不怀疑中外侧 PF 皮层在记忆项目中的注意选择中发挥作用,实际上第 6 章对这个想法做了很多讨论。 但是因为这个理论没有通过普遍性检验,它不能解释 PF 皮层的基本功能。 作者没有解决历史测试。
\subsection{细心检索}
监控、主动维护、操纵和注意选择的理论都涉及短期记忆中的项目。 他们的不同之处在于,他们是否提出 PF 皮层在记忆中维护项目、监视它们、在其中选择或操纵它们。 然而,所有这些理论都未能通过普遍性检验,因为有证据表明 PF 皮层也在从长期语义或情景记忆中检索信息中发挥作用。 语义记忆由关于世界的事实组成,情景记忆涉及过去的自传事件。
\par 
当受试者从记忆中检索单词或其他语义知识时,腹侧 PF 皮层的尾部会发生激活。 Thompson-Schill 和她的同事 (Novick et al. 2009) 提出,检索涉及在竞争词的神经表征中进行选择。 Badre 和 Wagner (2002) 建议,它涉及他们所谓的注意力检索。 事实上,这两个建议是密切相关的,因为记忆中项目之间的竞争将导致对注意力检索的要求。
\par 
当受试者接受源记忆测试时,腹侧 PF 皮层中的激活发生在更多的吻端 (King et al. 2005)。 Kostopoulos 和 Petrides (2008) 特别提出这个中腹侧 PF 皮层在从长期记忆中非自动(注意)检索项目中发挥作用。
\par 
然而,这些发现仅涉及 PF 皮质的一部分,而不是全部。 Petrides (2005) 通过对比中外侧 PF 皮层和中腹 PF 皮层的功能承认了这一局限性。 因此,注意力检索理论无法提出 PF 皮层作为一个整体所做的事情。 基于这些理由,它没有通过普遍性检验。 它也未能通过历史测试,因为它赋予了 PF 皮层非灵长类哺乳动物可以执行的功能:从长期记忆中检索竞争项目。 “注意力”的调用本身并不能提供足够的精度来通过这个测试。注意力检索理论未能通过解剖学测试,因为它没有解释 PF 皮层的连接如何让它单独执行此功能。
\subsection{结构化事件知识}
Wood 和 Grafman (2003) 提出,从工作记忆、适应性编码、注意力或检索等认知过程的角度来看 PF 皮层可能没有帮助。 他们指出,关于其他大脑区域的理论通常涉及那里代表的知识。 例如,颞叶皮层的一部分代表语义知识,而鼻周皮层编码、存储和代表关于物体的知识。
\par 
因此,Wood 和 Grafman 建议我们需要一个 PF 皮层的表征理论,一个解决 PF 皮层编码、存储和表征的知识的理论。 他们提出 PF 皮层代表结构化的事件知识,不同的子区域代表不同类型的此类知识。 例如,他们认为背侧 PF 皮层代表事件序列,如在计划中,而腹内侧 PF 皮层代表社会规则,因为它们在一系列遭遇中展开。 这个理论有几个优点。 它试图从整体上解释 PF 皮层,并试图解释范围广泛的数据。
\par 
无论多么有价值,我们都不接受仅凭表征知识就能解释 PF 皮层功能的观点。 表征知识和认知过程都对 PF 皮层的基本功能做出重要贡献。人工神经网络的特性说明了这一点。 图 10.1 所示的网络模拟了 Genovesio 等人的抽象策略任务。 ( 2005 ),第 7 章详细解释。 和所有的三层网络一样,它有一个输入层、一个输出层和一个隐藏层。 作为输入,它接收一个编码前一个目标(左、右或上)、前一个提示(A、B 或 C)和当前提示(也是 A、B 或 C)的信号。 隐藏层将输入映射到输出,输出表示当前目标(左、右或上)。 当网络执行这个映射时,它会产生一个输出,从而产生一个目标,它模拟了一个认知过程。 但是产生这个目标的突触权重对应于“重复-停留”和“变化-转移”策略的代表性知识。 网络既有知识又有执行过程。
\par 
因此,我们与 Wood 和 Grafman 的不同之处在于,要完全理解 PF 皮层需要了解其认知过程(例如目标生成)及其表征知识(例如上下文-目标映射、抽象规则和策略)。以牺牲另一个为代价强调其中一个的危险在于它会导致通用性测试失败。 该理论也未能通过解剖学测试,因为它没有说明连接如何解释 PF 皮层在结构化事件知识中的独特功能。
\par 
尽管如此,我们的提议与他们理论的许多方面是一致的。 他们的理论和我们的理论都强调事件、背景和目标的顺序、等级、抽象和整合的重要性。 此外,与 Grafman 和他的同事 (GomezBeldarrain et al. 2004) 一样,我们强调 PF 皮层在处理遥远的时间范围中的作用。 Tulving (2005) 将此功能称为心理时间旅行,第 9 章解释了想象场景和参与心理试错行为的重要性。
\subsection{监督注意系统}
如前一节所述,Wood 和 Grafman (2003) 将他们的理论与注意力控制等加工理论进行了对比。 Norman 和 Shallice (1980) 提出了一个理论,将“意志”与自动行为进行对比。 通过有意的行动,他们指的是对行动的注意力控制。 需要注意行为的情况包括预期结果未能发生的情况,例如因为环境发生了变化。 在后来的一篇论文中,Shallice (1982) 特别提出这种监督注意系统依赖于 PF 皮层。
\par 
在后来的研究中,Stuss 和 Alexander (2007) 研究了一大群额叶病变患者。 他们比较了内侧、左侧和右侧损伤的影响,并得出结论,它们执行了独立但相互关联的行为控制过程。 与 Shallice 合作,他们证明内侧和腹侧 PF 损伤对 Stroop 任务的表现产生不同的影响(Alexander 等人,2007 年),内侧和背侧 PF 损伤对任务切换有不同的影响(Shallice 等人,2007 年), 并且内侧 PF 病变和右侧 PF 病变对持续注意力有不同的影响 (Stuss et al. 2005)。 Stuss (2006) 以及 Shallice 和 Cooper (2011) 提出,这些和其他数据指向监督注意系统中的功能专业化。
\par 
我们的提案包含了这些想法的某些方面。 例如,在第 8 章中,我们回顾了当人或类人猿从事新任务或任务情况发生变化时颗粒 PF 皮层变得更加参与的证据,我们指出这种参与随着任务变得自动化而减少。 因此,我们的提议结合了 PF 皮层在行为注意力控制中的作用。
\par 
然而,其他皮层区域也在这种控制中发挥作用。 Baddeley 和 Sala (1998) 得出的结论是,“中央执行机构”可以被视为一个监督注意力系统,其他区域,如后顶叶皮层,也在这些功能中发挥着至关重要的作用。 如果是这样,那么监督注意力理论就无法通过特异性测试。 然而,我们接受该理论可能在未来的某个时间被重新表述以使其特定于 PF 皮层。
\par 
此外,这些研究中的患者有大的病灶,定位的尝试依赖于绘制病灶重叠的相对粗糙的技术,没有对可能在距离重叠区域很远的地方产生损伤的白质损伤进行任何认真的评估 . 第 1 章解释说,正是出于这个原因,我们非常依赖可以在类人猿研究中进行的更具选择性的灰质损伤。
\par 
监督注意力理论也未能解决解剖学测试,因为它没有说明连接如何允许 PF 皮层做它所做的事情。 它不涉及历史测试。 而且,与其他引用执行功能概念的理论一样,这些想法的表述非常笼统,以防止证伪。
\subsection{多需求系统、全局工作空间和通用智能}
我们之前谈到了与类人猿研究相关的多需求理论和一般问题解决的各个方面。 然而,这些想法的主要推动力来自对人类的研究。 Dehaene 等人。 ( 1998 ) 例如,建议使用术语全局工作空间来描述 PF 皮层的功能。 他们的想法引用了领域通用性的概念,即能够从更专业的系统中调用信息并整合该信息以做出决策和选择。 通常,工作空间被视为连续行动,因此它一次只能处理一个选择。
\par 
全局工作空间理论与 Duncan (2010b) 提出的多重需求理论密切相关。 在影像学研究的荟萃分析中,Duncan 和 Owen (2000) 描述了背侧 PF 皮层中用于各种任务的激活,包括那些测试感知、运动学习和工作记忆的任务。 当人们进行流动推理时,激活发生在几乎相同的区域(Duncan 等人,2000 年)。 将这两组发现放在一起,邓肯提出,大部分 PF 皮层的功能是支持几乎任何困难或苛刻的行为。
\par 
这些理论指向 PF 皮层的重要功能,例如解决问题和 PF 皮层跨认知域整合信息的能力。 我们的建议与全球工作空间和多重需求理论一样,强调跨感官模式和认知领域的信息整合。 我们同意 PF 皮层与在苛刻情况下解决问题、跨感官和认知领域解决问题以及对行为的细心控制有关。
\par 
然而,正如所阐述的那样,这些理论未能或未能解决多项测试。 他们不涉及历史测试。 他们未能通过特异性测试,因为 Duncan 的多重需求系统和 Dehaene 等人的全局工作空间理论。 包括后顶叶皮层和前额叶皮层。 例如,Woolgar 等人。 (2010) 表明,前额叶和后顶叶病变都会损害进行流体推理的能力。 并且成像实验经常导致PF皮层、后顶叶皮层和一部分内侧额叶皮层的联合激活。 例如,Duncan ( 2010b ) 就是这样定义他的多重需求系统的,所以它从一开始就没有通过特异性检验。
\par 
这些理论似乎也缺乏反驳所需的精确性,因此在它们目前的版本中,它们可能无法通过可证伪性测试,因此我们在表 10.2 的适当列中输入查询。
\subsection{概括}
与基于类人猿研究的提案一样,那些主要基于人类研究的提案未能通过所应用的五项测试中的一项或多项。 这些理论都没有恰当地解决历史检验问题,有些理论永远也做不到。 此外,这些理论往往很少关注连接解剖学,因此无法通过解剖学测试。 一些理论未能通过特异性测试,因为它们没有说明 PF 皮层做了哪些后顶叶皮层没有做的事情。 其他理论仅依赖于少数任务,因此无法通过普遍性检验。 有些理论可以解释任何事情,因此无法通过可证伪性测试。
\section{我们的提案根据相同的标准进行评估}
在考虑文献中的理论时,我们一直持批评态度,但很明显,我们的提议与其中许多理论有相似之处。 事实上,如果没有,那将很奇怪,因为理论依赖于数据,而 PF 皮层理论试图解释类似范围的数据。 然而,我们认为替代理论未能通过我们的提议可以通过的测试。
\par 
所以现在我们评估我们自己的提案,我们通过两种方式进行评估。 首先,我们考虑在根据历史、解剖学、特异性和普遍性测试进行评估时它的表现如何。 当然,我们在设计提案时考虑了这些标准,但我们仍然需要解释它是如何满足这些标准的。 其次,我们提出各种方法来反驳我们的提议,从而满足可证伪性测试。
\subsection{历史测试}
灵长类 PF 皮层的综合理论必须考虑到进化。 第 2 章解释了大部分颗粒状 PF 皮层首先出现在早期灵长类动物中,而额外的 PF 区域是在类人猿灵长类动物中进化而来的。 我们提出了这些新区域赋予早期灵长类动物和类人灵长类动物在其进化史上特定时间和地点的具体优势(第 2 章和第 8 章)。 我们认为它们的新 PF 区域为早期灵长类动物提供了适应细枝生态位的选择性优势。 类人猿进化的额外 PF 区域使它们能够根据单个事件生成目标,从而减少在食物匮乏和消耗期间做出的觅食选择错误。
\par 
我们强调,在发展这些想法的过程中,我们的分析基于证据,而不是像现有文献中有时出现的那样,基于关于进化“必须”如何运作的假设。 当然,我们需要更多的证据,而且我们知道并非我们拥有的所有证据都具有相同的可靠性水平。
\subsection{解剖学测试}
第 3 章到第 7 章解释了 PF 皮层的连接如何决定它的功能,第 8 章将这一概念扩展到整个 PF 皮层。 以眼眶 PF 皮层为例。 第 4 章指出,它与嗅觉、味觉、内脏、体感和视觉皮层以及杏仁核的联系使其处于独特的位置,可以对食物和液体的特定感官特性进行编码,并评估它们的价值 -最新的生物学需求。 同样,第 6 章和第 7 章解释了背侧和腹侧 PF 皮层的功能部分取决于它们分别与后顶叶皮层和颞叶皮层的连接。
\subsection{特异性试验} 
一个成功的理论必须解释为什么 PF 皮层可以执行提议的功能,但大脑的其他部分不能。 在关于 PF 皮层细分的每一章(第 3-7 章)中,我们将 PF 皮层的功能与其他区域(例如后顶叶、颞叶、前运动或海马皮层)的功能进行了对比。 第 8 章将这一论点扩展到整个 PF 皮层。 在我们看来,这还不够,所以说说灵长类动物 PF 皮层的作用。 在这本书中,我们还说了其他大脑区域无法做到的事情。
\subsection{一般性测试}
普遍性检验要求成功的理论必须考虑 PF 皮层可用数据的广泛范围以及 PF 皮层所有部分的功能。 表 8.1、8.2 和 9.1 分别汇总了损伤效应、细胞活性特性和成像激活的列表。 如前所述,文献中的许多理论仅依赖于一项或几项任务,或者专注于 PF 皮层的一个或几个部分。 与文献中的许多其他提案相比,我们的提案涵盖了更广泛的任务和细胞类型。 它还占 PF 皮层的所有部分。 这样,它就通过了通用性测试。
\section{我们提议的可证伪性}
如果一个理论未能通过可证伪性测试,那么它是否满足其他标准并不重要。 在建议对我们的提案进行测试时,我们牢记它们是可行的,并且不依赖于在不久的将来几乎没有应用前景的方法,例如从皮层中的所有神经元同时记录或开发检测单个细胞活动的成像方法。
\par 
在考虑与我们的建议相冲突的观察结果时,我们通过历史、解剖学、特异性和普遍性测试来组织它们。 对于特异性测试,我们将讨论分为涉及 PF 皮层以外区域的测试和涉及灵长类动物以外物种的测试。
\subsection{历史测试}
我们的理论强调 PF 皮层的进化历史,我们的观点无疑会引起神经科学家的注意。 对神经科学家来说,关于大脑进化的讨论似乎是推测性的,即使是基于可靠的证据。 有几个因素促成了这种看法。 首先,我们没有我们想要的所有证据。 当然,这种困难适用于神经科学中的所有事物,但对于数百万年前发生的事件而言,它似乎更加令人生畏。 其次,大脑和轴突连接不会变成化石,因此古生物学对大脑的洞察不如对牙齿和骨骼的洞察。 第三,神经科学家常常对趋同和平行进化的可能性抱有不适当的怀疑。 导致猫和类人猿的独立进化是否会产生正面导向的眼睛、视网膜特化(如中央凹)、平滑追踪眼球运动以及皮质区域(如颞下皮层)? 对于许多神经科学家来说这似乎不太可能,但它确实发生了。 例如,中央凹在脊椎动物历史上独立进化了很多次,包括在某些蜥蜴、早期鸟类和早期单缘灵长类动物中 (Ross 2004)。
\par 
因此,在狗、猫或羊身上发现类似于下颞叶皮层并投射到某人称为前额叶皮层的东西,不会对我们关于灵长类动物 PF 皮层进化的想法构成挑战。 另一方面,非灵长类哺乳动物腹侧或中外侧 PF 皮层同系物的令人信服的证据将与我们的提议相矛盾。
\par 
请注意,在考虑对我们提议的反对意见时,将一个区域称为前额叶区域或举另一个例子,称为额极是不够的。 拟议的同源性必须得到一系列属性的支持,这些属性共同识别该区域并将其与其他区域区分开来。 由于第 2、3 和 4 章都解释的原因,仅指出相似之处是不够的。 这些标准必须是特定区域的诊断标准。 这个标准是严格的,但如果非灵长类哺乳动物具有颗粒状 PF 皮层的同源物,一般或特定区域,如中外侧或极 PF 皮层,则可以满足。
\par 
更一般地说,任何严重破坏我们关于灵长类 PF 皮层进化史结论的发现都会挑战我们的提议,因为这些结论是其基础。
\subsection{解剖学测试}
灵长类动物 PF 皮层的完整理论应该可以解释为什么它的连接能够实现所提出的功能。 我们在第 3 章到第 8 章中尝试过这样的解释,但正如我们在第 1 章中解释的那样,公认的解剖结构会不时发生变化。这是应该的。神经解剖学家会犯错误,因为他们使用的方法有局限性,而且通常是严重的局限性。 这个问题既适用于连通性分析,也适用于皮层区域的划分。
\par 
在第 1 章中,我们指出了独立于观察者的细胞构造分析的价值。 总有一天,类人猿和人类都可以使用这种地图,并且已经出现了一些合理的初始步骤。 加深对颗粒状 PF 皮层细分的理解可能会改变我们解释我们在第 3-7 章中描绘为连接指纹的连接的方式。 例如,当我们说连接在某个区域终止时,我们的解释可能是错误的。这些轴突可能终止于我们现在认为属于一个区域的一小块皮层,但有朝一日,独立于观察者的定义可能会导致其包含在其他区域。这些修订可能会挑战我们的论点,或者至少会导致实质性修订。
\par 
解剖学上的微小变化不会严重挑战我们的提议。 我们期望随着新结果的出现或更多已发表的解剖学结果进入讨论而发生变化。 但重大分歧可能会大大破坏我们的提议。 例如,如果未来的研究表明 PF 皮层对运动前皮层的输出很少,那么理解皮层如何将目标转化为行动目标就会遇到困难。 几十年来,人们对从 PF 到前运动皮层的连接的功能意义、范围和组织的态度时好时坏。 未来,主流观点可能会再次改变。 同样,如果未来的神经解剖学家得出结论,颗粒状 PF 皮层的后顶叶投射来自顶叶的一个受限部分,该部分缺乏处理时间、顺序和空间信息的特性,那么人们可能会拒绝我们对背侧如何 PF 皮层做它做的事。
\par 
解剖学测试要求我们说出它的连接如何让 PF 皮层执行其功能。 如果实验产生的证据表明,将 PF 皮层的各个部分彼此断开,其关键功能完好无损,我们的提议就会失败。 例如,来自海马体的事件相关输入影响中外侧、腹侧、背侧和极 PF 皮层的目标生成的想法假设它们至少部分通过内在 PF 连接来实现(第 8 章) . 如果一个实验程序阻止了这些内在联系,并且先进的行为继续有增无减,我们的提议就无法涵盖这样的结果。 海马体所有 PF 目标的失活或断开连接可能会阻断必要的中继。 我们并不是要特别挑出海马输入,尽管它们看起来特别有趣; 类似的方法可以测试上下文信息的其他来源,例如来自后顶叶、颞叶或嗅觉皮层的信息。
\subsection{特异性测试:其他领域}
如果有证据表明大脑的其他部分做我们所说的 PF 皮层做的事情,特别是如果它相对直接进入运动前区,我们的提议就会失败。
\par 
关于进入运动前区的想法会不时发生变化。 例如,基底神经节的传统观点认为它是来自许多皮层区域(例如颞下皮层)的感觉信息的汇聚点,并通过丘脑输出到前运动皮层和初级运动皮层。 过去,神经解剖学家将基底神经节视为从下颞叶皮层和 PF 皮层到运动前区的路径 (Kemp和 Powell 1971)。 正如我们在第 8 章中解释的那样,由于 Strick 和他的同事 (Middleton和Strick 2000) 的工作,这种观点已经过时了,他们强调平行的皮层-基底神经节环路,其中只有一些环路可以访问前运动 和初级运动皮层。 但传统的想法可能会在未来重生或重新受到重视,如果真的发生了,我们将不得不修改我们提案的某些方面。
\par 
大脑皮层的其他部分与颗粒状 PF 皮层密切合作,包括后顶叶皮层。 正如第 6 章所解释的,我们将后顶叶皮层视为向颗粒状 PF 皮层提供上下文信息。 但是,如果发现后顶叶皮层的某些部分基于该上下文生成目标,使用非空间和空间上下文以及有关特定结果的信息,这些发现将与我们的提议相矛盾。 目前,我们认为现有数据不会反驳我们的提议,但未来的结果可能会。
\par 
我们之前提到,海马体也具有颗粒状 PF 皮层的许多特性。 条件视觉运动映射期间海马体的细胞活动发生变化(Cahusac 等人 1993 年;Yanike 等人 2009 年),这与颗粒状 PF 皮层中的情况非常相似(Pasupathy 和 Miller 2005 年)。 如果未来的研究表明海马体执行我们归因于颗粒状 PF 皮层的功能,我们的提议就会失败。
\par 
进一步的研究还可以比较与我们的建议相关的信号的时间,尤其是那些反映目标生成的信号。 图 10.2 显示,在编码当前目标的信号在中外侧 PF 皮层中发展之前,策略信号在眼眶 PF 皮层中发展(Tsujimoto 等人 2011a)。此示例比较两个 PF 区域,但相同的方法可以比较任意两个区域中的信号时序。 如果目标信号在 PF 皮层以外的区域在专注控制的目标生成过程中更早地发展,这将被视为反对我们提议的证据。
\par 
乍一看,此类数据似乎已经可用。 Wallis 和 Miller (2003a) 比较了 PF 皮层和背侧运动前皮层中规则编码信号的延迟。 正如我们在第 7 章中解释的那样,这些细胞编码了与样本匹配和不匹配的规则。规则信号在出现在 PF 皮层之前在前运动皮层中形成。 这一发现似乎对 PF 皮层产生目标的想法提出了挑战。 然而,正如第 8 章所解释的,关于目标生成的建议仅适用于注意力控制。 当类人猿和人自动控制他们的行为时,PF 皮层变得相对脱离。 这项任务的结果依赖于类人猿已经过度学习到自动化程度的规则(Wallis和 Miller 2003a;Muhammed 等人 2006)。 在抽象行为指导规则或策略的学习过程中,没有人研究过 PF 和运动前皮层神经元活动的相对时间。
\subsection{特异性试验:其他物种} 
如果我们认为粒状PF皮层在灵长类动物与其他哺乳动物谱系分化后进化是正确的(第 2 章),那么PF皮层理论不能完全依赖于非灵长类哺乳动物的数据。然而,目前对啮齿类动物PF皮层的看法认为,它要么由灵长类动物所有 PF 区域、颗粒状和非颗粒状的混合物组成,要么由灵长类动物PF皮层的微型化版本组成。无论哪种观点,啮齿动物PF皮层的功能都被认为与灵长类动物PF皮层的功能或多或少相同。如果这个观点是正确的,那么我们的建议就是错误的。它将无法通过特异性测试,因为我们归因于灵长类动物特定区域的功能可以由大脑的其他部分执行。如果他们可以在其他动物身上这样做,那么同源区域可能在灵长类动物中具有相同的能力。至少,这样的发现将更多地转移到我们这边的“证明”负担上。
\par 
我们在第3章中针对内侧 PF 皮层和第4章中的眼眶 PF 皮层分别简要地解决了这个问题。第2章处理了许多一般性问题,仅简要提及了行为证据。在这里,我们更详细地讨论了行为证据。
\par 
就其本身而言,仅来自老鼠的行为数据并不能告诉我们关于类人猿灵长类动物的任何信息。 添加来自小鼠的观察结果不会增加更多,因为这些密切相关的鼠类啮齿动物最近发生了分歧。 趋同和平行进化很常见,啮齿动物和灵长类动物谱系分裂了约 70–90 Ma(见图 2.8)。 现代啮齿类动物和灵长类动物因此在很长一段时间内分别进化,在此期间它们已经适应了它们遇到的问题和机遇。在许多情况下,他们可能遇到过类似的问题或机会,并且可能以类似的方式解决或利用它们。
\par 
图 2.8 显示了如何选择足够多样性的哺乳动物进行研究。我们在谱系上方放置了一个标签(‘#’),它们共同告诉我们许多哺乳动物的共同祖先。如果不提及最后的共同祖先,现代哺乳动物之间的相似性几乎没有证明价值。
\par 
为了使这个建议具体化,我们指出了四组哺乳动物供未来研究:马岛猬、鼹鼠、兔子和刺猬。 当然,人们可以选择其他群体中的物种,但为了讨论起见,这些哺乳动物就可以了。 结合来自啮齿动物和灵长类动物的信息,大多数或所有这些物种的共同特征应该反映出它们最后共同祖先的特征,在图 2.8 中也用标签(“#”)标记。 所有这些物种都是小动物,可以在标准心理学实验室进行可行的修改后进行测试。 所有这些都由神经解剖学家研究过,其中两个是家庭宠物。 研究这些生物的障碍是概念上的,而不是实用上的。
\par 
所有或大多数这些物种执行我们归因于类人灵长类动物的颗粒状 PF 皮层的行为的能力——在同一水平上——将证明这些特征是在早期哺乳动物中进化的。 如果是这样的话,那我们的提议肯定是错误的。
\par 
尽管只比较两个现存物种的原则几乎没有告诉我们任何信息,但我们意识到许多读者会对老鼠和类人猿的比较感兴趣。 我们通过考虑延迟反应任务和据说评估空间记忆的其他测试来开始我们对该主题的讨论。 在比较老鼠和类人猿在这些任务上的表现时,我们需要清楚地区分三个问题:
\par 
1. 鉴于大鼠缺乏颗粒状 PF 皮层,它们能否执行依赖于类人猿颗粒状 PF 皮层的任务?
\par 
2. 对于依赖于类人猿颗粒状 PF 皮层的任务,大鼠颗粒状 PF 皮层损伤后是否会出现明显的损伤?
\par 
3. 当问题二的答案是肯定的时,这些损伤在大鼠和类人猿身上的损伤后是否具有相同的特征?
\par 
第 6 章解释了背侧 PF 皮层,尤其是它的中外侧部分(区域 46),对于正确执行类人猿的延迟反应和延迟交替任务至关重要。 在去除中外侧 PF 皮层之前没有接受过训练的类人猿根本无法学习延迟反应任务,即使在以 1 秒的短暂间隔进行 500 次试验后也是如此(Battig 等人,1960 年)。 背侧 PF 皮质损伤较大的类人猿在 4000 次试验后也无法学习延迟交替任务 (Goldman和Galkin 1978)。 当类人猿在损伤之前已经学会了任务规则时,双侧 PF 皮层损伤导致延迟反应任务的表现下降到机会水平,并无限期地保持在那里。 以 5 秒的延迟间隔,一些类人猿在机会水平上进行 1000 次试验,此时实验者通常会停止对它们进行测试(Mishkin 1957 年;Goldman 等人 1971 年)。 当以 4 秒间隔进行训练时,即使以更简单的 2 秒间隔进行测试,类人猿仍保持在机会水平 (Kojima ,1982)。 因此毫无疑问,在类人猿中,执行这些任务需要中外侧 PF 皮层,这是一个颗粒区域。
\par 
在这种情况下,一些神经科学家认为老鼠可以学习这种空间记忆任务意义重大。 这一发现回答了问题一。 大鼠可以学习工作记忆任务,例如 8 项选择径向臂迷宫任务(Olton 等人 1982 年;Kesner 1989 年)、延迟反应任务(Kolb 等人 1974 年)和延迟交替任务(Thomas和 Brito 1980 年;班纳曼等人 2001 年)。 他们还可以学习延迟匹配到位置和延迟非匹配到位置任务,并且无论是在操作室 (Sloan,2006) 还是 T 迷宫 (Dias和 Aggleton 2000) 中进行测试都可以这样做。
\par 
老鼠执行这些任务的能力不足为奇,因为与类人猿可以执行的许多任务相比,它们很简单。此外,正如第5章和第6章所解释的,这些任务不测试特定的认知过程,因此可以通过许多不同的方式解决它们。
\par 
问题二涉及大鼠的损伤效应。 第 3 章解释了一些神经科学家的观点,即大鼠的内侧 PF 皮层是一个颗粒状区域,与类人猿和人类的中外侧PF皮层是同源的,一个颗粒状区域。 为了支持这个想法,Kolb 等人。 (1974) 发现大鼠内侧 PF 皮层的损伤导致延迟反应任务的显着缺陷。 还有其他类似的结果。 我们对这些损伤效应的重要性或严重性没有异议。 但正如我们在第 3 章中解释的那样,这些发现并没有说明同源性,因为在类人猿中,这些任务的显着损伤也伴随着无颗粒 PF 皮层的损伤,包括前扣带皮层和前边缘皮层(Meunier 等人,1997 年;Rushworth ,2003 年)。 因此,如果简单地询问内侧 PF 皮层的损伤是否会对大鼠的延迟反应任务造成显着损害,答案是会。 但是这个结果并没有为识别同源性提供合适的诊断标准。
\par 
问题三超出了将损伤后行为简单分类为受损或正常表现的范围。 对大鼠和类人猿损伤效应细节的检查表明,大鼠和类人猿解决问题的方式不同。 首先,在延迟反应任务(Kolb 等人 1974 年)、径向臂迷宫(Kesner 1989 年)和操作室中的位置匹配或不匹配(Sloan  2006 年)的实验中,没有采取任何措施来 防止老鼠通过定向到正确的一侧或接近它来弥补延迟。 其次,如果动物在 T 迷宫中进行延迟交替测试,因此无法通过定向策略解决问题,则试验间隔非常大,动物可以根据熟悉度或新近度来解决问题(Sanderson 等 等人,2010 年)。
\par 
大鼠也有基于记忆导航到位置的方法(Kolb,1994 年),并且内侧 PF 皮层和海马体之间的连接在某些空间任务中调节此功能。 这些结构的损伤导致大鼠必须在空间场中导航的任务受损,例如 Morris 水迷宫任务和径向臂迷宫任务 (Kolb 1994)。 正如在 T 型迷宫中测试的那样(Markowska 等人,1989 年),穹窿的横切会导致延迟的非匹配样本任务受损,并且以相同方式测试的猕猴也会出现相同的结果(Murray 1989 年)。因此,这些任务的损害可能反映了导航系统的损坏,在类人猿中,导航系统与到达系统并行运行。 然而,我们注意到,海马体和周围区域有损伤的大鼠在延迟反应任务中表现正常,除非延迟超过 30 秒(Alvarez 等人,1994 年)。因此,对于这个特定任务,导航方面的帐户似乎不太可能。
\par 
尽管它们很重要,但大鼠内侧 PF 皮层损伤后的损伤并不像患有中外侧 PF 皮层损伤的类人猿的延迟反应和延迟交替任务损伤那么严重。 内侧 PF 损伤后,接受延迟反应任务训练的大鼠在 140 次试验中可以在 2 秒的中值间隔内达到 90$\%$ 的正确表现(Kolb 等人,1994 年)。 在另一项空间记忆测试中,具有类似病变的大鼠最初表现出轻微的损伤,但仅在 60 次试验中就恢复了正常水平的表现(Kolb 等人,1974 年)。 研究人员需要使用滴定程序来证明损伤效果,因为损伤大鼠在短时间内成功解决了问题(Kolb,1994 年); 中外侧 PF 皮层受损的类人猿永远无法做到这一点。
\par 
根据我们目前所说的,我们可以拒绝内侧PF皮层与灵长类动物的中外侧PF皮层(区域 46)或任何其他颗粒状 PF区域同源的想法。 Brown 和 Bowman (2002) 接受了这个结论,试图提出类比的理由,而不是同源性。 按照这种观点,啮齿类动物的内侧PF皮层和灵长类动物的颗粒状PF皮层都在空间工作记忆中发挥作用,因此它们具有“相同”的功能。 然而,在第 6 章和本章中,我们解释了工作记忆理论对于灵长类 PF 皮层的不足,一般来说,或者特别是对于中外侧PF皮层(46 区)。 类比论证的关键问题在于它无法区分动物所从事的许多不同类型的空间分析。例如,类人猿在空间中航行,在空间参照系中伸手去拿和操纵物体,判断相对 精细和粗略尺度上的距离,并将一个地方与另一个地方区分开来。 空间信息可以根据内在参照系或外在参照系来计算,通常分别称为自我中心和非自我中心。 迄今为止设计的行为任务没有足够精确地区分这些因素,无法将内侧 PF 皮层与灵长类动物PF皮层的任何颗粒部分进行类比。
\par 
因此,关键问题是老鼠和类人猿是否以相同的方式解决延迟反应任务。 第 6章解释了我们如何认为类人猿解决了延迟反应任务,并且该解释诉诸了干扰、规则和前瞻性编码的概念。通过前瞻性编码,我们指的是在目标不可见时在内存中保持目标,例如,在延迟响应任务的延迟期间。 我们的研究认为,类人猿通过使用规则(最近的视觉事件的位置告诉它们下一步该做什么)和前瞻性编码该目标来执行延迟反应任务,以克服先前事件的干扰,包括先前的线索和目标 选择。我们会考虑证明老鼠正是以这种方式执行延迟反应任务的证据,如果它可以产生的话,将会引起相当大的兴趣。 如果有证据表明大鼠额叶皮层某些部分的损伤导致的损伤与类人猿中外侧PF皮层损伤引起的损伤一样严重,那将更加有趣。 如果类似的结果出现在多种哺乳动物中,按照前面解释的思路,那就表明我们的提议是错误的。
\par 
读者可能仍然想知道为什么当类人猿在中外侧 PF 皮层损伤后大脑的大部分也完好无损时,它们无法解决延迟反应任务带来的简单问题。第6章给出了前一段重申的原因。换句话说,有缺陷的类人猿要么:(1)无法理清过去视觉事件的顺序,(2)不能记住或使用最近事件指导当前选择的规则,或(3)不能 使用前瞻性编码来克服来自以前试验记忆的干扰。
\par 
由于其独特的进化历史,类人灵长类动物通过这种方式解决了延迟响应任务带来的问题。因此,正常的类人猿不需要诉诸定向策略——所以也许由于这个原因,受损的类人猿也不会这样做。此外,由于正常的类人猿依赖于以注视为中心的参照系(第 2 章和第 5 章)到达,它们依赖于它们的中外侧 PF 皮层为运动前区提供该坐标系中的目标,尤其是对于不可见的目标。(对于可见目标,顶叶后部和运动前区就足够了。)说类人猿使用视觉参照系来触及不可见目标似乎有悖常理,但 Shadmehr 和 Wise (2005) 引用的证据简要总结于 第 2 章表明情况确实如此。
\par 
到目前为止,我们已经通过延迟响应任务和空间信息处理的相关测试来解决这个问题。 但是颗粒状PF皮层的功能远远超出了延迟反应任务所评估的功能。在第8章中,我们提出了灵长类动物PF皮层赋予的一组更普遍的优势:快速学习、通过应用抽象规则和策略来减少错误的能力,以及将选择与基于单个事件的结果联系起来的能力。
\par 
请注意,这些优势中没有一个具有全有或全无的性质。 我们并不声称其他哺乳动物在某些特殊情况下无法快速学习(类似于味觉厌恶效应或小鸡的印记)。 我们不认为抽象规则和策略的使用本身仅限于灵长类动物。 对于选择-结果或刺激-结果关联,我们只说灵长类动物进化了两个关键进步,涉及视觉优势和基于单一事件的信用分配。 其他哺乳动物可以看到,其他哺乳动物可以学习刺激-结果关联,其他哺乳动物可以快速学习,而其他哺乳动物可以使用抽象策略。 我们只是声称,在它们的进化过程中,类人灵长类动物能够更好更快地做出觅食选择,并且由于这些能力的进步而比它们的祖先出现更少的错误。 自然选择需要优势,而不是全新的能力。 一次试验学习和十次试验学习之间没有神奇的区别,只是后一种情况下的额外错误。 并且通过并行演化,可以通过不同的方式实现相同的容量。 因此,我们的论证不需要我们说出类人猿(或其他灵长类动物)可以完成老鼠(或其他哺乳动物)不能完成的特定任务。
\par 
然而,我们发现有趣的是,灵长类动物已经掌握了一些啮齿动物从未有过的任务。 没有老鼠曾经学习过Genovesio等人的策略任务。( 2005 ) 报道了类人猿(第 7 章)。据我们所知,还没有人尝试训练老鼠来完成这项任务,但如果有人能做到这一点,我们会觉得非常了不起。
\par 
虽然我们认为没有必要,但考虑到啮齿动物可能无法完成的任务在启发式上是有用的。 为了公平地进行这种比较,我们应该提出一项尚未有人教老鼠或类人猿完成的任务。 所以我们预测类人猿,而不是老鼠,可以掌握以下任务:学习了 Genovesio等人的“重复-停留”和“变化-转移”策略任务。 ( 2005 ),受试者现在必须学会逆转这些策略,转而发展“重复-转移”和“改变-停留”策略,并像在逆转集实验中那样反复逆转(第 4 章)。 这样做的能力将反映我们怀疑需要颗粒 PF 皮层的学习能力。
\par 
在另一项任务中,受试者稍后需要学习使用提示在这两组策略之间切换:“重复-停留”和“改变-转移”以响应一个提示; 响应另一个提示的“repeatshift”和“change-stay”。 此任务最易于处理的版本将使用大约五个试验的块,因为这些策略处理的是试验中携带的信息。
\par 
第三个这样的任务可能涉及依赖于类人猿颗粒状PF皮层的行为组合。 例如,如第 7 章所述,受试者首先会在条件视觉运动任务上形成强大的学习集。 对于三种新颖的刺激,每一种都会指导受试者三种行为中的哪一种会产生奖励。 我们知道类人猿只需几次试验就能学会这些映射。 接下来,受试者将学习物体就地场景任务以熟练掌握,第 3 章和第 8 章对此进行了解释。 与任务的标准版本不同,每个场景将包含三个可能的选择。大鼠版本的任务可以使用色调作为背景和气味作为选择。 在这次培训之后,受试者需要将这两项任务结合起来。 他们必须根据背景刺激(上下文)学习选择哪个物体(或气味),以及基于试错法学习与该物体(或气味)相关的单独动作。 他们每个问题只有 20 次试验来学习这些关系,此时将呈现一组新的刺激和背景。 我们预测类人猿可以学习这样的任务,但老鼠可能做不到。
\par 
当然,任何学习特定任务的失败都可能只是反映了训练方法的不足或主题学习的时间不足。例如,猕猴可以学习反向奖励权变任务,在该任务中它们必须学会选择两个食物量中较小的一个以获得较大的食物量(Murrayetal.2005)。然而,Silberberg 和 Fujita (1996) 的一份报告称,类人猿无法解决这个问题。 他们未能教给他们的动物这项任务的原因很简单:他们没有让他们的类人猿进行足够的试验。 在同一任务上有了更多的经验,类人猿确实可以很好地学习这项任务。 但是,六只类人猿平均需要进行 1087 次试验才能学会,而他们只允许他们的受试者进行这种体验的一小部分。
\par 
尽管这里提出了争论,但我们预计许多神经科学家会认为我们的建议被削弱,因为观察到一些其他哺乳动物可以学习与类人猿相同的东西并且可以有效地学习。例如,狗具有快速映射声音的能力,类似于人类学习词义的方式(Kaminski等人,2004年)。 正如我们之前解释的那样,我们的提议可以容纳少数物种中这种平行或趋同进化的有限数量的例子。 但只有这么多。
\par 
辨别学习集提供了文献中已有的示例。 我们在第 8 章中解释说,类人猿在第二次试验中的表现接近 90$\%$,并且强大的学习集取决于下颞叶皮层和 PF 皮层之间的相互作用。 我们将这一发现放在其他人的背景下,在这些背景下类人猿可以使用单个事件来减少错误。 因此,如果老鼠能够发展出与类人猿一样快且程度相同的辨别学习集,我们的立场将受到挑战。
\par 
Slotnick 等人的一篇论文。 (2000) 提出了这样的主张,但由于两个原因,它是不能令人信服的。 首先,他们使用了一个去不去的任务。 对于每一个问题,老鼠都将鼻子伸进一个房间里闻气味。 实验者指定一种气味为“去”刺激,另一种为“不去”刺激。 这个任务有一个严重的缺陷:“go bias”。 因为“去”的成本非常小,默认情况下动物会付出代价,然后它们可以更快地进入下一个试验。 事实上,Slotnik 等人。 报告说,他们的老鼠犯的每一个错误都是委托错误,即对“不去”刺激的错误“去”反应。 因此,实验设计使“走”刺激与受试者几乎无关,这种设计缺陷意味着任务测试的是消退学习而不是辨别学习。
\par 
其次,也许更重要的是,Slotnik等人。设置他们的实验,使气味来自老鼠戳鼻子的同一个地方,奖励也发生在那个地方。因此,该实验在刺激、反应和奖励之间建立了完美的空间连续性。众所周知,这种空间连续性可以促进非常快速的学习(Cowey和 Weiskrantz 1968),事实上,Slotnik 等人研究的老鼠。 从训练开始就表现出高水平。 结果,他们的老鼠不必发展出辨别学习能力就可以很好地完成任务。 当奖励的位置与响应的位置分开时,老鼠学习一系列嗅觉辨别会慢得多 (Fagan 1985)。因此,Slotnik 等人的结果。 没有提供证据证明猕猴可以获得哪种辨别学习集(见图 8.2)。
\par 
Eichenbaum 等人的另一篇论文。(1986),被引用为展示老鼠的辨别学习集,但它没有提到关键试验,试验二的表现。
\par 
因此,在学习集上测试老鼠的条件与用于类人猿的条件大不相同。 对大鼠的学习集进行适当的测试需要在两种非食物刺激之间进行选择,这会导致稍后在不同的地方呈现奖励,就像在类人猿中进行的学习集实验一样。 视觉刺激应该完全足以测试这种能力,图 8.2 显示老鼠获得学习集的速度很慢。 在该文献中有一种普遍的感觉,即要求老鼠在视觉刺激中进行选择在某种程度上是不公平的,但他们在许多其他任务中这样做没有问题,只要实验者仔细选择刺激(Bussey 等人,2008 年)。 此外,松鼠的视力比老鼠好得多,而且它们的学习能力也较慢(见图 8.2)。然而,如果未来的研究表明,在与类人猿实验中使用的条件相当的条件下,在老鼠身上进行的第二次试验的正确性能接近 90$\%$,那么我们的提议将被视为无效。 如果在不同的哺乳动物样本中进行复制,就会表明我们的提议可能是错误的。
\par 
视觉运动映射的一次性学习也是如此。 类人猿和老鼠(Dunn  2005 ;Dumont 2007 )都可以学习条件视觉运动任务。 新映射的快速学习需要颗粒状 PF 皮层(第 7 章),但慢速学习则不需要(Bussey 等人,2001 年)。 如果可以证明老鼠可以在一次或几次试验中学习新的条件运动映射,就像类人猿一样,并且额叶损伤阻碍了一次试验学习,就像它们在类人猿身上所做的那样,结果将挑战我们的结论。 如果这些发现在多种哺乳动物中得到证实,它将被视为反对我们提议的证据。
\par 
最后,请注意。在第 3 章和第 4 章中,我们提出了关于无颗粒 PF 皮层功能的想法。 我们说这些区域在大鼠和类人猿中是同源的(见图 2.1)。 然而,这一论点并不意味着大鼠和类人猿的每一个连接、每一个细胞活动特性、每一个成像激活和每一个损伤效应都是相同的。 每个谱系已经分别进化了大约 70-9000 万年,并且在这么长的时间间隔内可能已经发生了相当多的变化。 特别是,当新的颗粒区域在灵长类 PF 皮质中进化时,这可能会导致旧区域发生重大变化。 如果这个警告看起来不公平,那不是我们的错;这就是进化的方式。
\subsection{一般性测试}
通用性测试要求我们的提议考虑PF皮层及其所有部分的不同数据。 然而,在我们对内侧PF皮层的讨论中,我们并没有过多地谈论区域 9 和 10,因为关于这些区域的实验工作很少被报道,尤其是在类人猿身上。 我们还引用了一些关于类人猿中颗粒状 PF 区域的结果。 我们对缺乏数据无能为力,但这必须被视为我们提案的弱点。
\par 
我们相信我们的提议相当好地解释了已发表的文献,但如果评论家可以指出我们的提议未能解释的一些重要发现,它至少会导致重新表述。 同样,如果批评者能够指出我们的提议未能涵盖的 PF 皮层的某些部分,撇开缺乏证据不谈,那也会对我们不利。
\chapter{结论}
这本书提出了关于灵长类动物前额叶皮层基本功能的方案.

\section{摘要}
本章将我们的建议与文献中的其他建议进行比较,并通过五个测试对每个建议进行评估:(1)该建议是否考虑到了PF皮层的进化史?(2)它是否解释了连接解剖学如何允许PF皮层执行所提议的功能?(3)它是否明确了PF皮层的功能与其他区域的功能有何不同?(4)是否与PF皮层的广泛发现相一致?(5)它的陈述是否精确到足以被测试?最后,我们提出一些测试我们建议的方法。
\section{介绍}
第一章阐述了五个目标,我们现在可以更全面地阐述这些目标:
\par
1.说颗粒状PF皮层让灵长类动物做了他们的祖先和其他哺乳动物做得效率较低的事情,如果有的话,以及说类人猿灵长类动物颗粒状PF皮层的扩张让他们比其他灵长类动物做得更好。
\par
2.说颗粒状PF皮层的轴突连接如何允许的,而不是其他区域,提供这些优势。
\par
3.作为一个整体解释粒状PF皮层的基本功能,并解释它的功能与大脑其他部分的功能有何不同。
\par 
4.在复杂的认知任务中,这个功能是如何解释在人类PF皮层中观察到的激活的。
\par 
5.解释我们的建议与文献中其他建议的不同之处,并告诉读者什么样的观察可以反驳它。
\par 
现在是时候评估一下我们在多大程度上实现了这些目标。第二章讨论第一个问题。它研究了PF皮层的进化,包括一些导致灵长类动物进化过程中特定进展的选择压力。我们认为,颗粒状PF皮层的进化是分阶段发生的,这些阶段伴随着其他进步,如视觉和手功能的发展。早期的前辈们的眼睛是向前看的,他们采用了新的移动、抓取和用手喂嘴的方式。在这些动物中出现了第一个颗粒状的PF区域,即PF尾部皮层和颗粒状的OFC。背侧、腹侧和极侧的PF皮层在类人猿灵长类动物中出现得更晚,因为它们变得更大,不得不与它们喜欢的食物严重短缺作斗争,更不用说竞争和捕食的威胁了。
\par 
第三-七章讨论第二个目标。这些章节解释了灵长类PF皮层各主要区域的连接如何允许它做它所做的事情,以及为什么只有它才能执行它的功能。我们逐个区域检查PF皮层,因为它的连接因区域而异:
\par 
1.第三章指出了内侧PF皮层与海马体、杏仁核和内侧前运动区之间的联系,这使得它能够更好地使用“内部”信号来指导在行动和行动规则之间的觅食选择,包括评估它们与所涉及的努力成本相关的当前价值。
\par 
2.第四章强调,眶侧PF皮层的连接使其能够更好地利用外部信号来指导觅食选择,在物体之间进行选择。眶PF皮层与许多感觉区域有联系,包括视觉、躯体感觉、味觉、嗅觉和内脏皮层。这些连接允许眶PF皮层发展有关特定行为结果的高维信息连接,并强调其视觉特征。我们认为,基于单个事件,细粒度OFC将特定结果分配给似乎导致它的特定选择。与杏仁核的相互联系根据当前需求提供了更新的结果评估。
\par 
3.第五章重点介绍了注意力和搜索功能。尾侧PF皮层从视觉区域接收到的连接传递了来自低阶和高阶视觉的信号,包括背侧和腹侧视觉流。基于与这些区域的皮质皮质连接以及控制脑干动眼肌核的皮质投射,尾侧PF皮层可以将显性和隐性注意力引导到潜在的行动目标上。我们认为,尾侧PF皮层(第8区),包括额叶眼场(FEF),参与了对学习产生的目标的搜索——目标导向的注意力——而不是反射性或刺激驱动的注意力。我们不认为FEF是眼球运动区或运动前区,而是将其视为前额叶皮层的一部分:一个将注意力导向有学习价值的物体和地方的部分,包括隐蔽注意力和显性注意力(以眼球运动的形式)。这样,它加强了中央凹和中央凹外信息的处理。
\par 
4.第六章强调了由背侧PF皮层产生目标,部分基于与后顶叶皮层的连接。这些投影提供了有关视觉事件的顺序、时间和位置的信息,这些信息构成了当前行为环境的重要部分。目标的产生也依赖于眶PF皮层关于与这些目标及其当前值相关的结果的信息。前额叶皮层的背侧提供了一种机制,可以消除以前事件记忆引起的干扰,它似乎是通过前瞻性地编码当前目标来做到这一点的,至少在一定程度上是这样。与前发动机区域的连接为实现这些目标提供了一条途径。
\par 
5.第七章提出,腹侧前额叶皮层根据视觉或听觉环境产生目标。它之所以能发挥这一功能,是因为它与下颞叶和上颞叶皮层、眶前皮层和杏仁核有联系。这些联系为它提供了视觉和听觉信号、预测的结果,以及根据当前生物需求对该结果的估值。符号由基本特征和整体对象之间的特征结合的中间层次组成。腹侧前额叶皮层使用同样的连接来应用抽象的规则和策略,从而将以前的经验转移到新的行为问题上。
\par 
在以这种方式将PF皮层拆开后,第八章将其重新组装起来。它将PF皮层作为一个整体来研究,并最终实现了我们的第三个目标:一个关于灵长类PF皮层基本功能的具体建议,重点是在灵长类中进化的区域。我们提出,颗粒状PF皮层生成的目标适合于当前环境和当前需求,并且它可以在单个事件的基础上做到这一点。因此,类人猿可以根据一个或几个经验解决广泛的问题,并且可以避免祖先强化学习机制中固有的许多错误。我们提出,在灵长类动物历史上的特定时间和地点(第2章和第8章),为了应对特定的适应压力,粒状PF区域进化为实现一种新的通用学习机制,这一机制增强了祖先的通用学习系统。祖先系统通过加强反馈来调整关联的强度,从而控制自动行为;灵长类动物解决问题的方式包括对行为的注意控制,以及更少的错误。
\par 
第9章讨论我们的第四个目标:解释在人类认知过程中观察到的基本PF功能是如何解释大脑激活的。它强调了重新再现的力量。因此,人类的PF皮层可以重新表示知觉状态、他人的意图和精神状态以及关系之间的关系。第9章还表明,人类的PF皮层阐述了PF皮层在其他类人猿中的功能;就像猴子的颗粒状PF皮层允许它们通过快速学习和抽象策略来避免错误一样,人类的PF皮层允许我们通过指令、模仿和心理试错行为来避免错误。
\par 
本章的剩余部分将讨论我们的第五个也是最后一个目标:将我们对灵长类动物PF皮层的描述与其他文献进行比较,并提出一些测试方法。因此,接下来的两个部分解释了其他的建议要么缺乏我们提供的进化视角,只适用于灵长类PF皮层的一部分,没有解释为什么它独特的连接组合解释了它的功能,没有说明灵长类PF皮层执行了哪些其他大脑区域不能执行的功能,未能解释PF皮层贡献的广泛行为,或者未能产生可验证的假设。
\par 
我们将可供选择的建议分为两组,分别在不同的部分进行讨论:一组主要依赖猴子的结果,另一组主要依赖人类的证据。当然,前一组的支持者也试图将其推广到人类,后一组的支持者也提到了他们从猴子身上得到的证据来支持他们的观点。所以我们这样划分只是为了方便,有时,当一个理论同时涉及猴子和人类的研究时,我们会在两个部分讨论。
\par 
我们遵循Wood和Grafman(2003)在表格中列出各种理论,并试图根据一套标准来评估它们。我们使用了五个标准,这表明一个成功的PF皮层理论应该:
\par 
1.结合PF皮层的进化史,特别是在灵长类动物中颗粒状PF皮层的出现和在高级类人猿中新颗粒状区域的出现:历史测试。
\par 
2.解释为什么PF皮层的连接解剖使其有可能执行所提议的功能:解剖测试。
\par 
3.识别PF皮层的特定功能,与大脑的其他部分形成对比:特异性测试。
\par 
4.考虑可用数据的广泛范围:通用性检验。
\par 
5.精确到可以被可行的观察所检验:可证伪性检验。
\section{目的}

\section{定义和术语}


\section{指纹}

\subsection{损伤和激活}

\subsection{损伤和活动}

\subsection{活动和激活}




\subsection{结论}



\chapter{结论} \label{chap:chap10}
这本书提出了关于灵长类动物前额叶皮层基本功能的方案.

\section{摘要}
本章将我们的建议与文献中的其他建议进行比较,并通过五个测试对每个建议进行评估:(1)该建议是否考虑到了PF皮层的进化史?(2)它是否解释了连接解剖学如何允许PF皮层执行所提议的功能?(3)它是否明确了PF皮层的功能与其他区域的功能有何不同?(4)是否与PF皮层的广泛发现相一致?(5)它的陈述是否精确到足以被测试?最后,我们提出一些测试我们建议的方法。
\section{介绍}
第一章阐述了五个目标,我们现在可以更全面地阐述这些目标:
\par
1.说颗粒状PF皮层让灵长类动物做了他们的祖先和其他哺乳动物做得效率较低的事情,如果有的话,以及说类人猿灵长类动物颗粒状PF皮层的扩张让他们比其他灵长类动物做得更好。
\par
2.说颗粒状PF皮层的轴突连接如何允许的,而不是其他区域,提供这些优势。
\par
3.作为一个整体解释粒状PF皮层的基本功能,并解释它的功能与大脑其他部分的功能有何不同。
\par 
4.在复杂的认知任务中,这个功能是如何解释在人类PF皮层中观察到的激活的。
\par 
5.解释我们的建议与文献中其他建议的不同之处,并告诉读者什么样的观察可以反驳它。
\par 
现在是时候评估一下我们在多大程度上实现了这些目标。第二章讨论第一个问题。它研究了PF皮层的进化,包括一些导致灵长类动物进化过程中特定进展的选择压力。我们认为,颗粒状PF皮层的进化是分阶段发生的,这些阶段伴随着其他进步,如视觉和手功能的发展。早期的前辈们的眼睛是向前看的,他们采用了新的移动、抓取和用手喂嘴的方式。在这些动物中出现了第一个颗粒状的PF区域,即PF尾部皮层和颗粒状的OFC。背侧、腹侧和极侧的PF皮层在类人猿灵长类动物中出现得更晚,因为它们变得更大,不得不与它们喜欢的食物严重短缺作斗争,更不用说竞争和捕食的威胁了。
\par 
第三-七章讨论第二个目标。这些章节解释了灵长类PF皮层各主要区域的连接如何允许它做它所做的事情,以及为什么只有它才能执行它的功能。我们逐个区域检查PF皮层,因为它的连接因区域而异:
\par 
1.第三章指出了内侧PF皮层与海马体、杏仁核和内侧前运动区之间的联系,这使得它能够更好地使用“内部”信号来指导在行动和行动规则之间的觅食选择,包括评估它们与所涉及的努力成本相关的当前价值。
\par 
2.第四章强调,眶侧PF皮层的连接使其能够更好地利用外部信号来指导觅食选择,在物体之间进行选择。眶PF皮层与许多感觉区域有联系,包括视觉、躯体感觉、味觉、嗅觉和内脏皮层。这些连接允许眶PF皮层发展有关特定行为结果的高维信息连接,并强调其视觉特征。我们认为,基于单个事件,细粒度OFC将特定结果分配给似乎导致它的特定选择。与杏仁核的相互联系根据当前需求提供了更新的结果评估。
\par 
3.第五章重点介绍了注意力和搜索功能。尾侧PF皮层从视觉区域接收到的连接传递了来自低阶和高阶视觉的信号,包括背侧和腹侧视觉流。基于与这些区域的皮质皮质连接以及控制脑干动眼肌核的皮质投射,尾侧PF皮层可以将显性和隐性注意力引导到潜在的行动目标上。我们认为,尾侧PF皮层(第8区),包括额叶眼场(FEF),参与了对学习产生的目标的搜索——目标导向的注意力——而不是反射性或刺激驱动的注意力。我们不认为FEF是眼球运动区或运动前区,而是将其视为前额叶皮层的一部分:一个将注意力导向有学习价值的物体和地方的部分,包括隐蔽注意力和显性注意力(以眼球运动的形式)。这样,它加强了中央凹和中央凹外信息的处理。
\par 
4.第六章强调了由背侧PF皮层产生目标,部分基于与后顶叶皮层的连接。这些投影提供了有关视觉事件的顺序、时间和位置的信息,这些信息构成了当前行为环境的重要部分。目标的产生也依赖于眶PF皮层关于与这些目标及其当前值相关的结果的信息。前额叶皮层的背侧提供了一种机制,可以消除以前事件记忆引起的干扰,它似乎是通过前瞻性地编码当前目标来做到这一点的,至少在一定程度上是这样。与前发动机区域的连接为实现这些目标提供了一条途径。
\par 
5.第七章提出,腹侧前额叶皮层根据视觉或听觉环境产生目标。它之所以能发挥这一功能,是因为它与下颞叶和上颞叶皮层、眶前皮层和杏仁核有联系。这些联系为它提供了视觉和听觉信号、预测的结果,以及根据当前生物需求对该结果的估值。符号由基本特征和整体对象之间的特征结合的中间层次组成。腹侧前额叶皮层使用同样的连接来应用抽象的规则和策略,从而将以前的经验转移到新的行为问题上。
\par 
在以这种方式将PF皮层拆开后,第八章将其重新组装起来。它将PF皮层作为一个整体来研究,并最终实现了我们的第三个目标:一个关于灵长类PF皮层基本功能的具体建议,重点是在灵长类中进化的区域。我们提出,颗粒状PF皮层生成的目标适合于当前环境和当前需求,并且它可以在单个事件的基础上做到这一点。因此,类人猿可以根据一个或几个经验解决广泛的问题,并且可以避免祖先强化学习机制中固有的许多错误。我们提出,在灵长类动物历史上的特定时间和地点(第2章和第8章),为了应对特定的适应压力,粒状PF区域进化为实现一种新的通用学习机制,这一机制增强了祖先的通用学习系统。祖先系统通过加强反馈来调整关联的强度,从而控制自动行为;灵长类动物解决问题的方式包括对行为的注意控制,以及更少的错误。
\par 
第9章讨论我们的第四个目标:解释在人类认知过程中观察到的基本PF功能是如何解释大脑激活的。它强调了重新再现的力量。因此,人类的PF皮层可以重新表示知觉状态、他人的意图和精神状态以及关系之间的关系。第9章还表明,人类的PF皮层阐述了PF皮层在其他类人猿中的功能;就像猴子的颗粒状PF皮层允许它们通过快速学习和抽象策略来避免错误一样,人类的PF皮层允许我们通过指令、模仿和心理试错行为来避免错误。
\par 
本章的剩余部分将讨论我们的第五个也是最后一个目标:将我们对灵长类动物PF皮层的描述与其他文献进行比较,并提出一些测试方法。因此,接下来的两个部分解释了其他的建议要么缺乏我们提供的进化视角,只适用于灵长类PF皮层的一部分,没有解释为什么它独特的连接组合解释了它的功能,没有说明灵长类PF皮层执行了哪些其他大脑区域不能执行的功能,未能解释PF皮层贡献的广泛行为,或者未能产生可验证的假设。
\par 
我们将可供选择的建议分为两组,分别在不同的部分进行讨论:一组主要依赖猴子的结果,另一组主要依赖人类的证据。当然,前一组的支持者也试图将其推广到人类,后一组的支持者也提到了他们从猴子身上得到的证据来支持他们的观点。所以我们这样划分只是为了方便,有时,当一个理论同时涉及猴子和人类的研究时,我们会在两个部分讨论。
\par 
我们遵循Wood和Grafman(2003)在表格中列出各种理论,并试图根据一套标准来评估它们。我们使用了五个标准,这表明一个成功的PF皮层理论应该:
\par 
1.结合PF皮层的进化史,特别是在灵长类动物中颗粒状PF皮层的出现和在高级类人猿中新颗粒状区域的出现:历史测试。
\par 
2.解释为什么PF皮层的连接解剖使其有可能执行所提议的功能:解剖测试。
\par 
3.识别PF皮层的特定功能,与大脑的其他部分形成对比:特异性测试。
\par 
4.考虑可用数据的广泛范围:通用性检验。
\par 
5.精确到可以被可行的观察所检验:可证伪性检验。
\section{基于猴子证据的理论}
我们自己的建议主要依赖于猴子的证据,因此我们首先考虑这类观点。在表10.1中,"x"表示我们认为列在该行的理论行的理论没有通过特定的测试。问号('?')表示该理论通过了测试、但我们不确定它是否成功地做到了这一点。当用于可证伪性检验时,"?"表示尽管该理论的某些版本没有通过,但其他版本可能通过、即使还没有人把该理论推向这个方向。破折号('-')意味着该理论理论根本不涉及该测试。
\par 
不可避免的是,填写表格需要主观的判断。考虑一下历史测试。如果一个理论对进化论只字不提,就可以认为它没有通过这个测试。然而,我们有时可以看到某个理论可以解决这个问题的方法。 然而,有些理论与进化论的观点根本不一致。一个理论没有通过历史当一个理论把非灵长类物种所共有的一些行为能力假设为灵长类PF皮层的基本功能时,它就不能通过历史的检验。然而,有些理论与进化论的观点根本不一致。一个理论没有通过历史 当一个理论把非灵长类物种所共有的一些行为能力假设为灵长类PF皮层的基本功能时,它就不能通过历史的检验。
\par 
我们也意识到,这些测试并不完全独立。例如,解剖学测试是通过评估一个理论是否具体说明了PF皮层的连接是如何让它做什么的 允许它做它所做的事情。特异性测试评估的是其他区域是否也能做同样的事情。同样的事情,在这样做的时候,我们有时会考虑到连接的问题。
\par 
我们首先讨论了工作记忆理论与所有五个标准的关系 因为它的影响非常大。读者会发现,许多相同的观点 也适用于其他理论。因此,为了避免在讨论这些其他观点时出现重复、我们集中讨论特定的优势或特定的弱点,而不一定要处理所有的测试。
\subsection{工作记忆}
第5章和第6章回顾了导致关注工作记忆是PF皮层的主要功能(如果不是唯一功能的话)的结果(Goldman-Rakic 1987)。多年来,这一观点一直主导着文献,支持者关注的是回溯性的感官记忆。像许多其他关于PF皮层的理论一样,它主要来源于少数几个任务,在这种情况下主要是延迟反应和延迟交替任务。我们理解这种强调,考虑到PF皮层损伤对这些任务造成的严重损害(第6章)。但我们认为工作记忆理论未能通过一个成功理论的许多测试,因此我们现在依次讨论这些缺陷。
\par 
在考虑工作记忆理论如何面对历史考验时,我们要承认普鲁斯和戈德曼-拉基奇(1991a, b)的开创性贡献。尽管戈德曼-拉基奇是工作记忆理论的强大而坚定的支持者,但她与普鲁斯发表的比较研究提供了一个关键的见解,彻底颠覆了该理论。猴子的典型工作记忆任务所依赖的区域,是颗粒状PF皮层的一部分,在非哺乳动物或丛林婴儿中都不存在。然而,这些动物拥有强大的工作记忆能力。
\par 
因此,从比较的角度来看,工作记忆理论是没有意义的。第一章解释了老鼠可以完成需要工作记忆的任务,如径向臂迷宫,在本章的后面我们会更详细地讨论这些争论。然而,像其他非灵长类哺乳动物一样,老鼠缺乏颗粒状的PF皮层。因此,人们必须假设,一旦在类人猿类灵长类动物中出现粒状PF皮层,它就“接管”了其他区域以前执行的功能。这种想法既不朴素,也不可信。老鼠有着复杂的大脑,能够解决工作记忆任务带来的简单问题,这并不奇怪。需要解释的发现是,有中外侧前额叶皮层病变的猴子无法解决这些简单的问题。我们稍后再讨论那个话题。
\par 
第二个测试是解剖学测试,要求工作记忆理论解释灵长类动物PF皮层的连接如何导致其独特的功能。在她对工作记忆理论的主要介绍中,戈德曼-拉基奇(1987)对当时人们所知的PF皮层的连接进行了广泛的回顾。在Wilson etal .(1993)中,她在比较PF皮层背侧和腹侧的功能时,也提到了顶叶和颞叶的连接。
\par 
根据最近的证据,我们建议重新定义这种区别,即后顶叶皮层不仅提供空间,而且还提供时间和其他输入到背侧PF皮层(第6章)。但我们同意与颞叶和后顶叶皮层的连接是PF皮层功能的一个重要方面。最后,Goldman-Rakic还讨论了pf皮质的细胞结构如何支持工作记忆(Goldman-Rakic1995;Constantinidis et al.2001),尽管这些想法并没有直接针对解剖测试。
\par 
第三个测试是特异性测试,正如所阐明的那样,工作记忆理论未能说明PF皮层与后顶叶皮层有何不同。这一理论的支持者非常强调在PF皮层中存在具有延迟期活动的细胞。然而,如Chafee和Goldman-Rakic(1998)等人所示,后顶叶皮层也存在延迟期活动的细胞。工作记忆理论的支持者可以提出后顶叶皮层的记忆编码依赖于PF皮层,他们可以通过指出冷却后顶叶皮层导致后顶叶皮层延迟相关活动减少的事实来支持这一观点(Chafee和Goldman-Rakic 2000)。但是,同样的实验表明,冷却后顶叶皮层对PF皮层的活跃度也有类似的影响,因此,这种论点是没有说服力的。如果修正该理论,提出工作记忆依赖于前额叶皮层和后顶叶皮层之间的信息循环,那么它就无法通过特异性测试。
\par 
第四个检验是一般性检验,它评估理论是否能解释所有可靠的数据。表8.1列出了任务和损伤效应的选择性列表,可以看出工作记忆理论无法解释许多已发表的结果。例如,腹侧PF皮层病变的猴子在去不去辨别任务(Iversen和Mishkin 1970)、同时匹配到前样例任务(Rushworth et al. 1997a)和条件视觉运动任务(Bussey et al. 2001)中表现出损伤。这些任务在提示的呈现和随后的动作之间都没有延迟期。然而,很明显,戈德曼-拉基奇和她的同事(Wilson et al. 1993)希望工作记忆理论至少适用于PF皮层的整个外侧表面,如果不是整个PF皮层的话。
\par 
该理论也未能解释在PF皮层中发现的许多细胞类型(表8.2)。正如其支持者所阐述的那样,工作记忆理论在很大程度上依赖于所谓记忆时期的细胞活动。在简单的任务中,如动眼肌延迟反应任务,许多细胞似乎对空间位置的记忆进行编码。当然,这一发现与理论是弱一致的(Funahashi et al. 1989)。但是,一项更仔细的对照研究表明,只有少数PF细胞编码一个被记住的位置,而不是一个被关注的位置(Lebedev et al. 2004)。因此,工作记忆理论也未能通过一般性检验。
\par 
此外,如果该理论适用于整个PF皮层,它在其他方面也未能通过一般性检验。Tsujimoto等人(2010)研究了猴子执行动眼力延迟反应任务时极缘PF皮层的细胞活动。在延迟期间,他们没有发现编码记忆地点或任何其他东西的活动。在Genovesio等人(2011)对中外侧和尾侧PF皮层的研究中,猴子判断相对距离,很少有细胞编码记忆位置,仅比预期的细胞多2 - 5%
\par 
最后一个考验是,一个有效的理论必须是可证伪的。在这里,我们输入一个关于工作记忆理论的查询,因为它可以用许多不同的方式表述。如果这个词仅仅是指在一段时间内记住先前线索的能力,那么它是可证伪的,是错误的。但是,正如第一章所解释的那样,巴德利和希契(1974)首次提出的工作记忆理论包括一个“中央执行人员”,而戈德曼•拉基奇(1998)在她的工作记忆理论的制定中也包括了中央执行人员。在这种情况下,没有任何观察能真正检验这一理论,因为几乎所有有趣的任务都涉及某种中央执行功能。
\par 
由于工作记忆理论未能通过通用性、特异性和历史检验,我们得出结论,工作记忆不是PF皮层的基本功能。
\subsection{跨时间偶然性}
Fuster(2008)提出,PF皮层在延迟期间整合信息方面发挥作用。他称这种关系为跨时间偶然性,因为它涉及跨越时间间隔整合信息。例如,对于延迟响应任务,正确的选择取决于在延迟期间之前发生的事件。所以跨时间整合和层次的概念抓住了PF皮层功能的这一方面。
\par 
在他的书中,Fuster(2008)回顾了他所看到的PF皮层的进化史。然而,他只处理了我们在第二章中提出的几个问题,我们不同意他的论点的关键方面。例如,福斯特认为啮齿动物、食肉动物和灵长类动物的PF皮层是同源的,这是我们所反对的观点。最重要的是,Fuster并没有具体说明PF皮层的各个部分赋予了在它们第一次出现时进化出这些结构的物种的优势。所以福斯特的理论没有通过历史的检验。
\par 
Fuster(2008)也回顾了PF皮层的连接。他认为PF皮层位于感知-行动循环的顶端,这一观点与我们关于层次结构的观点主要在特异性程度和术语上有所不同。因此,在这方面,他的理论解释了为什么PF皮层可以执行它所做的功能(解剖测试),而为什么其他区域不能(特异性测试)。
\par 
Fuster提供了病变和细胞活性数据的广泛回顾,因此他试图解释广泛的数据范围。然而,考虑到层次和“跨时间整合”的概念并不适用于所有的数据,他采用了各种解释,而不是像我们在第8章中所做的那样,就PF皮层的基本功能作为一个整体提出一个单一的建议。因此,福斯特得出结论,PF皮层在工作记忆、集合和行为抑制等方面发挥作用。我们在表中输入一个“X”,这意味着尽管他试图解释广泛的数据范围,但他没有提供超越层次结构调用的综合建议。
\subsection{计划和顺序}
到目前为止所讨论的理论强调了在延迟期间维护和集成信息。这些理论通常强调回溯性记忆和记忆中感官信号的维持。但是第6章和第7章回顾了PF皮层中延迟期活动对目标进行前瞻性编码的证据。前瞻编码是一种工作记忆,但不是PF皮层工作记忆理论支持者所设想的那种。
\par 
前瞻性编码使得Tanji和Hoshi(2008)强调了PF皮层在规划中的作用,尤其是在一系列目标的规划中。正如第6章所提到的,Mushiake等人(2006)发现了在解决视觉迷宫任务的一系列运动中为未来步骤编码的细胞。
\par 
在他们的综述中,Tanji和Hoshi引用了他们自己实验室的研究,表明PF皮层细胞编码目标序列的抽象结构(Shima et al. 2007)。这些作者和其他人一样,认为PF皮层在其顶端有一个层次结构。例如,他们表明,preSMA中的细胞编码特定序列的时间组织(Shima和Tanji 2000)。通过回顾这种能力背后的联系,Tanji和Hoshi的规划理论在通过解剖测试方面取得了一些进展。
\par 
Hoshi(2008)遵循同一路线,专门比较了PF皮层背侧和腹侧的细胞活性与运动前皮层背侧和腹侧的细胞活性以及初级运动皮层的细胞活性。他进一步讨论了这些区域之间的联系,并试图满足解剖学和特异性测试。
\par 
计划包括准备一系列的目标。因此,我们一致认为,目标、目标序列和目标序列的抽象的前瞻性捕获了PF皮层功能的一个重要方面。但作为PF皮层的理论,序列规划理论无法通过一般性检验。同样的结论也适用于任何过于依赖序列的理论。回想一下,在猴子中,颗粒状PF皮层的损伤会导致缺乏目标序列的简单任务受损,例如条件视觉运动任务(Bussey et al. 2001),匹配样本任务(Rushworth et al. 1997a),或没有延迟的去不去任务(Iversen和Mishkin 1970)。PF皮层在计划序列方面的理论并不能解释这些结果。
\par 
我们承认Tanji和Hoshi(2008)在他们的提案中包含了PF皮层有助于“执行控制”的许多方面的想法,包括对行动的选择性注意,有意行动的选择,以及行为规则的实施。然而,他们的方法面临着一个困难,即当人们将PF皮层的计划理论推广到包括执行控制的更多方面时,就很难知道什么是证伪。
\subsection{时间扩展事件}
Gaffan和他的同事认为,PF皮层处理并代表“暂时扩展”或“暂时复杂”事件(Browning和Gaffan 2008)。第8章解释了最初产生这一理论的具体实验。简单地说,它包括教猴子选择一张图片会导致另一张图片呈现2秒,然后才会提供奖励(见图8.8A)。在控制条件下,延迟不被屏幕上的任何刺激所填充。Browning等人发现,颞下皮层和PF皮层之间的连接断开的猴子在实验(填充延迟)条件下有损伤,但在对照组(未填充延迟)条件下没有损伤(见图8.8B)。只有前一种情况在奖励之前有一系列事件。
\par 
Wilson等人(2010)认为,这一理论提供了对PF皮层整体功能的解释,并将重点从模糊和不可测试的概念(如执行功能)转变为一种重要的表征性知识。我们同意后一点。我们也同意,学习在不同时间框架内展开的事件序列,抓住了PF皮层功能的一个重要方面,特别是因为它有助于上下文层次(第8章)。
\par 
然而,这一理论显然无法通过一般性检验,因为,正如前面提到的,根据任何常规分析,患有PF损伤的猴子在执行缺乏事件序列的任务时可能会受损。另一方面,如果一个人认为所有的任务都有一个事件序列,那么这个理论就变得不可证伪。
\subsection{条件性学习}
Goldman-Rakic(1987)和Passingham(1993)都强调,受PF皮层病变影响的任务往往具有条件性结构(第6-8章)。在条件性任务中,正确的选择取决于一个指令线索。以这种方式解释,这类任务不仅包括明显的任务,如有条件的视觉运动学习和成对的联想学习,还包括延迟反应任务。条件性学习理论抓住了这样一个事实:PF皮层在根据当前环境灵活地生成目标方面起着关键作用,不管是在延迟反应任务上有延迟,还是在条件性视觉运动学习任务上没有延迟。该理论是由 Passingham(1993)提出的理论,涉及到了解剖学测试。
\par 
然而,其他领域也关键性地参与到条件行为中。例如,运动前皮层的病变会导致条件性视觉运动任务的严重损害、前运动皮层的病变会导致条件性视觉运动任务的严重损害。(Halsband和Passingham 1985; Petrides 1987),而投射到运动前区的丘脑核的病变也是如此。投射到运动前区的丘脑核也是如此(Canavan等人,1989)。对灵长类PF的全面描述皮层的全面说明必须把它与其他区域的贡献区分开来。因此,条件性学习理论未能通过特异性测试。
\par 
该理论也未能通过通用性测试。它不能解释为什么PF皮层的病变会对上一节中描述的使用时间上延伸的事件的任务造成损害。那是一个简单的带有奖励延迟的辨别任务、 而不是一个条件性任务(Browning和Gaffan 2008)。
\subsection{行为抑制,反应抑制,抑制控制}
关于PF皮层(或其部分)的一些理论强调行为抑制、抑制控制或反应抑制(例如Roberts 和 Wallis 2000;Eagle et al. 2008)。例如,PF皮层病变的患者在需要进行抗眼跳时抑制proaccade的功能受损(Ploner et al. 2005),而腹侧PF皮层病变导致停止信号反应时间任务的停止功能受损(Aron et al. 2003)。这些发现与目标的产生和消除有关,但并不表明PF皮层的基本功能是抑制控制。
\par 
行为抑制理论通过了历史检验,因为它认为 PF 皮层进化为抑制自动或优势行为,包括本能行为。 它还通过了特异性测试,因为它说明了 PF 皮层所做的其他区域不做的事情。
\par 
因此,行为抑制的概念在我们的建议中占有一席之地。 当默认行为、习惯性行为或优势行为停止产生预期结果时,对这些行为的抑制一定是 PF 皮层所做工作的一部分(第 3 章和第 8 章)。 我们的提议强调了 PF 皮层的肯定功能,但它通过认识到专注的目标生成意味着抑制自动生成的目标(第 8 章)间接包含了这些负面功能。 默认或优势行为,如注视吸引注意力的刺激,需要被抑制以追求其他目标,如在反扫视任务中。 PF 皮层必须取消目标以及生成目标,并且它必须选择要避免的对象和位置以及要采取行动的对象和位置。 从这个意义上说,我们的建议包括行为抑制等负面功能,但它是隐含的而不是明确的。
\par 
然而,抑制理论未能通过普遍性检验,因为它忽略了 PF 皮层执行的肯定功能。 以反应和奖励的形式表达,该理论预测 PF 皮层的损伤会导致异常的持久性反应,不再产生奖励,称为坚持。 然而,证据表明,在做出无回报的选择后坚持过多并不比在做出有回报的选择后坚持过少更为突出。
\par 
例如,米尔纳 (Milner, 1963) 强调了她的观察结果,即额叶病变患者在威斯康星卡片分类任务中会犯持续性错误。 当 Mishkin (1964) 提出 PF 皮层损伤导致“中枢组持续存在”时,他接受了这个想法。但此类患者也会犯随机错误(Barcelo 和 Knight 2002)。 除了未能改变产生负面反馈的规则(坚持不懈)之外,他们还未能坚持产生积极反馈的规则(坚持不力)。
\par 
逆转任务也可以检验行为抑制理论,该理论预测 PF皮层损伤应该主要是在无奖励试验(负反馈)后增加错误。动作逆转任务的结果与这一预测相矛盾(见图3.8)。正常的类人猿会保持奖励行为,而内侧PF皮层受损的类人猿则较少这样做(Kennerley 等人 2006 年;Rudebeck 等人 2008 年)。 同样,在物体反转任务中(见图 4.8),有眼眶 PF 皮层损伤的类人猿使用正反馈的效率低下,但几乎可以正常使用负反馈(Rudebeck 和 Murray 2008)。 卡米尔等人。 (2011) 研究了具有可比病变的患者,并证实了两项任务的这些发现(见图 4.9)。
\par 
条件视觉运动任务也可以检验行为抑制理论。 第 7 章解释了一些类人猿在执行此任务时自发地采用了“改变-转移”和“重复-停留”策略 (Wise 和 Murray 1999)。 如果 PF 皮层的功能主要是抑制先前的反应,那么 PF 皮层损伤不应该对“停留”策略造成损害。 坚持先前的奖励反应应该受益于导致坚持不懈的病变。 事实上,图 8.6 显示,腹侧和眼眶 PF 皮层联合损伤的类人猿在“停留”和“转移”策略中表现出相同(和严重)的损伤。 抑制理论还预测 PF 皮层损伤应该避免持续的、过度学习的条件性视觉运动关联,也称为刺激-反应 (S-R) 关联或习惯(第 3 章)。 然而,腹侧和眼眶 PF 皮质的联合损伤会导致严重损伤(Bussey 等人,2001 年)。
\par 
同样根据抑制理论,三臂强盗任务中的高波动性条件(见图 4.5 和 4.6)应导致眼眶 PF 皮质损伤后的大损伤。 有缺陷的类人猿应该坚持以前的选择,而不是灵活地改变他们的选择。 然而,受损的类人猿通常会“追踪”或“匹配”不稳定的奖励概率(Walton 等人,2010 年)。
\par 
反向奖励权变任务提供了对行为抑制理论的直接检验。 在这项任务中,选择较少量的食物会产生较大的量,反之亦然。 根据该理论,PF 皮层的损伤应该会导致抑制优势反应的损害:获取更多食物。 事实上,眼眶 PF 皮层的损伤对学习这项任务没有影响 (Chudasama et al. 2007)。
\par 
该理论的支持者可能会争辩说,尽管有所有这些相互矛盾的证据,但其他结果似乎也支持它。去不去歧视任务提供了一个例子。 在这项任务中,一个物体指示类人猿做出“反应”,而另一个物体指示类人猿不做任何反应。 腹侧 PF 皮层受损的类人猿往往会在“不通过”试验中错误地“进行”(Iversen和 Mishkin 1970),称为委托错误。乍一看,这个结果似乎反映了抑制不适当反应的缺陷。然而,在Iversen和Mishkin使用的任务中,只有正确的“go”试验才会产生奖励,这意味着“going”成为默认响应。 如果没有特定的控制程序,类人猿会对“不进行”试验产生强烈的“进行”偏见,以便更快地进行下一次“进行”试验,从而获得下一次获得奖励的机会。
\par 
McEnenay 和 Butter (1969) 通过对类人猿进行一系列逆转测试来克服“去不去”任务中的这个缺陷,这建立了“不去”成为对刺激的默认反应的条件。 在实验者确定了默认行为后,OFC 受损的类人猿在下一次逆转后“走”的速度很慢。 总的来说,去不去歧视任务的结果表明,受伤的类人猿异常缓慢地改变他们的行为,但他们不支持反应抑制方面的解释。 相反,这些结果强调了 PF 皮层的肯定功能:快速学习(第 8 章)。
\par 
Jones 和 Mishkin (1972) 对眼眶和腹侧 PF 皮层进行了联合损伤,他们将坚持定义为一系列较长的试验,在逆转后表现低于机会水平。 他们发现了他们所期望的,行为抑制理论的支持者经常引用这一发现。 然而,Bussey 等人。 ( 2001 ) 进行了相同的损伤,发现快速学习和抽象策略应用(例如“lose-shift”)均受到严重损害。 和 Izquierdo 等人。 (2004) 观察到眼眶 PF 皮层损伤的类人猿未能形成逆转学习集。 第 8 章解释了快速学习、抽象策略和反转集都需要使用单个事件来生成目标。 从这个角度来看,琼斯和米什金获得的结果没有为行为抑制理论提供支持,他们也没有为第 8 章提出的肯定建议提供支持。
\par 
似乎支持行为抑制理论的另一个结果涉及狨猴。 腹侧 PF 皮层的损伤会导致超维转移任务受损(Dias 等人,1996 年)。 第 7 章解释说,在这项任务中,类人猿必须转变为关于哪个刺激维度与两种新刺激之间的选择相关的新规则(见图 7.10)。 迪亚斯等。 根据他们称之为注意力集的先前规则,根据未能抑制选择来解释他们的结果。 然而,PF 皮层损伤的影响不会持续存在:动物在学习第一次异次元转变后的第二次异次元转变中表现正常(Dias 等人,1997 年)。 因此,这些结果似乎反映了快速学习的肯定功能,尤其是在新情况下,而不是规则抑制的损害。
\par 
最后,灭绝任务的结果也被引用来支持行为抑制理论 (Butter 1969)。 眼眶 PF 皮层受损的类人猿对刺激的“反应”持续时间比正常类人猿更长,一旦这些动作不再产生奖励(见图 4.4)。 快速学习的障碍也是导致这一结果的原因。
\par 
因为它没有通过普遍性检验,我们可以拒绝行为抑制理论而不否认 PF 皮层在抑制功能中的重要作用。 除此之外,经常被引用以支持该理论的研究结果与类人猿通过快速学习和抽象策略(第 8 章)使用事件来减少错误的提议非常一致:一个肯定的功能而不是一个消极的功能。
\subsection{类别、规则和策略的抽象}
Miller 和他的同事提出,PF 皮层代表抽象类别 (Miller et al. 2003 ; Roy et al. 2010) 和抽象规则 (Miller 和 Buschman 2008)。 原始证据来自 Freedman 等人的研究。 ( 2002 ) 关于视觉分类,Nieder 等人。 ( 2002 ) on number, and of White and Wise ( 1999 ) and Wallis et al. ( 2001 ) 关于抽象规则。 与此同时,Genovesio 等人。 (2005) 强调了 PF 皮层在表示抽象策略中的作用。
\par 
我们已将这些数据纳入我们的提案,部分是通过在解释分层处理时诉诸抽象概念。 PF 皮层在上下文、目标和结果层次结构中的位置解释了它可以做什么而其他区域不能做什么。
\par 
但是 PF 皮层的纯抽象理论没有通过普遍性测试。 例如,如前所述,患有腹侧 PF 病变的类人猿在没有延迟的简单任务中表现出障碍 (Iversen 和 Mishkin 1970),该任务处理具体的刺激和反应。 它们还在简单的条件视觉运动映射上显示出巨大的缺陷,同样没有延迟,甚至在它们通过大量训练变得自动化之后也是如此(Bussey 等人,2001 年)。 这些有条件的视觉运动任务取决于涉及特定刺激和特定动作的具体刺激-反应关联。类人猿们在最近和遥远的过去都反复经历过这些映射。 它们不依赖于抽象规则、策略或类别。
\par 
此外,来自成像的证据表明,PF 皮层在生成具体和抽象目标方面发挥作用。 罗等人。 (2008) 特别比较了与选择具体手指运动和选择抽象规则相关的激活。 对于这两项任务,中外侧 PF 皮层的激活峰值没有差异。
\subsection{快速学习}
米勒和他的同事测试了猕猴快速学习的能力。 他们教授规则 (Miller 和 Buschman 2008)、提示-响应关联 (Cromer et al. 2011a) 和新类别 (Antzoulatos 和 Miller 2011)。 他们还比较了 PF 皮层或纹状体中的细胞编码选择的相对时间 (Miller和 Buschman 2008; Antzoulatos和 Miller 2011)。
\par 
很明显,米勒和他的同事们特别感兴趣的是快速学习,我们也有同样的兴趣。 然而,我们以不同的方式处理了这个问题,表明颗粒状 PF 皮层的损伤阻碍了快速学习(第 8 章)。我们的建议的不同之处在于强调通过单一事件学习的重要性,并将 PF 皮层功能与其他减少错误的方法联系起来,例如抽象规则和策略的应用。
\par 
就其本身而言,快速学习理论未能通过普遍性测试。 第 7 章和第 8 章解释了颗粒 PF 皮层的一个关键功能涉及抽象行为指导策略的应用。 类人猿必须学习这些策略,但一旦学会,它们只需要将它们应用到手头的问题上。 快速学习理论不解释策略任务的损伤,也不解释 PF 皮层损伤后的其他损伤,例如延迟反应任务。
\par 
米勒和他的同事们并没有专门针对历史测试,尽管他们可以这样做。 因此,我们就这一点提出疑问。
\subsection{分级处理}
我们的提议说,颗粒状 PF 皮层位于背景、结果和目标层次结构的顶端。 当然,其他人强调了 PF 皮层层次结构的重要性。 其中一些理论源自对人类的研究(Koechlin 等人 2003 年;Badre 2008 年),但该想法起源于Jones和Powell(1970 年),他们基于类人猿的皮质皮质连接。 这个想法影响了关于 PF 皮层的几种理论,这些理论通常指向抽象的逐渐增加,因为一个人在 PF 皮层内移动得更多(Badre 2008)。 然而,等级理论的侧重点有所不同。 一些指向域通用性(Wilson 等人 2010),另一些指向关系整合(Wendelken 等人 2008),还有一些指向产生目标选择的因素的复杂性和涉及的时间范围(Summerfield 和 Koechlin 2009) ). 任何 PF 皮层理论都可以包含层次结构的概念。
\par 
正如第 8 章提到的,Fuster (2008) 强调了 PF 皮层位于感知-动作层次结构的顶端这一观点。 他的理论假设了一系列行动途径,其中 PF 皮层形成了最间接的途径。 然而,Fuster 没有具体说明 PF 皮质功能与后顶叶皮质功能之间的差异,后顶叶皮质也参与感知-动作映射。 结果,他的理论没有通过特定城市的检验。 简单地提到层次结构并不能解释灵长类动物 PF 皮层做什么而大脑的其他部分不做,除了这些其他区域在较低的层次结构级别运行。
\par 
我们的提议明确指出,它位于处理层次结构的顶端,允许颗粒状 PF 皮层独特地整合从当前背景和事件中生成目标所需的所有信息,这在很大程度上基于对特定结果及其当前的知识 价值。 然后 PF 皮层向前运动皮层提供空间目标,前运动皮层计算运动计划以实现目标。 正如第 7 章和第 8 章所解释的,下颞叶皮层没有产生目标所需的特定结果信息,它也不直接投射到运动前皮层。 因此,与下颞叶皮层相比,颗粒状 PF 皮层具有所有需要的连接:访问前运动皮层、来自颞叶视觉区域的对象和提示信息、来自编码事件的海马体系统的输入,以及有关事件的结果信息 食物和液体的视觉特性,以及它们最新的生物学价值(第 4 章)。
\par 
同样的论点适用于后顶叶皮层、运动前皮层、颞叶皮层的其余部分,以及海马体的各种组合。 Fuster 没有说,在每种情况下,这些区域的功能有何不同,因为它们在感知-行动层次结构中的地位较低。 我们的提议具体说明了区别:颗粒状 PF 皮层具有所有这些必要的连接,而其他区域则没有。
\subsection{一体化} 
Miller 和 Cohen (2001) 已经强调,正如我们的提议一样,PF 皮层整合了来自所有感觉域的信息,以便选择未来的目标。 他们的理论比文献中的其他理论更像我们的提议。 在他们的评论中,Miller 和 Cohen 借鉴了类人猿研究和人类成像研究。 所以我们在人类研究部分再次讨论他们的想法。 在这里,我们关注主要来自类人猿研究的方面。
\par 
PF 皮层在类人猿中发挥整合功能的证据来自实验,例如 Rao 等人的实验。 ( 1997 ), 谁表明 PF 皮层中的细胞整合了来自背侧和腹侧视觉流的视觉信息。 当每种表示作为行动目标时,这些细胞编码空间和非空间视觉信息。
\par 
虽然我们承认他们的想法与我们的相似,但米勒和科恩 (Miller and Cohen) (2001) 在论文中提出的建议并未解决历史或特异性测试。 一个成功的理论应该解释为什么灵长类动物的 PF 皮层已经进化到可以做它所做的事情,或者为什么它的连接决定了如何单独做到这一点。 然而,我们在他们的提案中看不到任何内容,这会阻止对其进行详细说明以解决这些关键问题。
\par 
我们输入可证伪性测试的查询,因为集成功能的调用,如执行功能,可能非常模糊以至于无法进行测试。
\subsection{自适应编码,一般问题解决}
Duncan (2001) 提出,像我们一样,寻找 PF 皮层的单一、基本功能可能是错误的。 相反,他认为 PF 皮层可能有助于各种各样的认知功能,尤其是当任务变得困难时。
\par 
邓肯的大部分提议(他称之为多需求理论)都依赖于人类数据,因此我们将对他的想法进行更彻底的考虑推迟到下一节。 但是为了支持他的理论,他引用了类人猿细胞活动研究的证据,这些研究表明,很大一部分细胞在各种各样的任务中表现出与任务相关的活动。 我们知道,特定类别的经验会改变 PF 皮层中类别的表示(Roy 等人,2010 年),这里仅举一个 Duncan 称为自适应编码的例子。 Gaffan (2002) 也提出 PF 皮层可以被视为通用的问题解决者。
\par 
然而,仍有待证明 PF 皮层中的细胞比大脑其他部分(例如后顶叶、颞叶或海马回皮层)中的细胞具有更强的适应性。 它们可能会,但在某种意义上,大脑皮层的所有部分都在“学习”,因此没有一个区域可以垄断自适应编码。 它们在了解的内容上有所不同,在其他一些方面也有所不同,但 PF 皮层之外的自适应编码的缺失或缺乏仍有待证明。
\par 
尽管如此,邓肯和加凡都提出了几个有价值的想法。 自适应编码、一般问题解决和跨域处理的概念捕捉了一些关于颗粒 PF 皮层及其基本功能的重要信息。 我们认为我们的提案与这些想法一致,但相信它更具体和完整,部分原因是它解决了 PF 皮层的进化问题,部分原因是它解释了为什么 PF 皮层的连接解剖决定了它的作用。 换句话说,我们的理论通过了历史和解剖学测试,但邓肯和加凡的著作都没有涉及。
\par 
如果朝着那个目标发展,也许一般的问题解决和多需求理论可以通过这些测试。 在我们的提案(第 8 章)中,我们强调颗粒状 PF 皮质作为减少错误的适应性,并且它的许多区域在我们谱系历史的特定时间和地点进化。 例如,我们提出腹侧PF 皮层、背侧 PF 皮层,可能还有极地 PF 皮层随着类人灵长类动物体型的增加而进化,并开始依赖特别不稳定和竞争激烈的觅食环境中的食物资源。正如我们在本章前面和第 8 章所说,我们的提议表明,颗粒 PF 皮层的进化为类人猿提供了系统发育上新的通用问题解决者,以配合依赖于缓慢调整的较旧的通用学习系统 基于强化反馈的关联。
\subsection{动物学习论}
除了到目前为止提出的理论之外,还有另一个关于动物学习的更普遍的想法,可以用来解释灵长类动物 PF 皮层的基本功能。 动物学习理论认为,所有学习都源于建立和修改刺激、反应和结果之间的关联。 我们称之为祖先强化学习系统,我们认为灵长类动物 PF 皮层的进化主要是为了增强它。 旧的通用学习系统在动物的历史早期就已经发展起来,并为它们提供了很好的服务。 事实上,祖先的通用学习系统的力量解释了为什么灵长类动物以外的动物在没有颗粒状 PF 皮层的情况下也能相处得很好。
\par 
如果所有学习都取决于刺激、反应和结果之间的关联,那么所有 PF 介导的学习也必须与刺激、反应和结果之间的关联有关。 根据这种观点,除了感觉处理、概括和不同程度的联想复杂性之外,动物在行为能力上几乎没有差异。 该理论通过说灵长类动物 PF 皮层的贡献来自第 8 章讨论的那种综合功能来解决特异性测试。 一般来说,动物学习理论认为,PF 皮层允许灵长类动物做与其他动物相同的事情,而且做得更好。 例如,与其他哺乳动物相比,卡他林灵长类动物的三色视觉进化允许在特定光谱范围内进行更精细的颜色辨别。 否则,根据动物学习理论,鸽子、猪和人都以同样的方式学习。
\par 
当应用于 PF 皮层功能时,动物学习理论未通过可证伪性测试。 换句话说,因为动物学习理论是不可证伪的,所以它在 PF 皮层的应用同样是不可证伪的。 尽管任何行为都可以用刺激、反应和结果来描述,但这并不意味着所有动物都通过相同的学习机制来学习它们之间的关联。 事实上,将这些概念应用于行为的能力并不能说明潜在的学习机制。
\par 
此外,动物学习理论未能通过历史检验,因为它引用了一种功能,该功能早在哺乳动物的无颗粒 PF 皮层或灵长类动物的颗粒 PF 皮层出现之前就已进化。
\par 
然而,我们采用了动物学习理论的两个方面。 第 3 章解释说,在大鼠中,内侧 PF 皮层的某些损伤会导致结果导向行为受损,但习惯性行为完好无损。 内侧 PF 皮层其他部分的病变具有相反的效果。 因此,正如我们在第 3 章中讨论的那样,第一个在哺乳动物中进化的 PF 区域——无颗粒 PF 区域——调节通过祖先强化学习机制获得的行为。 因此,人们可以将无颗粒 PF 皮层视为哺乳动物祖先完全依赖强化学习系统与灵长类动物颗粒 PF 皮层后来增强之间的中间阶段。
\par 
其次,像 Balleine 和他的同事一样,我们也区分需要细心处理的行为和不需要细心处理的行为。 第 8 章论证了 PF 皮层在目标的专注生成过程中变得活跃。 然而,我们与动物学习理论家的不同之处在于拒绝将结果导向行为等同于对行为的细心控制。 事实上,Balleine 和 O’Doherty (2010) 甚至将所有行为分为习惯或结果导向行为,他们称之为目标导向行为。 他们将这个想法应用到人类身上。
\par 
然而,许多以结果为导向的行为是在没有注意到它们的情况下发生的。 序列行为已得到广泛研究,但也许最具启发性的例子涉及动物学习理论家青睐的学习类型:刺激-反应-结果 (S–R–O) 关联。
\par 
Johnsrude 和她的同事 (2000) 研究了健康人和眼眶 PF 皮层或杏仁核受损的患者。 受试者将注意力集中在一项任务上,该任务要求他们计算在给定的一系列试验中出现在不同选定位置的红点数量。 有时会出现黑点。 随着一个黑点或红点,一个视觉图案出现了,当这个点是红色的时候,受试者就可以获得奖励:糖果或葡萄干。 受试者不知道的是,各种模式在 10$\%$、50$\%$ 或 90$\%$ 的时间内与奖励相关。 后来,受试者表明他们在工具上习惯于选择高价值模式而不是低价值模式,尽管他们不知道自己为什么做出选择。 也就是说,他们不经意地学习了作为结果导向行为基础的 S-R-O 关联。 有趣的是,受试者为他们的选择编造了完全不相关的理由,比如“这个图案看起来很有趣”之类的话。 这一发现表明,结果导向行为不止一种。 从这个意义上说,动物学习理论未能通过普遍性检验,因为它无法解释这些不同种类的结果导向行为。
\subsection{概括}
刚刚调查的理论都没有像第 2 章那样解决 PF 皮层的进化问题。 尽管一些理论的支持者回顾了 PF 皮层的连接,但他们很少这样做是为了说明为什么只有 PF 可以做它所做的事情,就像我们在第 3 章到第 8 章中所做的那样。 许多理论并没有像我们所做的那样解释 PF 皮层与大脑皮层其他部分的不同之处。 其他理论无法解释重要发现。 并且有些理论是如此普遍以至于它们无法被证伪。
\section{基于人类受试者证据的理论}
表 10.2 列出了在很大程度上依赖于人类研究的理论,并根据表 10.1 中使用的相同标准进行了评估。 除了以一种敷衍的方式比较类人猿和人类而不参考他们的共同祖先之外,没有一个理论专门针对历史测试。 然而,我们冒昧地指出了提议的功能在哪些地方无法通过历史测试,甚至可以想象。
\subsection{监控}
Petrides (1994) 和我们一样认为,工作记忆理论未能捕捉到 PF 皮层的基本功能。 相反,他和他的同事针对 PF 皮层的作用提出了一个两阶段模型 (Owen et al. 1996a)。 该理论的一部分认为中外侧 PF 皮层的功能是监测记忆中的项目,另一部分提出腹侧 PF 皮层控制从长期记忆中非自动检索项目。
\par 
监视的概念来自有序的对象任务,也称为自排序任务或主题排序任务。 在给人们的任务的图片版本中,受试者必须以他们选择的任何顺序指向图片,唯一的规则是他们不能在一次试验中两次指向同一张图片。 这条规则意味着受试者必须建立对他们到目前为止所选择的图片的记忆,以便识别那些仍然可用于当前选择的图片。 Petrides 建议此过程涉及监控内存中的项目列表。
\par 
术语监控也可以应用于 n-back 任务。与有序对象任务一样,受试者看到或听到一系列项目,例如字母,他们必须监视它们在一系列中的位置。 因此,内存中项目的监控不同于内存中项目的维护,因为项目在内存中以某种方式被标记或区分。 在我们看来,关键是所有这些任务都涉及秩序。
\par 
该理论的任何表述都没有以任何严肃的方式解决历史检验问题,但 Champod 和 Petrides(2007 年)试图解决特异性问题。 他们声称中外侧 PF 皮层的功能是监控,而后顶叶皮层的功能是处理记忆中的项目。
\par 
这种解释有两个问题。 首先,成像数据显示在两种情况下两个区域都有激活,尽管程度不同。 其次,Postle 等人。 (2006) 将 rTMS 应用于 PF 皮层,这种临时损伤破坏了记忆中的项目操作。 因此,监控理论未能通过普遍性检验。
\par 
正如所阐述的那样,该理论解决了解剖学测试,因为它根据连通性解释了 PF 皮层功能的两个阶段——监测和检索。 但它并没有像我们的建议那样解释为什么 PF 皮层的连接决定了它独特的功能。 中外侧 PF 皮层的连接无法区分监控或操纵记忆中的项目。 因此,我们为此测试输入一个查询。
\subsection{积极维护记忆中的项目}
Ungerleider (1995) 提出中外侧 PF 皮层在信息的主动维护中发挥作用。 酒井等。 ( 2002a ) 采用这个术语来解释中外侧 PF 皮层中延迟周期激活在保护空间项目免于分心方面的影响。 对 PF 皮层功能的讨论通常会引用这个概念 (D’Esposito 2007)。
\par 
要具有任何价值,术语主动维护必须与被动维护形成对比,被动维护是一种不需要注意力的记忆维护。 双任务范式可以操纵注意力来测试这种区别。例如,如果受试者必须参与另一项涉及发音的任务,则受试者在记住一系列字母或数字时会出错,这会扰乱“语音循环”中的复述 (Baddeley 1986)。 以同样的方式,人们可以通过要求受试者对不相关的目标进行扫视来干扰空间项目的排练,这会破坏工作记忆的“视觉空间暂存器”中保存的项目(Guerard 等人,2009 年)。 在任何一种情况下,记忆都会因为竞争任务的需求受到干扰而受到影响。 因此保养可以说是用心或主动。
\par 
我们承认,人的 PF 皮层可能有助于记忆中项目的主动排练,第 6 章介绍了这种排练。 然而,作为 PF 皮层的理论,主动维护理论无法解释大部分数据。 它狭隘地关注与工作记忆理论相同的观察结果,因此未能通过普遍性检验。 事实上,主动维护理论与工作记忆理论差别不大,因此无法通过相同的测试。 早些时候,我们解释说,类人猿的 PF 皮层损伤会导致许多任务受损,而这些任务几乎不需要主动维护或工作记忆。 此外,许多众所周知的 PF 损伤后的损伤与主动维持无关,例如单词回忆或情景记忆检索的损伤。
\par 
对于特异性测试,我们输入一个查询。 该理论提出,PF 皮层,而不是后顶叶皮层,对于主动维护至关重要。 但它并没有对这些差异提供有说服力的说明。
\subsection{执行控制和目标维护}
在他们的 PF 皮层理论中,Miller 和 Cohen (2001) 像我们一样拒绝工作记忆理论。 我们在这里再次讨论他们的理论,除了前面关于类人猿的部分中的讨论之外,因为它在很大程度上依赖于应用于人类受试者的 Stroop 任务。 Miller 和 Cohen 提出 PF 皮层的功能是“积极维护代表目标和实现目标的手段的活动模式”。 该理论表明,PF 皮层通过施加自上而下的偏差对其他区域施加执行控制。 Miller 和 Cohen 并不像我们的提案那样强调目标的产生,但人们可以将这一想法读入他们对文献的处理中。
\par 
在 Stroop 任务中,受试者看到用蓝色墨水拼写的“红色”等词。 在一种情况下,他们只需要大声朗读单词,而在另一种情况下,他们需要报告墨水的颜色。 在第二种情况下,自动或优势反应是说“红色”,从而读出这个词,这是不正确的。 正如米勒和科恩所说,在斯特鲁普条件下,受试者必须牢记规则,即报告墨水的颜色,这比仅仅阅读单词需要更多的注意力。 麦克唐纳等人。 (2000) 发现当受试者准备报告墨水颜色时中外侧 PF 皮层会激活,但当他们准备阅读墨水拼写的单词时则不会(自动响应)。
\par 
在 Stroop 任务中,规则的表示保存在内存中,用于阅读单词或报告墨水颜色。 Miller 和 Cohen 提出,PF 皮层根据所需任务对低阶机制施加自上而下的偏见,我们在第 5 章中回顾了这种偏见的证据。
\par 
很明显,米勒和科恩 (Miller and Cohen) (2001) 提出的理论在很多方面与我们的建议(第 8 章)相似。 事实上,我们依赖于许多相同的证据,因此人们会期望有相当大的相似性。
\par 
然而,我们注意到一些重要的差异。 首先,我们的理论坚定地将粒状 PF 皮层置于比较的角度; 米勒和科恩提出的理论则不然。 其次,我们的提议强调使用单一事件来指导行为。 我们认为这种能力提供了 PF 皮层赋予灵长类动物的关键适应性优势之一,这一点 Miller 和 Cohen 没有解决,至少没有直接解决。 我们将目标的表示(前额叶功能)与实现目标的方式分开,我们将其归因于前运动皮层。
\subsection{情节控制}
Koechlin 和 Summerfield (2007) 对比了他们所谓的情境控制和情景控制。 上下文控制是指特定上下文对适当动作的规范。 情节控制是指由于适用于特定情节或事件的规则而对该控制进行的修改。 例如,假设在别人家里的情况下,当有人敲门时,一个人不愿意去应门。 然而,如果房主之前曾要求该人应门,则此(询问)事件将建立一个临时规则,并且该人的行为将不同于上下文本身所规定的。 科奇林等人。 (2003) 报道了在情景控制期间 PF 皮层的激活,峰值激活靠近背侧 PF 皮层和腹侧 PF 皮层之间的边界。
\par 
在这种情况下,术语情节是指导致规则存储在内存中的事件,这种用法不同于第 8 章中术语事件的使用,后者将事件视为主体所做的事情以及发生的事情 结果。 因此,正如 Miller 和 Cohen 所建议的(2001 年),情景控制的概念类似于在记忆中积极维护目标或规则。
\par 
然而,Summerfield 和 Koechlin (2009) 进一步发展了他们的提议。 他们认为,沿着 PF 皮层的外侧和内侧部分存在从尾端到头端的层次结构,这与指定上下文或评估值的时间差异有关。 Grafman 和他的同事 (Krueger et al. 2007) 在事件频率方面提出了类似的建议。 在第 9 章中,我们对 PF 皮层内的尾端到喙端层次结构采取了不同的观点。 我们根据重新表示来解释这种层次结构,而不调用时间范围内的差异。 当然,许多平行的层次结构可以共存于 PF 皮层中,并且有可能通过重新表示出现长时域。
\par 
情景控制理论不涉及历史测试。 但是,可以详细说明这样做,建议在我们这样做时随着新区域的发展添加额外的层。 它也没有解决解剖学测试问题,尽管未来的一些公式可能会这样做。
\subsection{操纵}
Postle等人 (1999) 引入了术语“操纵”来解释当受试者重新排列记忆中的项目时,他们观察到中外侧 PF 皮层的延迟期激活。 向受试者提供五个字母,在操作条件下,他们必须在延迟期间按字母顺序重新排序。 在此基础上,他们提出了 PF 皮层的操纵理论。
\par 
然而,在我们看来,他们的结果可能反映了对处理订单信息的依赖,而不是对内存中项目的操作。 在第 6 章中,我们建议当项目的顺序(无论是空间的还是非空间的)对执行任务至关重要时,激活发生在中外侧 PF 皮层。 n-back 任务就是一个例子。 我们还审查了当人们在刺激中产生新顺序时该区域激活的证据。
\par 
解决推理问题也可以说涉及对内存中项目的操作,而不是简单的维护。 我们承认,当人们进行类比推理时,中外侧 PF 皮层会发生激活(Prabhakaran 等人,1997 年)。 在此类任务中,例如 Raven 的渐进矩阵任务,受试者在尝试解决推理问题时会看到所有测试项目,因此该任务对感官短期记忆的负荷最小。 但是,与 Postle 和 D'Esposito 使用的任务一样,这类问题涉及序列,因此涉及顺序。 在这种情况下,空间秩序尤为重要。
\par 
然而,操纵理论未能通过普遍性检验。 可以在不一定涉及操作的任务上找到激活。 例如,我们从 Pochon 等人那里知道。 ( 2001 ) 当受试者准备回忆空间项目时,延迟期激活发生在中外侧 PF 皮层,即使他们不需要操纵记忆中的项目,因为他们稍后会按照呈现的顺序回忆它们。
\par 
PF皮层的操纵理论也未能通过解剖学测试,因为它没有提出操纵如何发生的解剖学解释,并且与源自人类研究的其他理论一样,它没有解决历史测试。
\subsection{注意选择}
Passingham 和 Rowe (2002) 提出了一项关于执行功能的具体建议,该建议与监督有关。 他们认为中外侧 PF 皮层在注意力选择中发挥作用。 在他们的第一个实验中 (Rowe et al. 2000),人类受试者在三个位置看到了提示,他们必须记住这些提示。 延迟一段时间后,出现一条线,受试者必须将光标移动到该线穿过的记忆提示位置。 为了完成这项任务,受试者必须在记忆中的三个位置中进行选择。 Passingham 和 Rowe 提出,他们通过使用注意力来增强记忆中相关位置的表示来做到这一点。
\par 
注意选择的概念与短期记忆中的监控项目非常相似。 两者都涉及标记或强调某些表示等。 注意选择一词的优点是,除了感官信息外,它还适用于目标的产生和实现目标所需的行动。 例如,当受试者需要生成一系列手指运动时,他们可以通过注意其中一个键、其中一个手指或键和手指的位置来实现。
\par 
尽管有其优势,注意选择理论未能通过普遍性检验。 正如 Goldman-Rakic 和 Leung (2002) 指出的那样,即使受试者不需要在记忆中的项目中做出选择,激活也会在回忆时发生在成像研究中。 在他们的影像学研究中,Leung 等人。 ( 2005 ) 通过简单地询问是否有一个探测项目在集合中来测试一组空间项目的记忆。 此任务不需要注意选择记忆中的项目,只需要完整地回忆一组位置。 我们不怀疑中外侧 PF 皮层在记忆项目中的注意选择中发挥作用,实际上第 6 章对这个想法做了很多讨论。 但是因为这个理论没有通过普遍性检验,它不能解释 PF 皮层的基本功能。 作者没有解决历史测试。
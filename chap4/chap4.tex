\chapter{眶额皮层:基于输出结果选择目的}
这本书提出了关于灵长类动物前额叶皮层基本功能的方案。

\section{概述}
眶额皮层有助于评估和选择对象未来行动的目标及其联系解释了其独特的功能。 它与嗅觉、内脏、味觉、体感和皮层的视觉区域,这些信号的结合产生了丰富的,特定结果的高维表示,一个由想象。 眶额皮层与杏仁核更新的相互作用根据当前的生物学需求评估这些成果。由于这些联系,看到有营养的食物或视觉标志与他们相关联唤起他们的味道和他们当前的动机价值。 虽然眶额皮层的很大一部分将物体链接到特定的结果,另一部分对一般结果这样做,在“共同货币”。 通过将行为结果分配给特定的觅食事件,而不是此类事件的平均值,眶额皮层灵长类动物提供了关键的选择优势。
\section{介绍}
第 3 章论证了内侧皮层的连接允许它做出选择在行动中,特别是当这些选择发生在没有外部感官的情况下时告诉动物该做什么的提示。 本章介绍眶额皮层。 它认为眶额皮层的连接允许它为外部提示做选择内侧皮层对“内部”提示的作用。\par
与内侧皮层一样,眶额皮层既有颗粒状部分,也有颗粒状部分。
第 2 章论证了早期哺乳动物进化出的颗粒状部分,而颗粒状部分
部分在早期灵长类动物中进化。 与第 3 章一样,我们需要利用老鼠的证据
深入了解颗粒状的。 原因是一样的:我们对灵长类动物的无颗粒皮层。 也像第 3 章一样,我们需要说什么是粒度部分眶额皮层可以做到其颗粒部分不能做到的。 我们采纳这个想法灵长类动物的眼眶额皮层使用单个事件将特定结果与导致这些结果的选择:一种刺激-结果关联的形式。\par
显然,大多数动物都可以学习刺激-结果关联。 巴甫洛夫条件反射依赖于它,就像仪器条件反射依赖于学习行动—结果关联一样。 除了无柄形式,我们假设所有动物物种都可以是仪器和经典条件。 所以我们面临一个挑战:我们需要说什么哺乳动物眶额皮层的颗粒部分以及颗粒部分是什么灵长类动物的眶 额皮层。\par
在本章中,我们区分了刺激-结果关联的两个方面。 首先
涉及刺激和结果之间的预测关系:可能性
给定刺激就会发生结果。 第二个涉及评估
结果在当前生物学需求方面的价值。 这两个方面,我们分别称为关联映射和动机评估,都会影响觅食选择,并且两者
需要随着情况的变化而更新。\par
\section{定义和术语}


\section{指纹}

\subsection{损伤和激活}

\subsection{损伤和活动}

\subsection{活动和激活}




\subsection{结论}



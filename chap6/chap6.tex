\chapter{背侧前额叶皮层:基于最近事件生成目标}
背侧PF皮层有助于根据顺序、时间和空间环境生成目标,它的连接解释了为什么只有它才能做到这一点。背侧PF皮层,包括中外侧PF皮层(46区),通过与后顶叶皮层、前运动皮层和PF皮层的其他部分连接来发挥作用。顶叶连接提供了许多用于生成目标的空间和时间背景。与运动前区域的联系导致这些目标的实现,通常是通过手的运动。与眶侧PF皮层的连接使背侧PF皮层能够根据单个事件预测目标选择的具体结果。背侧PF皮层位于背侧视觉流的末端,因此它可以规划目标序列,它可以具体或抽象地指定这些目标。在产生目标后,背侧PF皮层可以前瞻性地编码它们,直到行动的时候到来。鉴于背侧PF皮层在类人猿灵长类动物中进化(第2章),我们认为它在使用最近视觉事件的顺序、时间和位置来指导觅食选择和生成效率优化的目标序列方面具有优势。

\section{介绍}

\section{目的}

\section{定义和术语}


\section{指纹}

\subsection{损伤和激活}

\subsection{损伤和活动}

\subsection{活动和激活}




\subsection{结论}



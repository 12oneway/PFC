\chapter{人类前额叶皮层:从指令和想像中产生目标}
这本书提出了关于灵长类动物前额叶皮层基本功能的方案。

\section{概述}
本章的目的是:检验我们关于PF皮层基本功能的建议是否可以解释当人们执行复杂认知任务时在那里发生的成像激活。人类PF皮层中的许多激活似乎反映了对背景、目标和结果层次结构的阐述,从背景中生成目标,使用事件和抽象规则来选择目标,或目标的前瞻性编码。 其中一些激活发生在可能出现在类人猿或人类进化过程中的区域中,我们试图通过重新表示这一概念来解释这些。现有PF区域的扩展以及新区域的可能出现,导致人脑的大小和形状发生变化,连接方式也发生变化。我们提出了一个观点:现代人和猴子的前额叶皮层执行从他们最后的共同祖先继承的共同功能:它以减少错误的方式产生目标。在一个关键的进化过程中,人类前额叶皮层进一步减少了错误,因为人们可以从指令或模仿中学习,并且因为人们可以在采取任何行动之前进行心理试错行为。 得出结论:人类可能可以完全避免错误。
\section{介绍}
第8章提出PF皮层具有简单的基本功能。本章探讨该功能是否可以解释在受试者执行复杂认知任务时在人类PF皮层中看到的成像激活。
\par
读者会体会到我们这项工作的艰巨性。我们在第8章中提出的建议在很大程度上取决于来自猴子的数据。这些数据主要涉及地点和物体之间的选择,如通过伸手或眼球移动来实现的。说这些功能与类比推理或做出道德判断相去甚远,至少可以说是一种轻描淡写的说法。
\par
这个问题类似于O’Keefe和Nadel(1978)首次提出海马体具有导航功能时所面临的问题。当时,导航机制如何支持情景记忆似乎还是个谜,尽管从那时起,这种组合就变得不那么神秘了(Burgessetal.2002)。我们现在可以理解海马体在导航中的祖先作用如何在进化过程中得到详细阐述,以涵盖复杂的认知功能,如情景记忆。同样,本章试图根据为做出更好的觅食选择而进化的祖先机制的阐述来解释复杂认知任务的激活(第8章)。
\par
有足够的时间来详细说明。只需考虑一个事实,即旧大陆猴子和人类的最后一个共同祖先生活在23万年或更久以前(Ma)(Kay等人,2004年;Kumar等人,2005年)。这个事实意味着这两个世系已经分别进化了数千万年,在那段时间里,人类和猴子的大脑肯定都发生了变化。在此期间,人类的大脑变得比同样体重的猴子预期的大4.8倍(MacLeod等人,2003年)。与此同时,人脑发展出了自己的专业化,这在整本书中都有评论(Passingham2008)。
\par
本章首先简要总结了人类大脑在进化过程中大小和形状的变化,重点是颗粒状PF皮层。然后它处理连接和组织,包括新区域的可能出现。本章的其余部分解决了关键问题:第8章为类人前脑皮层提出的简单功能能否解释高级认知期间人类前脑皮层中发生的成像激活?
\section{人类进化中的额叶}
遗传证据表明,类人猿在5-7Ma(Kumar等人,2005年)与人类世系分道扬镳,而最早的化石可能是原始人Sahelanthropustchadensis(Guy等人,2005年),其年代约为7Ma(LeFur等人,2009年)。这种灵长类动物的颅骨容量约为360–370cm3,与小黑猩猩大致相同(Guyetal.2005)。现代人脑平均约为1350–1550cm3(Sowelletal.2007)。当然,人往往比黑猩猩大,但我们的大脑仍然比相同体型的假想类人猿的大脑大3.5倍(MacLeod等人,2003年)。
\par
所以原始人的大脑开始时很小,至少以现代人类的标准来看是这样。这段历史类似于第2章为类人猿大脑描绘的历史:在其早期进化历史中,大脑相对于身体尺寸而言相对较小,随后在其后期进化过程中大脑尺寸出现“等级增加”。
不幸的是,我们无法测量化石原始人额叶的大小。与第2章描述的化石类人猿大脑不同,人科动物硬脑膜的厚度阻止了在其头骨内表面形成清晰的脑沟印迹。所以我们只能评论大脑的整体形状。由此,我们知道在原始人进化过程中,前额叶变得更宽、更圆。
\par
福尔克等人(2000)制作了纤细而健壮的南方古猿的内脑模型,早期原始人的寿命约为1.5–2.5Ma。在“健壮”的南方古猿旁人属中,额叶具有相对尖的形状,类似于现代黑猩猩和大猩猩。在细长的南方古猿A.africanus中,额叶的形状略微更圆。我们很幸运有一个保存完好的南方古猿的头骨,它来自南非,年代约为2Ma(Carlsonetal.2011)。眼眶和极PF皮层的形状表明向更圆的形式过渡似乎在人属的原始人中。
\par
Bruner和Holloway(2010)测量了南方古猿额叶的最大宽度,并将这一结果与直立人和尼安德特人这两种后来进化的原始人的宽度进行了比较。相对于内脑的最大宽度,更晚近的原始人的额叶宽度超过了南方古猿。这一发现表明,在原始人类进化过程中,额叶的相对大小有所增加。
\par
现代人类可能是从与海德堡人(Homoheidelbergensis)有亲缘关系的原始人进化而来,有时也被称为古人类。来自埃塞俄比亚60万年前(Ka)的Bodo头骨和来自赞比亚300Ka的Kabwe头骨(Conroyetal.2000)来自这个物种。布克斯坦等人(1999)测量了这些头骨中大脑内壳前部的斜率,并将其与早期现代人类的两个头骨的斜率进行了比较:来自埃塞俄比亚的OmoI和OmoII头骨,日期为195Ka(1969年)。尽管古代人类头骨与早期现代人的头骨不同,眉骨较大,但内额脑壳的倾斜度并无不同。这一发现表明额叶的形状在古代人类中达到了现代状态,~300-600Ka。当然,形状本身告诉我们关于PF皮层的信息很少,但这些发现表明它的形状在我们最近的进化中相对较早地稳定下来。
\par
研究结果表明,在早期现代人类发生一项关键技术革命之前的某个时间,额叶已经发育良好。根据刀片技术和骨骼工具的证据,McBrearty和Brooks(2000)认为这些发展发生在~100Ka,但它们似乎直到很久以后才得到很好的确立(d'Errico&Stringer2011)。
\par
这些人的亲属在60–80Ka之前从非洲分散开来(Mellars2006),基因证据表明所有的亚洲人和欧洲人都是他们的后裔。这些祖先取代了尼安德特人,创造了更专业的工具包,并适应了世界上几乎所有的环境。
\section{现代灵长类动物的PF皮层大小}
人们只能从化石中获得关于额叶进化的线索,因为形状和大小告诉我们关于PF皮层的组织或功能的信息太少了。当然,颗粒状PF皮层仅构成额叶的一部分。
第2章解释了在类人猿进化过程中大脑尺寸的增加伴随着新的颗粒状PF区域的产生,特别是腹侧、背侧和极PF皮层。类似的事情也可能在人类进化过程中发生,研究现存灵长类动物的颗粒状PF皮层可以对这种可能性产生一些了解。不幸的是,尽管人们对人类大脑的进化给予了广泛关注,但将人类大脑与其他灵长类动物的大脑进行比较的文献并不支持像我们在第2章中回顾的类人灵长类动物那样可靠的结论。尽管如此,它确实支持一些初步的建议。
\par
Brodmann(1912)估计了颗粒状PF皮质的表面积,并将其与整个新皮质的表面积联系起来。正如Brodmann所见,颗粒状PF皮层占猕猴新皮层的11%,黑猩猩约17%,人类约28%(见图2.6)。
\par
这些差异非常大。如果Brodmann的数据是正确的,人类的PF皮层平均为34,770mm2,而黑猩猩为6719mm2。这使得它在人类身上大了约5倍,尽管一个典型的人的体重只比黑猩猩重10-20公斤。当人们认为初级运动皮层(区域4)在人类和黑猩猩中的大小略有不同时,差异就更加显着了(Preuss2011)。
\par
塞门德费里等人(2002)对Brodmann对黑猩猩的估计提出异议。他们估计位于类人猿中央前沟嘴侧的额叶皮层的百分比为26-30%,并发现它与人类相应的年龄百分比(29-33%)仅略有不同。但是,正如他们承认的那样,估计位于中央前沟头端的皮质百分比与估计颗粒状PF皮质的百分比之间存在差异。
\par
Passingham(2008)指出Bailey等人(1950)认为黑猩猩中央前沟的一些头侧皮质,区域FC,是颗粒状的而不是颗粒状的。从Brodmann(1912)的黑猩猩大脑图谱可以清楚地看出,他可能已将这个区域排除在他对颗粒状PF皮层的测量之外。这个因素可能部分地解释了Semendeferi等人给出的估计。
\par
埃尔斯顿等人(2006)使用Brodmann的数据来估计颗粒状PF皮层占据额叶的百分比。他们发现颗粒状PF皮层占人类额叶的80%,黑猩猩额叶的55%,长臂猿的53%,卡他林(旧世界)猴子的45-50%,41-46桔梗(新世界)猴子的%,链霉灵长类动物的41-43%(见图2.6E)。这些发现表明,颗粒状PF皮层在人类进化过程中显着扩大。
\par
最近对基因表达的分析对人类大脑如何在进化过程中变得如此大以及它们在胎儿发育过程中如何变得如此大产生了一些重要的见解。张等人(2011)检查了基因的位置和转录程度,在灵长类动物和啮齿动物分化后在灵长类动物中进化而来。图2.8以分支图的形式说明了这种分裂。他们称这种分裂后进化的基因为新基因,而不是从灵长类动物和啮齿动物的最后共同祖先那里继承的旧基因。与小鼠大脑相比,Zhang等人发现人类大脑发育过程中新基因的转录增强,其中大部分发生在新皮质中。年轻的基因编码许多转录因子,控制发育模式和速度。他们还展示了比旧基因更快地改变他们编码的氨基酸的证据。张等人得出结论,选择大脑功能某些方面的因素有助于年轻基因的起源。出于目前的目的,我们发现他们最感兴趣的发现是,在新皮质区域中,许多人类特异性基因的转录特别发生在PF皮质中。
\par
到目前为止,我们已经讨论了额叶的整体形状和颗粒状PF皮层的大小。但这些数据无法告诉我们特定区域的情况。极地PF皮层可能是最早出现在类人猿中的区域之一(第2章)。它在类人猴中很小,但它已成为人类额叶中最大的细胞构造区域(Öngüretal.2003)。极有可能是原始人进化过程中极地PF皮层的扩张导致了前面提到的形状变化。古人类头侧额骨的圆形和加宽可能意味着极PF皮层的扩张达到其现代状态300-600Ka。
\par
塞门德费里等人(2001)估计了现代类人猿和人类极地PF皮层(区域10)的范围。相对于整个大脑,与黑猩猩相比,极地PF皮层大约是人类大脑的两倍(图9.1)。
\par
图9.1人类极地PF皮层(区域10)的扩展。选定灵长类动物的分支图,左侧(箭头)的近似发散时间为数百万年(Ma)。每个圆圈的直径编码极PF皮层的大小参数,由底部的比例给出,按每个分支图下方显示的灰度代码分类。插图显示了人类极地PF皮层的位置。左侧,内侧视图,嘴侧向右,背侧向上;顶部,右侧,侧视图,嘴侧向左,背侧向上;右下角,腹面观,嘴向左,侧面向上。从TsujimotoS、GenovesioA、WiseSP修改而来。认知科学趋势15:169–76,©2011,经Elsevier许可。
\par
图9.2。在选定的灵长类动物群体中,极PF皮层体积的回归作为脑体积的函数。实线:来自猿类数据的回归;虚线:外推到人类大脑的大小。顶部:线性刻度。底部:对数刻度,单位斜率由虚线表示。缩写:G,长臂猿;A,类人猿;哈,人类。修改自SemendeferiK、ArmstrongE、SchleicherA、ZillesK、VanHoesenGW。人类和类人猿的前额皮质:区域10的比较研究。美国体质人类学杂志114:224–41,©2001,JohnWileyandSons。
\par
同一作者对第13区进行了类似的分析,发现与其他类人猿相比,人类没有这种扩张。这些发现支持了极地PF皮层在人类进化过程中显着扩展的结论。
然而,Holloway(2002)淡化了这些结果。他指出,人脑极PF皮层的大小仅比大脑与人脑一样大的猿类的预测大6%。图9.2显示了极PF皮层大小与类人猿大脑大小的对数对数回归(Semendeferi等人,2001年),它表明人类价值仅位于回归线上方。
\par
但是Holloway(2002)假设如果人类极PF皮层符合对数-对数回归线,或者接近符合,那么它具有相等的容量。这种信念忽略了一个事实,即对数-对数回归线的斜率为1.6,大大超过了单位斜率(1.0)。陡坡意味着较大的大脑比较小的大脑具有更大比例的极PF皮层。所以我们不能假设有类似的容量。
神经科学以外的一个例子解释了这一点(Gould1973)。已灭绝的爱尔兰麋鹿有巨大的鹿角,但当将它们的长度与衡量体型(肩高)进行对比时,它们的长度接近回归线。因为斜率超过了统一,所以它们的鹿角占身体的比例更大。因此,它们的功能更有效,大概是在吸引配偶方面。因此,结构符合回归线的事实并不意味着等效的功能能力。
\par
\textbf{总结}
\par
人脑不仅仅是猕猴和人类最后一个共同祖先的大脑的放大版。大脑的某些部分比其他部分扩展得更多。当考虑作为一个整体的新皮层的一部分时,颗粒状PF皮层在人类中比在猕猴中大~2.5。
\par
然而,大小只能告诉我们一点点,所以我们接下来考虑人类和其他大脑在微观结构和内部连接方面的差异。
\subsection{微观结构和内部连接}
在他们对极PF皮层的研究中,Semendeferi等人(2010)发现,与类人猿相比,人类第3层细胞体之间的这一区域存在大量神经纤维网。换句话说,细胞体的间距更大。与猿类相比,这种解剖学特征可能反映出人类有更多的树突、树突棘和末端。这一特征指向颗粒PF皮层的综合功能的详细说明,一般而言,特别是极PF皮层。
\par
Elston(2001)测量了PF皮层第3层锥体细胞树突上的棘数。他发现人类大脑中的这些细胞比猕猴大脑中的细胞多70%。由于连接终止于脊柱,这一发现意味着与猕猴相比,每个细胞都可以在人类PF皮层中整合更多信息。
\par
Elston(2007)还绘制了各种灵长类动物(包括人类)的棘数与PF皮层大小的关系图。他发现PF皮层越大,树突棘的数量就越多。鉴于人类PF皮质的大尺寸,脊柱的数量与预期非常吻合。这似乎不是细胞大小增加的产物。
\par
内在联系在皮层下白质中运行,有两种类型。申克等人(2005)区分位于白质核心的长联合纤维和连接PF皮层回旋内相邻区域的较短纤维。他们比较了一些灵长类动物的这些值。在人类中,包含长关联纤维的白质体积与人类大脑的预测一致。然而,短纤维比人类预期的更广泛。
\par
\textbf{总结}
\par
在第8章中,我们回顾了证据表明,当一个人提升了针对上下文、目标和结果的各种处理层次时,刺的比例会增加。我们认为,这种解剖学特征允许大脑在越来越抽象的层次上形成表征。人类PF皮层可以将这种发展提升到一个新的水平。此外,人类PF皮层似乎有大量短纤维将一个PF区域连接到另一个。因此,人类PF皮层可能特别适合整合信息。在后面的部分中,我们提供证据表明PF皮层中的一些激活反映了整合来自不同认知领域的信息的能力。
\subsection{外部连接}
上一节考虑了连接数,但没有考虑远程连接的整体模式。最近的研究利用了水沿着轴突在外部和内部扩散的事实(Basser&Ozarlsan2009)。这一特性催生了一种研究人脑连接的方法,称为扩散张量成像(DTI)。施马曼等人(2007)使用这种方法的修改来绘制猴子的连接图,并发现类似于标准轴突纤维追踪方法的结果。然而,DTI有严重的局限性。这些方法永远不会像在猴子和其他动物身上使用的方法那样敏感,这些方法可以揭示微观水平上的联系,并且在足够重要的情况下,还可以揭示电子显微镜水平上的联系。它们不能像猴子身上使用的方法那样揭示投射的确切起源和终止。尽管如此,DTI数据提供了有关人类PF皮层连接的宝贵信息。
\par
克罗克森等人(2005)将PF皮质分为七个部分:背侧PF、腹侧PF、外侧眶PF、中央眶PF、内侧眶PF、前扣带回和扣带回。他们使用Hubbard等人设计的方法测量了种子区域与另一个区域连接的可能性。(2005),并发现人类PF皮层连接的一般模式与猴子相似。例如,后顶叶皮层与背侧PF皮层相连,而下层颞叶皮层与腹侧和眼眶PF皮层相连。在猴子和人类的大脑中,杏仁核都与眼眶PF皮层和前扣带回皮层相连。
同一组研究人员使用DTI将前扣带皮层分为九个区域(Beckmannetal.2009)。人脑的结果与猴子轴突纤维追踪研究的结果相似(Carmichael&Price1996)。例如,在内侧PF皮层中,膝前和膝下区域与两个物种的眼眶PF皮层、杏仁核、下丘脑和腹侧纹状体的联系最强。
\par
DTI还表明,丘脑内侧核(MD)和PF皮质之间的整体连接模式在人类和猴子中是相似的(Kleinetal.2010)。MD的内侧部分与眼眶PF皮质相连,尾侧MD与内侧PF皮质相连,包括前扣带皮层,外侧MD与背侧PF皮质相连。
\par
据我们所知,人类和猕猴PF皮层连接的整体模式似乎相似。这种相似性大概反映了我们最后一个共同祖先的血统。然而,现代猕猴和人类已经分别进化了20到3000万年,这意味着在任何一个谱系中都可能发展出特化。在接下来的部分中,我们将指出人类大脑的三个可能的专业化:布罗卡区、极PF皮层的外侧部分和背侧扣带回皮层。
\subsection{布罗卡区}
布罗卡区,布罗德曼区44和45,位于腹侧运动前皮层(区域6)的前面。从尾端到头端,腹侧前运动皮层是无颗粒的,44区是异常颗粒状的,45区是颗粒状的(Petridesetal.2005)。在拓扑可比较的区域中,可以在猴子(图9.3)中发现相同的进展。
\par
克莱因等人(2007)使用DTI绘制了人脑中44和45区连接的整体模式。他们得出结论,区域44和45以其连接区分,与细胞结构定义的区域44和45非常吻合(图9.3)。DTI数据显示,在人脑中,44区与下顶叶皮层相连,45区与颞叶皮层相连(Freyetal.2008)。和克罗克森等人(2005)在猴子中使用DTI发现了相同的结果模式,Petrides和Pandya(2009)使用标准示踪剂也是如此。
\par
里林等(2008)以这一发现为起点,比较了人类、黑猩猩和猴子的布罗卡区。他们发现45区与中颞叶皮层的联系比44区更强。Rilling等人比较了黑猩猩和人类,发现人类的中颞叶皮层与布罗卡区的联系比黑猩猩的更广泛。最后,在人类中,这条通路在左半球比在右半球更大更广泛,而在黑猩猩中它是对称的。随着儿童的成长,这些连接的强度在左半球增加,但在右半球却没有(Paus等人,1999年)。
\par
图9.3(A)人脑中的布罗卡区。缩写:CS,中央沟;SF,外侧裂;6V,腹侧区域6。(B)猴子第4层的密度相对于相同三个额叶区域的平均值。经MacmillanPublishersLtd.许可转载。PetridesM、CadoretG、MackeyS.布罗卡区猕猴同系物的口面部躯体运动反应。自然435:1235-8,©2005,自然出版集团。
\par
布罗卡区在人类PF皮层左侧进化的想法当然不是什么新鲜事,但成像激活支持了这个想法。人们可以记住发音(Conrad1972)并且可以推理单词(Goel&Dolan2004)。当他们这样做时,成像激活往往发生在左半球。但是当他们处理视觉空间信息时,激活往往发生在右半球(Smithetal.1996)。尽管这里审查的比较证据还远非定论,但它似乎与在人类大脑中观察到的语言专业化大体一致。
\subsection{极地PF皮层}
对布罗卡区的研究表明,与人类和黑猩猩的最后共同祖先相比,人类的连通性发生了变化。此外,文献包含暗示性证据表明人脑中的两个区域在猴子中缺乏同系物:极PF皮层的外侧部分,10区的一部分,以及背侧扣带回皮层,32区的一部分。
\par
塞门德费里等人(2001)表明极PF皮层的外侧部分存在于类人猿和人类中,但不存在于猴子中。如果被接受,这个想法将意味着它是在类人猿和人类的祖先中进化而来的。然而,他们的结论完全取决于拓扑学和细胞构造证据。鉴于后者取决于主观标准,需要更多的证据来接受这个想法,尽管它似乎有道理。
\par
纳尔逊等人(2010)发现了支持证据,其形式为下后顶叶皮层中心的激活与外侧极PF皮层的激活之间存在相关性。火星等(2011)比较了人类和猴子的静息状态相关性。他们在中外侧PF皮层和极PF皮层之间的边界上选择了一个感兴趣区域。在人类中,该区域的静息激活与下顶叶皮层中央部分的激活相关,这与Nelson等人的发现一致。但是火星等人没有在猴子中发现相应的相关性,这一发现与轴突运输研究中两个区域之间缺乏联系一致(Petrides&Pandya2007)。这些发现与外侧极PF皮层(区域10)或与其相邻的区域在猴-猿分裂后进化的建议一致。但是,需要进一步的证据证明这种可能性。
\par
如果极PF皮层的外侧部分在人类和类人猿中进化,那么这个区域可以被视为位于处理层次结构的顶部。Summerfield和Koechlin(2009)为外侧PF皮层提出了一个从尾端到头端的层次结构,它们与指定动作的上下文的复杂性有关。和Koechlin等人(1999)已经建议最延髓的部分,外侧区域10,在处理次要目标时牢记主要目标。
\par
有限数量的证据支持这样一种观点,即极PF皮层的外侧部分(区域10)出现在人类和猿脑中,但不存在于猴脑中。它可能会在颗粒状PF皮层侧面的尾端到头端层次结构中创建一个新级别。



